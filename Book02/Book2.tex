%%%%%%
% BOOK 2
%%%%%%
\pdfbookmark[0]{Book 2}{book2}
\pagestyle{plain}
\begin{center}
{\Huge ELEMENTS BOOK 2}\\
\spa\spa\spa
{\huge\it Fundamentals of Geometric Algebra}
\end{center}\newpage

%%%%%%%
% Definitions
%%%%%%%
\pdfbookmark[1]{Definitions}{def2}
\pagestyle{fancy}
\cfoot{\gr{\thepage}}
\lhead{\large\gr{STOIQEIWN \ggn{2}.}}
\rhead{\large ELEMENTS BOOK 2}

\begin{center}
{\large Definitions}
\end{center}

1.~Any rectangular parallelogram is said to be contained by the two
 straight-lines containing the right-angle.
 
2.~And in any parallelogrammic figure, let any one whatsoever of the parallelograms about
its diagonal,  (taken) with its two complements, be called
a gnomon.

%%%%%%
% Prop 2.1
%%%%%%
\pdfbookmark[1]{Proposition 2.1}{pdf2.1}

\begin{center}
{\large Proposition 1$^\dag$}
\end{center}

If there are two straight-lines, and one of them is cut into any number of pieces whatsoever, then the rectangle contained by the two straight-lines is
equal to the (sum of the) rectangles contained by the uncut (straight-line), and every one of the pieces (of the cut straight-line).

\epsfysize=2.2in
\centerline{\epsffile{Book02/fig01e.eps}}

Let $A$ and $BC$ be the two straight-lines, and let $BC$ be cut, at random,
at points $D$ and $E$. I say that the rectangle contained by $A$ and $BC$ is equal
to the rectangle(s) contained by $A$ and $BD$, by $A$ and $DE$, and, finally, by $A$ and
$EC$.

For let $BF$ have been drawn from point $B$, at right-angles to $BC$ [Prop.~1.11],
and let $BG$ be made equal to $A$ [Prop.~1.3], and let $GH$ have been drawn through
(point) $G$, parallel to $BC$ [Prop.~1.31], and let $DK$, $EL$, and $CH$ have been drawn through (points) $D$, $E$, and $C$ (respectively), parallel to $BG$ [Prop.~1.31].

So the (rectangle) $BH$ is equal to the (rectangles) $BK$, $DL$, and $EH$. And
$BH$ is  the (rectangle contained) by $A$ and $BC$. For it is contained by $GB$ and $BC$, and $BG$ (is) equal to $A$. And  $BK$ (is) the (rectangle contained) by $A$ and $BD$.
For it is contained by $GB$ and $BD$, and $BG$ (is) equal to $A$. And $DL$ (is) the (rectangle contained) by $A$ and $DE$.
For $DK$, that is to say $BG$ [Prop.~1.34], (is) equal to $A$. Similarly, $EH$ (is)
also the (rectangle contained) by $A$ and $EC$. Thus, the (rectangle contained) by $A$ and $BC$ is
equal to the (rectangles contained) by $A$ and $BD$,  by $A$ and $DE$, and, finally,
by $A$ and $EC$.

Thus, if there are two straight-lines, and one of them is cut into any number 
of pieces whatsoever, then the rectangle contained by the two straight-lines is
equal to the (sum of the) rectangles contained by the uncut (straight-line), and every one of the pieces (of the cut straight-line). (Which is) the very thing it was required to show.
{\footnotesize \noindent$^\dag$ This proposition is a geometric version
of the algebraic identity: $a \,(b+c+d+\cdots) = a\,b + a\,c + a\,d + \cdots$.}

%%%%%%
% Prop 2.2
%%%%%%
\pdfbookmark[1]{Proposition 2.2}{pdf2.2}

\begin{center}
{\large Proposition 2$^\dag$}
\end{center}

If a straight-line is cut at random then the (sum of the) rectangle(s) contained by the whole
(straight-line), and each of the pieces (of the straight-line), is equal to the square on the whole.

\epsfysize=2.2in
\centerline{\epsffile{Book02/fig02e.eps}}

For let the straight-line $AB$ have been cut, at random, at point $C$. I say that the
rectangle contained by $AB$ and $BC$, plus the rectangle contained by 
$BA$ and $AC$, is equal to the square on $AB$.

For let the square $ADEB$ have been described on $AB$ [Prop.~1.46], and
let $CF$ have been drawn through $C$, parallel to either of $AD$ or $BE$ [Prop.~1.31].

So the (square) $AE$ is equal to the (rectangles) $AF$ and $CE$. And $AE$ is the square on $AB$.  And $AF$ (is) the rectangle contained by the (straight-lines) $BA$ and $AC$. For it is contained by $DA$ and $AC$, and $AD$ (is) equal to $AB$. 
And $CE$ (is) the (rectangle contained) by $AB$ and $BC$. For $BE$ (is) equal
to $AB$. Thus, the (rectangle contained) by $BA$ and $AC$, plus the
(rectangle contained) by $AB$ and $BC$, is equal to the square on $AB$.

Thus, if a straight-line is cut at random then the (sum of the) rectangle(s) contained by the whole
(straight-line), and each of the pieces (of the straight-line), is equal to the square on the whole. (Which is) the very thing it was required to show.
{\footnotesize \noindent$^\dag$ This proposition is a geometric version
of the algebraic identity: $a\,b+a\,c=a^2$ if $a=b+c$.}

%%%%%%
% Prop 2.3
%%%%%%
\pdfbookmark[1]{Proposition 2.3}{pdf2.3}

\begin{center}
{\large Proposition 3$^\dag$}
\end{center}

If  a straight-line is cut at random then the rectangle contained by the
whole (straight-line), and one of the pieces (of the straight-line),
is equal to the rectangle contained by (both of) the pieces, and the square on
the aforementioned piece.

\epsfysize=2.in
\centerline{\epsffile{Book02/fig03e.eps}}

For let the straight-line $AB$ have been cut, at random, at (point) $C$.
I say that the rectangle contained by $AB$ and $BC$ is equal to the
rectangle contained by $AC$ and $CB$, plus the square on $BC$.

For let the square $CDEB$ have been described on $CB$ [Prop.~1.46],
and let $ED$ have been drawn through to $F$, and let $AF$ have been
drawn through $A$, parallel to either of $CD$ or $BE$ [Prop.~1.31].
So the (rectangle) $AE$ is equal to the (rectangle) $AD$ and the (square) $CE$.
And $AE$ is the rectangle contained by $AB$ and $BC$.
For it is contained by $AB$ and $BE$, and $BE$ (is) equal to $BC$. And
$AD$ (is) the (rectangle contained) by $AC$ and $CB$. For $DC$
(is) equal to $CB$. And $DB$ (is) the square on $CB$. Thus, the rectangle
contained by $AB$ and $BC$ is equal to the rectangle contained by $AC$ and
$CB$, plus the square on $BC$.

Thus, if a straight-line is cut at random then the rectangle contained by the
whole (straight-line),  and one of the pieces (of the straight-line),
is equal to the rectangle contained by (both of) the pieces, and the square on
the aforementioned piece. (Which is) the very thing it was required
to show.
{\footnotesize \noindent$^\dag$ This proposition is a geometric version
of the algebraic identity: $(a+b)\,a = a\,b + a^2$.}

%%%%%%
% Prop 2.4
%%%%%%
\pdfbookmark[1]{Proposition 2.4}{pdf2.4}

\begin{center}
{\large Proposition 4$^\dag$}
\end{center}

If a straight-line is cut at random then the square on the whole (straight-line)
is equal to the (sum of the) squares on the pieces (of the straight-line),  and twice the
rectangle contained by the pieces.

\epsfysize=2.2in
\centerline{\epsffile{Book02/fig04e.eps}}

For let the straight-line $AB$ have been cut, at random, at  (point) $C$. I say that the square on $AB$
is equal to the (sum of the) squares on $AC$ and $CB$, and twice the rectangle
contained by $AC$ and $CB$.

For let the square $ADEB$ have been described on $AB$ [Prop.~1.46], and let $BD$ have been joined, and let $CF$ have been drawn through $C$, parallel to either of $AD$ or $EB$ [Prop.~1.31],
and let $HK$ have been drawn through $G$, parallel to either of $AB$ or $DE$ [Prop.~1.31].
And since $CF$ is parallel to $AD$, and $BD$ has fallen across them, the external
angle $CGB$ is equal to the internal and opposite (angle) $ADB$ [Prop.~1.29].
But, $ADB$ is equal to $ABD$, since the side $BA$ is also equal to $AD$ [Prop.~1.5].
Thus, angle $CGB$ is also equal to $GBC$. So the side $BC$ is equal to the side $CG$ [Prop.~1.6]. But, $CB$ is equal to $GK$, and $CG$ to $KB$ [Prop.~1.34]. Thus, $GK$
is also equal to $KB$. Thus, $CGKB$ is equilateral. So I say that (it is) also right-angled. For since $CG$ is parallel to $BK$ [and the straight-line $CB$ has fallen across them], the angles $KBC$ and $GCB$ are thus equal to two 
right-angles
[Prop.~1.29]. But $KBC$ (is) a right-angle. Thus, 
$BCG$ (is) also a right-angle.
So the opposite (angles) $CGK$ and $GKB$ are also right-angles [Prop.~1.34]. Thus,
$CGKB$ is right-angled. And it was also shown (to be) equilateral. Thus,
it is a square. And it is on $CB$. So, for the same (reasons), $HF$ is  also a square.
And it is on $HG$, that is to say [on] $AC$ [Prop.~1.34]. Thus, the squares $HF$ and
$KC$ are on $AC$ and $CB$ (respectively). And  the (rectangle) $AG$ is
equal to the (rectangle) $GE$ [Prop.~1.43]. And $AG$ is  the (rectangle contained)
by $AC$ and $CB$. For $GC$ (is) equal to $CB$. Thus, $GE$ is also equal to the
(rectangle contained) by $AC$ and $CB$. Thus, the (rectangles)
$AG$ and $GE$ are equal to twice the (rectangle contained) by $AC$ and $CB$.
And $HF$ and $CK$ are the squares on $AC$ and $CB$ (respectively). Thus, the
four (figures) $HF$, $CK$, $AG$, and $GE$ are equal to the (sum of the) squares on $AC$ and
$BC$, and twice the rectangle contained by $AC$ and $CB$. But, the  (figures) $HF$, $CK$, $AG$, and $GE$ are (equivalent to) the whole of $ADEB$, which is the square on $AB$. 
Thus, the square on $AB$ is equal to the (sum of the) squares on $AC$ and $CB$, and twice
the rectangle contained by $AC$ and $CB$.

Thus, if a straight-line is cut at random then the square on the whole (straight-line) is equal to the (sum of the) squares on the pieces (of the
straight-line), and twice the
rectangle contained by the pieces. (Which is) the very thing it was required to
show.
{\footnotesize \noindent$^\dag$ This proposition is a geometric version
of the algebraic identity: $(a+b)^2 = a^2 + b^2+ 2\,a\,b$.}

%%%%%%
% Prop 2.5
%%%%%%
\pdfbookmark[1]{Proposition 2.5}{pdf2.5}

\begin{center}
{\large Proposition 5$^\ddag$}
\end{center}

If a straight-line is cut into equal and unequal (pieces) then the rectangle contained by the
unequal pieces of the whole (straight-line), plus the square on the (difference) between the (equal and unequal)
pieces, is equal to the square on  half (of the straight-line).

\epsfysize=1.8in
\centerline{\epsffile{Book02/fig05e.eps}}

For let any straight-line $AB$ have been cut---equally at $C$, and unequally at $D$.
I say that the rectangle contained by $AD$ and $DB$, plus the square
on $CD$, is equal to the square on $CB$.

For let the square $CEFB$ have been described on $CB$ [Prop.~1.46], and
let $BE$ have been joined,
and let $DG$ have
been drawn through $D$, parallel to either of $CE$ or $BF$ [Prop.~1.31], and again let $KM$
have been drawn through $H$, parallel to either of $AB$ or $EF$ [Prop.~1.31], and again let
$AK$ have been drawn through $A$, parallel to either of $CL$ or $BM$ [Prop.~1.31].
And since the complement $CH$ is equal to the complement $HF$ [Prop.~1.43],
let the (square) $DM$ have been added to both. Thus, the whole (rectangle) $CM$ is equal to the whole (rectangle) $DF$. But, (rectangle) $CM$ is equal to (rectangle) $AL$, since $AC$ is
also equal to $CB$ [Prop.~1.36]. Thus, (rectangle) $AL$ is also equal to (rectangle) $DF$.
Let (rectangle) $CH$ have been added to both. Thus, the whole (rectangle)
$AH$ is equal to the gnomon $NOP$. But,  $AH$ is the (rectangle
contained) by $AD$ and $DB$. For $DH$ (is) equal to $DB$. Thus, the
gnomon $NOP$ is also equal to the (rectangle contained) by $AD$ and $DB$. Let
$LG$, which is equal to the (square) on $CD$,  have been added to both.
Thus, the gnomon $NOP$ and the (square) $LG$ are equal to the rectangle contained by
$AD$ and $DB$, and the square on $CD$. But, the gnomon $NOP$ and
the (square) $LG$ is (equivalent to) the whole square $CEFB$, which is on $CB$. Thus, the rectangle contained
by $AD$ and $DB$, plus the square on $CD$, is equal to the square on $CB$.

Thus, if a straight-line is cut into equal and unequal (pieces) then the rectangle contained by the
unequal pieces of the whole  (straight-line), plus the square on the (difference) between the (equal and unequal)
pieces, is equal to the square on  half (of the straight-line). (Which is) the
very thing it was required to show.
{\footnotesize \noindent$^\dag$ Note the (presumably mistaken) double use of the label $M$ in the Greek text.\\[0.5ex]
$^\ddag$ This proposition is a geometric version
of the algebraic identity: $a\,b +[(a+b)/2-b]^2 = [(a+b)/2]^2$.}

%%%%%%
% Prop 2.6
%%%%%%
\pdfbookmark[1]{Proposition 2.6}{pdf2.6}

\begin{center}
{\large Proposition 6$^\dag$}
\end{center}

If a straight-line is cut in half, and any straight-line added to it straight-on,
then the rectangle contained by the whole (straight-line) with the (straight-line) having being added, and the (straight-line) having being added,   plus the square on  half (of the original straight-line), is equal to the square on the
sum of  half  (of the original straight-line) and the (straight-line) having been added.

\epsfysize=1.8in
\centerline{\epsffile{Book02/fig06e.eps}}

For let any straight-line $AB$ have been cut in half at point $C$, and let any
straight-line $BD$ have been added to it straight-on. I say that the rectangle
contained by $AD$ and $DB$, plus the square on $CB$, is equal to the square on
$CD$.

For let the square $CEFD$ have been described on $CD$ [Prop.~1.46], and let $DE$ have been
joined, and let $BG$ have been drawn through point $B$, parallel to either of $EC$
or $DF$ [Prop.~1.31], and let $KM$ have been drawn through point $H$, parallel
to either of $AB$ or $EF$ [Prop.~1.31], and finally let $AK$ have been drawn
through $A$, parallel to either of $CL$ or $DM$ [Prop.~1.31].

Therefore, since $AC$ is equal to $CB$, (rectangle) $AL$ is also equal to (rectangle) $CH$ [Prop.~1.36]. But, (rectangle) $CH$ is equal to (rectangle) $HF$ [Prop.~1.43]. Thus, (rectangle) $AL$ is also equal to (rectangle) $HF$. Let (rectangle) $CM$ have
been added to both. Thus, the whole (rectangle) $AM$ is equal to the gnomon $NOP$.
But, $AM$ is  the (rectangle contained) by $AD$ and $DB$. For
$DM$ is equal to $DB$. Thus, gnomon $NOP$ is also equal to the
[rectangle contained] by $AD$ and $DB$.
Let $LG$, which is equal to the square on $BC$, have been added to both.
Thus, the rectangle contained by $AD$ and $DB$, plus the square on $CB$, is equal
to the gnomon $NOP$ and the (square) $LG$. But the gnomon $NOP$ and 
the (square) $LG$ is (equivalent to) the whole square $CEFD$, which is on  $CD$. Thus, the rectangle
contained by $AD$ and $DB$, plus the square on $CB$, is equal to the square on $CD$.

Thus, if a straight-line is cut in half, and any straight-line added to it straight-on,
then the rectangle contained by the whole (straight-line) with the (straight-line) having being added, and the (straight-line) having being added,   plus the square on  half (of the original straight-line), is equal to the square on the
sum of  half  (of the original straight-line) and the (straight-line) having been added. (Which is) the very thing it was required to show.
{\footnotesize \noindent$^\dag$ This proposition is a geometric version
of the algebraic identity: $(2\,a+b)\,b + a^2 = (a+b)^2$.}

%%%%%%
% Prop 2.7
%%%%%%
\pdfbookmark[1]{Proposition 2.7}{pdf2.7}

\begin{center}
{\large Proposition 7$^\dag$}
\end{center}

If a straight-line is cut at random then the sum of the squares on the
whole (straight-line), and one of the pieces (of the straight-line),  is equal to twice the rectangle
contained by the whole, and the said piece, and the square
on the remaining piece.

\epsfysize=2.2in
\centerline{\epsffile{Book02/fig07e.eps}}

For let any straight-line $AB$ have been cut, at random, at point $C$.
I say that the (sum of the) squares on $AB$ and $BC$ is equal to
twice the rectangle contained by $AB$ and $BC$, and the square on $CA$.

For let the square $ADEB$ have been described on $AB$ [Prop.~1.46], and let the (rest of)
the figure have been  drawn.

Therefore, since (rectangle) $AG$ is equal to (rectangle) $GE$ [Prop.~1.43],
let the (square) $CF$ have been added to both. Thus, the
whole (rectangle) $AF$ is equal to the whole (rectangle) $CE$.
Thus, (rectangle) $AF$ plus (rectangle) $CE$ is double (rectangle) $AF$.
But, (rectangle) $AF$ plus (rectangle) $CE$ is the gnomon $KLM$, and the
square  $CF$. Thus, the gnomon $KLM$, and the square $CF$, is double the (rectangle)
$AF$.
But double the (rectangle) $AF$ is also twice the (rectangle contained)
by $AB$ and $BC$. For $BF$ (is) equal to $BC$. Thus, the gnomon $KLM$, and
the square $CF$, are equal to twice the (rectangle contained) by $AB$ and $BC$.
Let $DG$, which is the square on $AC$, have been added to both.
Thus, the gnomon $KLM$, and the squares $BG$ and $GD$, are equal to
twice the rectangle contained by $AB$ and $BC$, and the square on $AC$.
But, the gnomon $KLM$ and the squares $BG$ and $GD$ is (equivalent to)
the whole of
$ADEB$ and $CF$, which are the squares on $AB$ and $BC$ (respectively).
Thus, the (sum of the) squares on $AB$ and $BC$ is equal to
twice the rectangle contained by $AB$ and $BC$, and the square on $AC$.

Thus, if a straight-line is cut at random then the sum of the squares on the
whole (straight-line), and one of the pieces (of the straight-line), is equal to twice the rectangle
contained by the whole,  and the said piece, and the square
on the remaining piece. (Which is) the very thing it was required to show.
{\footnotesize \noindent$^\dag$ This proposition is a geometric version
of the algebraic identity: $(a+b)^2+a^2 =2\,(a+b)\,a+b^2$.}

%%%%%%
% Prop 2.8
%%%%%%
\pdfbookmark[1]{Proposition 2.8}{pdf2.8}

\begin{center}
{\large Proposition 8$^\dag$}
\end{center}

If a straight-line is cut at random then four times the rectangle contained
by the whole (straight-line), and one of the pieces (of the straight-line), plus the square on the remaining piece,
is equal to the square described on the whole and the former piece, as
on one (complete straight-line).

For let any straight-line $AB$ have been cut, at random, at point $C$. I say that
four times the rectangle contained by $AB$ and $BC$, plus the square on $AC$,
is equal to the square described on $AB$ and $BC$,  as on one (complete straight-line).

For let $BD$ have been produced in a straight-line [with the straight-line $AB$],
and let $BD$ be made equal to $CB$ [Prop.~1.3], and let the square $AEFD$
have been described on $AD$ [Prop.~1.46], and let the (rest of the) figure
have been drawn double.

\epsfysize=2.5in
\centerline{\epsffile{Book02/fig08e.eps}}

Therefore, since $CB$ is equal to $BD$, but $CB$ is equal to $GK$ [Prop.~1.34], and $BD$ to
$KN$ [Prop.~1.34], $GK$ is thus also equal to $KN$. So, for the same (reasons), $QR$ is equal
to $RP$. And since $BC$ is equal to $BD$, and $GK$ to $KN$, (square) $CK$ is thus
also equal to (square) $KD$, and (square) $GR$ to (square) $RN$ [Prop.~1.36]. But,
(square) $CK$ is equal to (square) $RN$. For (they are) complements
in the parallelogram $CP$ [Prop.~1.43]. Thus, (square) $KD$ is also equal
to (square) $GR$. Thus, the four (squares)
$DK$, $CK$, $GR$, and $RN$ are equal to one another. Thus, the four (taken
together) are quadruple (square) $CK$. Again, since $CB$ is equal to $BD$, but
$BD$ (is) equal to $BK$---that is to say, $CG$---and $CB$ is equal to $GK$---that is
to say, $GQ$---$CG$ is thus also equal to $GQ$. And since
$CG$ is equal to $GQ$, and $QR$ to $RP$, (rectangle) $AG$ is also equal to
(rectangle) $MQ$, and (rectangle) $QL$ to (rectangle) $RF$ [Prop.~1.36]. But, (rectangle)
$MQ$ is equal to (rectangle) $QL$. For (they are) complements in the parallelogram $ML$ [Prop.~1.43]. Thus, 
(rectangle) $AG$ is also equal  to (rectangle) $RF$. Thus, the four (rectangles) $AG$, $MQ$, $QL$, and $RF$ are equal
to one another. Thus, the four (taken together)
 are quadruple (rectangle)
$AG$. And it was also shown that the four (squares)
$CK$, $KD$, $GR$, and $RN$ (taken together are) quadruple (square) $CK$. 
Thus, the eight (figures taken together), which comprise the gnomon $STU$,
are quadruple (rectangle) $AK$. And since $AK$ is the (rectangle contained)
by $AB$ and $BD$, for $BK$ (is) equal to $BD$, four times the (rectangle contained)
by $AB$ and $BD$ is quadruple (rectangle) $AK$. But the gnomon $STU$ was also shown (to be equal to) quadruple (rectangle)
$AK$. Thus, four times the
(rectangle contained) by $AB$ and $BD$ is equal to the gnomon $STU$. 
Let $OH$, which is equal to the square on $AC$,  have been added to both.  
Thus, four times  the
rectangle contained by $AB$ and $BD$, plus the square on $AC$, is equal to
the gnomon $STU$, and the (square) $OH$. But, the  gnomon $STU$
and the (square) $OH$ is (equivalent to) the whole square $AEFD$, which is on
$AD$. Thus, four times the (rectangle contained) by $AB$ and $BD$, plus
the (square) on $AC$, is equal to the square on $AD$.
And $BD$ (is) equal to $BC$. Thus, four times the rectangle contained by $AB$ and $BC$, plus the square on $AC$, is equal to the (square) on $AD$, that is to
say the square described on $AB$ and $BC$,  as on one (complete straight-line).

Thus, if a straight-line is cut at random then four times the rectangle contained
by the whole (straight-line), and one of the pieces (of the straight-line), plus the square on the remaining piece,
is equal to  the square described on the whole and the former piece, as
on one (complete straight-line). (Which is) the very thing it was required to show.
{\footnotesize \noindent$^\dag$ This proposition is a geometric version
of the algebraic identity: $4\,(a+b)\,a+b^2 = [(a+b)+a]^2$.}

%%%%%%
% Prop 2.9
%%%%%%
\pdfbookmark[1]{Proposition 2.9}{pdf2.9}

\begin{center}
{\large Proposition 9$^\dag$}
\end{center}

If a straight-line is cut into equal and unequal (pieces) then the (sum of
the) squares on the unequal pieces of the whole (straight-line) is double
the (sum of the) square on half (the straight-line) and (the square) on the (difference)
between the (equal and unequal) pieces.

\epsfysize=1.8in
\centerline{\epsffile{Book02/fig09e.eps}}

For let any straight-line $AB$ have been cut---equally at $C$, and unequally
at $D$. I say that the (sum of the) squares on $AD$ and $DB$ is double
the (sum of the squares) on $AC$ and $CD$.

For let $CE$ have been drawn from (point) $C$, at right-angles to $AB$ [Prop.~1.11], and let it be made equal to each of $AC$ and $CB$ [Prop.~1.3],
and let $EA$ and $EB$ have been joined. And let $DF$ have been drawn through
(point) $D$, parallel to $EC$ [Prop.~1.31], and (let) $FG$ (have been drawn)
through (point) $F$, (parallel) to $AB$ [Prop.~1.31]. And let $AF$ have been joined.
And since $AC$ is equal to $CE$,  the angle $EAC$ is also equal to the (angle) $AEC$ [Prop.~1.5]. And since the (angle) at $C$ is a right-angle, the (sum of the) remaining angles
(of triangle $AEC$), $EAC$ and $AEC$, is thus equal to one right-angle [Prop.~1.32].
And they are equal. Thus, (angles) $CEA$ and $CAE$ are each half a right-angle.
So, for the same (reasons), (angles) $CEB$ and $EBC$ are also each half a
right-angle. Thus, the whole (angle) $AEB$ is a right-angle. And since
$GEF$ is half a right-angle, and $EGF$ (is) a right-angle---for it is
equal to the internal and opposite  (angle) $ECB$ [Prop.~1.29]---the remaining (angle) $EFG$ is thus
half a right-angle [Prop.~1.32]. Thus, angle $GEF$ [is] equal to $EFG$.
So the side $EG$ is also equal to the (side) $GF$ [Prop.~1.6]. Again, since the angle
at $B$ is half a right-angle, and (angle) $FDB$ (is) a right-angle---for again
it is equal to the internal and opposite  (angle) $ECB$ [Prop.~1.29]---the
remaining (angle) $BFD$ is half a right-angle [Prop.~1.32]. Thus, the angle
at $B$ (is) equal to $DFB$. So the side $FD$ is also equal to the side $DB$ [Prop.~1.6]. And since $AC$ is equal to $CE$,  the (square) on $AC$ (is) also equal to the (square) on $CE$. Thus, the (sum of the) squares on $AC$ and $CE$
is double the (square) on $AC$. And the square on $EA$ is equal to the (sum
of the) squares on $AC$ and $CE$. For angle $ACE$ (is) 
a right-angle [Prop.~1.47].
Thus, the (square) on $EA$ is double the (square) on $AC$. Again,  since
$EG$ is equal to $GF$, the (square) on $EG$ (is) also equal  to 
the (square) on $GF$. 
Thus, the (sum of the squares)
on $EG$ and $GF$ is double the square on $GF$.
And the square on $EF$ is equal to the (sum of the) squares on $EG$ and $GF$
[Prop.~1.47]. Thus, the square on $EF$ is double the (square) on $GF$.
And $GF$ (is) equal to $CD$ [Prop.~1.34]. Thus, the (square) on $EF$ is double the
(square) on $CD$. And the (square) on $EA$ is also double the (square) on $AC$.
Thus, the (sum of the) squares on $AE$ and $EF$ is double the (sum of the)
squares on $AC$ and $CD$. And the square on $AF$ is equal to the
(sum of the squares) on $AE$ and $EF$. For the angle $AEF$ is a right-angle [Prop.~1.47]. Thus, the square on $AF$ is double the (sum of the squares) on 
$AC$ and $CD$. And the (sum of the squares) on $AD$ and $DF$ (is) equal to
the (square) on $AF$. For the angle at $D$ is a right-angle [Prop.~1.47].
Thus, the (sum of the squares) on $AD$ and $DF$ is double the (sum of the)
squares on $AC$ and $CD$. And $DF$ (is) equal to $DB$. Thus, the
(sum of the) squares on $AD$ and $DB$ is double the (sum of the) squares on
$AC$ and $CD$.

Thus, if a straight-line is cut into equal and unequal (pieces) then the (sum of the) squares on the unequal pieces of the whole (straight-line) is double
the (sum of the) square on half (the straight-line) and (the square) on the (difference)
between the (equal and unequal) pieces. (Which is) the very thing it was required to show.
{\footnotesize \noindent$^\dag$ This proposition is a geometric version
of the algebraic identity: $a^2+b^2 = 2[([a+b]/2)^2 + ([a+b]/2-b)^2]$.}

%%%%%%
% Prop 2.10
%%%%%%
\pdfbookmark[1]{Proposition 2.10}{pdf2.10}

\begin{center}
{\large Proposition 10$^\dag$}
\end{center}

If a straight-line is cut in half, and any straight-line added to it straight-on,
then the sum of the square on the whole (straight-line) with the (straight-line) having
been added, and the (square) on the (straight-line) having been added, 
is double the (sum of the square) on half (the straight-line), and
the square described on the sum of half (the straight-line) and (straight-line)
having been added, as on one (complete straight-line).

\epsfysize=2in
\centerline{\epsffile{Book02/fig10e.eps}}

For let any straight-line $AB$ have been cut in half at (point) $C$, and let any
straight-line $BD$ have been added to it straight-on. I say that the
(sum of the) squares on $AD$ and $DB$ is double the (sum of the) squares on $AC$ and $CD$.

For let $CE$ have been drawn from point $C$, at right-angles to $AB$ [Prop.~1.11],
and let it be made equal to each of $AC$ and $CB$ [Prop.~1.3], and let $EA$ and
$EB$ have been joined. And let $EF$ have been drawn through $E$, parallel to $AD$
[Prop.~1.31], and let $FD$ have been drawn through $D$, parallel to $CE$ [Prop.~1.31]. And since some straight-line $EF$ falls across the parallel straight-lines $EC$ and $FD$, the (internal angles) $CEF$ and $EFD$ are thus equal
to two right-angles [Prop.~1.29]. Thus, $FEB$ and $EFD$ are less than two
right-angles. And (straight-lines) produced from (internal angles whose sum is) less than
two right-angles meet together [Post.~5]. Thus, being produced in the direction of $B$
and $D$, the (straight-lines) $EB$ and $FD$ will meet. Let them have been 
produced,
and let them meet together at $G$, and let $AG$ have been joined.
And since $AC$ is equal to $CE$, angle $EAC$ is also equal to (angle) $AEC$ [Prop.~1.5]. And the (angle) at $C$ (is) a right-angle. Thus, $EAC$ and $AEC$
[are] each half a right-angle [Prop.~1.32]. So, for the same (reasons), $CEB$ and $EBC$ are
also each half a right-angle. Thus, (angle) $AEB$ is a right-angle.
And since $EBC$ is half a right-angle, $DBG$ (is) thus also half a right-angle
[Prop.~1.15]. And $BDG$ is also a right-angle. For it is equal to $DCE$.
For (they are) alternate (angles) [Prop.~1.29].
Thus, the remaining (angle) $DGB$ is half a right-angle. Thus,
$DGB$ is equal to $DBG$. So side $BD$ is also equal to side $GD$ [Prop.~1.6].
Again, 
since $EGF$ is  half a right-angle, and the (angle) at $F$ (is) a right-angle,
for it is equal to the opposite (angle) at $C$
[Prop.~1.34], the remaining
(angle) $FEG$ is thus half a right-angle. Thus, angle $EGF$ (is) equal to $FEG$.
So the side $GF$ is also equal to the side $EF$ [Prop.~1.6].
And since [$EC$ is equal to $CA$] the square on $EC$ is [also] equal to the
square on $CA$. Thus, the (sum of the) squares on $EC$ and $CA$ is 
double the square on $CA$. And the (square) on $EA$ is equal to the (sum of the
squares) on $EC$ and $CA$ [Prop.~1.47]. Thus, the square on $EA$ is double the
square on $AC$. Again, since $FG$ is equal to $EF$, the (square) on $FG$ is also
equal to the (square) on $FE$. Thus, the (sum of the squares) on $GF$ and
$FE$ is double the (square) on $EF$. And the (square) on $EG$ is equal to
the (sum of the squares) on  $GF$ and $FE$ [Prop.~1.47]. Thus, the (square)
on $EG$ is double the (square) on $EF$. And $EF$ (is) equal to $CD$  [Prop.~1.34].
Thus, the square on $EG$ is double the (square) on $CD$. But it was also
shown that the (square) on $EA$ (is) double the (square) on $AC$. Thus,
the (sum of the) squares on $AE$ and $EG$ is double the (sum of the) squares
on $AC$ and $CD$. And the square on $AG$ is equal to the (sum of the) squares
on $AE$ and $EG$ [Prop.~1.47]. Thus, the (square) on $AG$ is double the (sum of the squares) on $AC$ and $CD$. And the (sum of the squares) on $AD$ and $DG$ 
is equal to the (square) on $AG$  [Prop.~1.47].
Thus, the (sum of the) [squares] on $AD$ and $DG$ is double the
(sum of the) [squares] on $AC$ and $CD$. And $DG$ (is) equal to $DB$.
Thus, the (sum of the) [squares] on $AD$ and $DB$ is double the
(sum of the) squares on $AC$ and $CD$. 

Thus, if a straight-line is cut in half, and any straight-line added to it straight-on,
then the sum of the square on the whole (straight-line) with the (straight-line) having
been added, and the (square) on the (straight-line) having been added, 
is double the (sum of the square) on half (the straight-line), and
the square described on the sum of half (the straight-line) and (straight-line)
having been added, as on one (complete straight-line). (Which is) the
very thing it was required to show.
{\footnotesize \noindent$^\dag$ This proposition is a geometric version
of the algebraic identity: $(2\,a+b)^2+b^2= 2\,[a^2+(a+b)^2]$.}

%%%%%%
% Prop 2.11
%%%%%%
\pdfbookmark[1]{Proposition 2.11}{pdf2.11}

\begin{center}
{\large Proposition 11$^\dag$}
\end{center}

To cut a given straight-line such that the rectangle contained by the whole (straight-line), and one of the pieces (of the straight-line), is equal to the
square on the remaining piece.

\epsfysize=2.75in
\centerline{\epsffile{Book02/fig11e.eps}}

Let $AB$ be the given straight-line. So it is required to cut $AB$ such that the rectangle contained by the whole (straight-line), and one of the
pieces (of the straight-line), is equal to the square on the remaining piece.

For let the square $ABDC$ have been described on $AB$ [Prop.~1.46],
and let $AC$ have been cut in half at point $E$ [Prop.~1.10], and
let $BE$ have been joined.  And let $CA$ have been drawn through to (point)
$F$, and let $EF$ be made equal to $BE$ [Prop.~1.3]. And let the square $FH$ have
been described on $AF$ [Prop.~1.46], and let $GH$ have been drawn
through to (point) $K$. I say that $AB$ has been cut at $H$ such as to make the
rectangle contained by $AB$ and $BH$ equal to the square on $AH$.

For since the straight-line $AC$ has been cut in half at $E$, and $FA$ has been
added to it, the rectangle contained by $CF$ and $FA$, plus the square on $AE$, is
thus equal to the square on $EF$ [Prop.~2.6]. And $EF$ (is) equal to $EB$. Thus,
the (rectangle contained) by $CF$ and $FA$, plus the (square) on $AE$, is equal
to the (square) on $EB$. But, the (sum of the squares) on $BA$ and $AE$ is equal
to the (square) on $EB$. For the angle at $A$ (is) a right-angle [Prop.~1.47].
Thus, the (rectangle contained) by $CF$ and $FA$, plus the (square) on $AE$,
is equal to the (sum of the squares) on $BA$ and $AE$. Let the square on $AE$ have
been subtracted from both. Thus, the remaining rectangle contained by
$CF$ and $FA$ is equal to the square on $AB$.  And $FK$ is the (rectangle contained)
by $CF$ and $FA$. For $AF$ (is) equal to $FG$. And $AD$ (is) the (square) on $AB$. Thus, the
(rectangle) $FK$ is equal to the (square) $AD$. Let (rectangle) $AK$ have been
subtracted from both. Thus, the remaining (square) $FH$ is equal to 
the (rectangle) $HD$. And  $HD$ is the (rectangle contained) by $AB$ and $BH$.
For $AB$ (is) equal to $BD$. And $FH$ (is) the (square) on $AH$.
Thus, the rectangle contained by $AB$ and $BH$ is equal to the square on $HA$.

Thus, the given straight-line $AB$ has been cut at (point) $H$ such as to make
the rectangle contained by $AB$ and $BH$ equal to the square on $HA$.
(Which is) the very thing it was required to do.

{\footnotesize \noindent$^\dag$ This manner of cutting a straight-line---so
that the ratio of the whole to the larger piece is equal to the ratio
of the larger to the smaller piece---is sometimes called the ``Golden Section''.}

%%%%%%
% Prop 2.12
%%%%%%
\pdfbookmark[1]{Proposition 2.12}{pdf2.12}

\begin{center}
{\large Proposition 12$^\dag$}
\end{center}

In obtuse-angled triangles, the square on the side subtending the obtuse
angle is greater than the (sum of the) squares on the sides containing the obtuse angle
by twice the (rectangle) contained by one of the sides around the obtuse
angle, to which a perpendicular (straight-line) falls, and the 
 (straight-line) cut off outside (the triangle)
 by the perpendicular (straight-line) towards the obtuse angle.
 
 \epsfysize=1.8in
\centerline{\epsffile{Book02/fig12e.eps}}
 
 Let $ABC$ be an obtuse-angled triangle, having the 
 angle $BAC$ obtuse. And let $BD$ be drawn from point $B$, perpendicular
 to $CA$ produced [Prop.~1.12]. I say that the square on $BC$ is greater than the
 (sum of the) squares on $BA$ and $AC$, by twice the rectangle contained
 by $CA$ and $AD$.
 
 For since the straight-line $CD$ has been cut, at random, at point $A$, 
 the (square) on $DC$ is thus equal to the (sum of the) squares on $CA$ and $AD$,
 and twice the rectangle contained by $CA$ and $AD$ [Prop.~2.4]. Let
 the (square) on $DB$ have been added to both. Thus, the
 (sum of the squares) on $CD$ and $DB$ is equal to the (sum of the) squares
 on $CA$, $AD$, and $DB$, and twice the [rectangle contained] by $CA$ and $AD$.
 But, the (square) on
 $CB$ is equal to the (sum of the squares) on $CD$ and $DB$. For the angle at $D$ (is) a right-angle [Prop.~1.47]. And  the (square) on 
 $AB$ (is) equal to the (sum of the
 squares) on $AD$ and $DB$  [Prop.~1.47]. Thus, the square on $CB$ is equal to the (sum of the)
 squares on $CA$ and $AB$, and twice the rectangle contained by $CA$ and $AD$.  So the square on $CB$ is greater than the (sum of the) squares on $CA$ and
 $AB$ by twice the rectangle contained by $CA$ and $AD$.
 
Thus, in  obtuse-angled triangles, the square on the side subtending the obtuse
angle is greater than the (sum of the) squares on the sides containing the obtuse angle
by twice the (rectangle) contained by one of the sides around the obtuse
angle, to which a perpendicular (straight-line) falls, and the 
 (straight-line) cut off outside (the triangle)
 by the perpendicular (straight-line) towards the obtuse angle.
 (Which is) the very thing it was required to show.
{\footnotesize \noindent$^\dag$ This proposition is equivalent
to the well-known cosine formula: $BC^{\,2} = AB^{\,2} + AC^{\,2} -2\,AB\,AC\,\cos BAC$, since $\cos BAC = - AD / AB$.}

%%%%%%
% Prop 2.13
%%%%%%
\pdfbookmark[1]{Proposition 2.13}{pdf2.13}

\begin{center}
{\large Proposition 13$^\dag$}
\end{center}

In acute-angled triangles, the square on the side subtending the acute angle
is less than the (sum of the) squares on the sides containing the acute
angle by twice the (rectangle) contained by one of the sides around the acute
angle, to which a perpendicular (straight-line) falls, and the 
 (straight-line) cut off inside (the triangle)
 by the perpendicular (straight-line) towards the acute angle.
 
 \epsfysize=2in
\centerline{\epsffile{Book02/fig13e.eps}}

 Let $ABC$ be an acute-angled triangle, having the  angle at (point) $B$ acute.
 And let $AD$ have been drawn from point $A$, perpendicular to $BC$ [Prop.~1.12].
I say that the square on $AC$ is less than the (sum of the) squares on $CB$ and $BA$, by twice the rectangle contained by $CB$ and $BD$.

For since the straight-line $CB$ has been cut, at random, at (point) $D$, the
(sum of the) squares on $CB$ and $BD$ is thus equal to twice the rectangle contained by $CB$ and $BD$, and the square on $DC$ [Prop.~2.7]. Let the
square on $DA$ have been added to both. Thus, the (sum of the) squares on
$CB$, $BD$, and $DA$ is equal to twice  the rectangle contained by $CB$ and $BD$,
and the (sum of the) squares on $AD$ and $DC$. But, the (square) on $AB$ (is) equal to the (sum
of the squares) on $BD$ and $DA$. For the angle at (point) $D$ is a right-angle [Prop.~1.47]. And the (square) on $AC$ (is) equal to the (sum of the squares)
on $AD$ and $DC$ [Prop.~1.47]. Thus, the (sum of the squares) on $CB$ and $BA$
is equal to the (square) on  $AC$, and twice the (rectangle contained) by $CB$
and $BD$. So the (square) on $AC$ alone is less than the (sum of the) squares on
$CB$ and $BA$ by twice the rectangle contained by $CB$ and $BD$.

Thus, in acute-angled triangles, the square on the side subtending the acute angle
is less than the (sum of the) squares on the sides containing the acute
angle by twice the (rectangle) contained by one of the sides around the acute
angle, to which a perpendicular (straight-line) falls, and the 
 (straight-line) cut off inside (the triangle)
 by the perpendicular (straight-line) towards the acute angle. (Which is)
 the very thing it 	was required to show.
{\footnotesize \noindent$^\dag$ This proposition is equivalent
to the well-known cosine formula: $AC^{\,2} = AB^{\,2} + BC^{\,2} -2\,AB\, BC\,\cos ABC$, since $\cos ABC = BD / AB$.}

%%%%%%
% Prop 2.14
%%%%%%
\pdfbookmark[1]{Proposition 2.14}{pdf2.14}

\begin{center}
{\large Proposition 14}
\end{center}

To construct a square equal to a given rectilinear figure.

\epsfysize=2in
\centerline{\epsffile{Book02/fig14e.eps}}

Let $A$ be the given rectilinear figure. So it is required to construct a
square equal to the rectilinear figure $A$.

For let the right-angled parallelogram $BD$,  equal to
the rectilinear figure $A$, have been constructed  [Prop. 1.45]. Therefore, if $BE$ is equal to $ED$ then
that (which) was prescribed has taken place. For the square $BD$, equal to the rectilinear figure $A$,
has been constructed. And if not, then
one of the (straight-lines) $BE$ or $ED$ is greater (than the other). Let $BE$ be greater, and let it have been
produced to $F$, and let $EF$ be made equal to $ED$ [Prop.~1.3].
And let $BF$ have been cut in half at (point) $G$ [Prop.~1.10]. And, with center $G$, and
radius one of the (straight-lines) $GB$ or $GF$, let the semi-circle $BHF$ have been drawn.
And let $DE$ have been produced to $H$, and let $GH$ have been joined.

Therefore, since the straight-line $BF$ has been cut---equally at $G$,
and unequally at $E$---the rectangle contained by $BE$ and $EF$, plus the
square on $EG$, is thus equal to the square on $GF$ [Prop.~2.5]. And $GF$ (is)
equal to $GH$. Thus, the (rectangle contained) by $BE$ and $EF$, plus the
(square) on $GE$, is equal to the (square) on $GH$. And the (sum of the) squares on $HE$ and $EG$  is equal to the (square)
on $GH$  [Prop.~1.47].
Thus, the (rectangle contained) by $BE$ and $EF$, plus the (square) on $GE$, is
equal to the (sum of the squares) on $HE$ and $EG$.
Let the square on $GE$ have been taken from both. Thus, the remaining
rectangle contained by $BE$ and $EF$ is equal to the square
on $EH$. But, $BD$ is the (rectangle contained) by $BE$ and $EF$. For $EF$ (is)
equal to $ED$. Thus, the parallelogram $BD$ is equal to the square on $HE$.
And $BD$ (is) equal to the rectilinear figure $A$. Thus, the rectilinear figure
$A$ is also equal to the square (which) can be described on $EH$.

Thus, a 
square---(namely), that (which) can be described on $EH$---has been constructed, equal
to the given rectilinear figure $A$. (Which is) the very thing it was required to do.
\newpage
\thispagestyle{plain}
~\\