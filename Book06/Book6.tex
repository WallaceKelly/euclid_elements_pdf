%%%%%%
% BOOK 6
%%%%%%
\pdfbookmark[0]{Book 6}{book6}
\pagestyle{plain}
\begin{center}
{\Huge ELEMENTS BOOK 6}\\
\spa\spa\spa
{\huge\it Similar Figures}
\end{center}\newpage

%%%%%%%
% Definitions
%%%%%%%
\pdfbookmark[1]{Definitions}{def6}
\pagestyle{fancy}
\cfoot{\gr{\thepage}}
\lhead{\large\gr{STOIQEIWN \ggn{6}.}}
\rhead{\large ELEMENTS BOOK 6}

\begin{center}
{\large Definitions}
\end{center}

1.~Similar rectilinear figures  are those (which) have  (their) angles separately equal and the (corresponding) sides about the equal angles proportional.

2.~A straight-line is said to have been cut in extreme and mean ratio when
as the whole is to the greater segment so the greater (segment is) to the lesser.

3.~The height of any figure is the (straight-line)  drawn
from the vertex perpendicular to the base.

%%%%%%
% Prop 6.1
%%%%%%
\pdfbookmark[1]{Proposition 6.1}{pdf6.1}

\begin{center}
{\large Proposition 1}$^\dag$
\end{center}

Triangles and parallelograms which are of the same height are to one another as their bases.

\epsfysize=2in
\centerline{\epsffile{Book06/fig01e.eps}}

Let $ABC$ and $ACD$ be triangles, and $EC$ and $CF$ parallelograms, of the same
height $AC$. I say that as  base $BC$ is to  base $CD$, so  triangle
$ABC$ (is) to  triangle $ACD$, and  parallelogram $EC$ to  parallelogram
$CF$.

For let the (straight-line) $BD$ have been produced in each direction to points $H$ and $L$, and
let [any number] (of straight-lines) $BG$ and $GH$ be made equal to  base $BC$,
and any number (of straight-lines) $DK$ and $KL$ equal to  base $CD$.
And let $AG$, $AH$, $AK$, and $AL$ have been joined.

And since $CB$, $BG$, and $GH$ are equal to one another, triangles
$AHG$, $AGB$, and $ABC$ are also equal to one another  [Prop.~1.38].
Thus, as many times as  base $HC$ is (divisible by)  base $BC$, so many times is  triangle $AHC$ also (divisible) by  triangle $ABC$. So, for
the same (reasons), as many times as  base $LC$ is (divisible) by
 base $CD$, so many times is triangle $ALC$ also (divisible) by 
triangle $ACD$. And if  base $HC$ is equal to base $CL$ then
triangle $AHC$ is also equal to  triangle $ACL$  [Prop.~1.38].
And if  base $HC$ exceeds  base $CL$ then  triangle
$AHC$ also exceeds triangle $ACL$.$^\ddag$ And if ($HC$ is) less (than 
$CL$ then  $AHC$ is also) less (than $ACL$).
So, their being four magnitudes,  two bases, $BC$ and $CD$, and  two
triangles, $ABC$ and $ACD$, equal multiples have been taken of  base
$BC$ and triangle $ABC$---(namely), base $HC$ and triangle
$AHC$---and other random equal multiples of base $CD$ and 
triangle $ADC$---(namely),  base $LC$ and triangle $ALC$. And it has been
shown that if  base $HC$ exceeds base $CL$ then  triangle
$AHC$ also exceeds  triangle $ALC$, and if ($HC$ is) equal (to $CL$ then $AHC$ is also)
equal (to $ALC$), and if ($HC$ is) less (than $CL$ then $AHC$ is also) less (than
$ALC$). Thus, as base $BC$ is to base $CD$, so triangle $ABC$ (is) to
triangle $ACD$ [Def.~5.5].  And since parallelogram $EC$ is double  triangle $ABC$, and parallelogram
$FC$ is double triangle $ACD$  [Prop.~1.34], and parts have the same
ratio as similar multiples [Prop.~5.15], thus
as triangle $ABC$ is to triangle $ACD$, so parallelogram $EC$ (is) to parallelogram
$FC$. In fact, since it was shown that as base $BC$ (is) to $CD$, so
triangle $ABC$ (is) to triangle $ACD$,  and as triangle $ABC$ (is) to triangle $ACD$, so
parallelogram $EC$ (is) to parallelogram $CF$, thus, also, as base $BC$ (is) to
base $CD$, so parallelogram $EC$ (is) to parallelogram $FC$ [Prop.~5.11].

Thus, triangles and parallelograms which are of the same height are to one another as their bases. (Which is) the very thing it was required to show.
{\footnotesize \noindent$^\dag$ As is easily demonstrated, this proposition
holds even when the triangles, or parallelograms, do not share a common
side, and/or are not right-angled.\\[0.5ex]
$^\ddag$ This is a straight-forward generalization of Prop.~1.38.}

%%%%%%
% Prop 6.2
%%%%%%
\pdfbookmark[1]{Proposition 6.2}{pdf6.2}

\begin{center}
{\large Proposition 2}
\end{center}

If some straight-line is drawn parallel to one of the
sides of a triangle then it will cut the (other) sides of the triangle
proportionally. And if (two of) the sides of a triangle are cut proportionally then
the straight-line joining the cutting (points) will be parallel to the remaining side of the
triangle.

\epsfysize=2in
\centerline{\epsffile{Book06/fig02e.eps}}

For let $DE$ have been drawn parallel to one of the sides $BC$ of triangle $ABC$.
I say that as $BD$ is to $DA$, so $CE$ (is) to $EA$.

For let $BE$ and $CD$ have been joined.

Thus, triangle $BDE$ is equal to triangle $CDE$. For they are on the same
base $DE$ and between the same parallels $DE$ and $BC$  [Prop.~1.38].
And $ADE$ is some other triangle. And equal (magnitudes)
have the same ratio to the same (magnitude) [Prop.~5.7]. Thus, as triangle $BDE$ is to [triangle]
$ADE$, so triangle $CDE$ (is) to triangle $ADE$. But, as triangle $BDE$ (is) to
triangle $ADE$, so (is) $BD$ to $DA$. For, having the same height---(namely), the
(straight-line) drawn from $E$ perpendicular to $AB$---they are to one another as their bases
[Prop.~6.1].
So, for the same (reasons), as triangle $CDE$ (is) to $ADE$, so $CE$ (is) to $EA$.
And, thus, as $BD$ (is) to $DA$, so $CE$ (is) to $EA$ [Prop.~5.11].

And so, let the sides $AB$ and $AC$ of triangle $ABC$ have been cut proportionally (such that)
as $BD$ (is) to $DA$, so $CE$ (is) to $EA$. And let $DE$ have been joined. I say that
$DE$ is parallel to $BC$.

For, by the same construction, since as $BD$ is to $DA$, so $CE$ (is) to $EA$, but
as $BD$ (is) to $DA$, so triangle $BDE$ (is) to triangle $ADE$, and as $CE$
(is) to $EA$, so triangle $CDE$ (is) to triangle $ADE$ [Prop.~6.1], thus, also,  as triangle $BDE$ (is) to triangle
$ADE$, so triangle $CDE$ (is) to triangle $ADE$ [Prop.~5.11]. Thus, triangles $BDE$ and $CDE$ each have the same ratio to $ADE$. Thus, triangle $BDE$ is equal to triangle $CDE$ [Prop.~5.9].
And they are on the same base $DE$. And equal triangles, which are also on the same base, are also between the same parallels  [Prop.~1.39]. Thus,
$DE$ is parallel to $BC$.

Thus, if some straight-line is drawn parallel to one of the
sides of a triangle, then it will cut the (other) sides of the triangle
proportionally. And if (two of) the sides of a triangle are cut proportionally, then
the straight-line joining the cutting (points) will be parallel to the remaining side of the
triangle. (Which is) the very thing it was required to show.

%%%%%%
% Prop 6.3
%%%%%%
\pdfbookmark[1]{Proposition 6.3}{pdf6.3}

\begin{center}
{\large Proposition 3}
\end{center}

If an angle of a triangle is cut in half, and the
straight-line cutting the angle also cuts the base, then the segments of the base
will have the same ratio as the remaining sides of the triangle. And if
the segments of the base have the same ratio as the remaining sides of the
triangle, then the straight-line joining the vertex to the cutting (point) will
cut the angle of the triangle in half.

Let $ABC$ be a triangle. And let the angle $BAC$ have been cut in half
by the straight-line $AD$. I say that as $BD$ is to $CD$, so $BA$ (is) to $AC$.

For let $CE$ have been drawn through (point) $C$ parallel to $DA$.  And, $BA$ being drawn
through, let it meet ($CE$) at (point) $E$.$^\dag$

\epsfysize=2in
\centerline{\epsffile{Book06/fig03e.eps}}

And since the straight-line $AC$ falls across the parallel (straight-lines)
$AD$ and $EC$, angle $ACE$ is thus equal to $CAD$  [Prop.~1.29].
But, (angle) $CAD$ is assumed (to be) equal to $BAD$. Thus, (angle)
$BAD$ is also equal to $ACE$. Again, since the straight-line $BAE$ falls
across the parallel (straight-lines) $AD$ and $EC$, the external angle $BAD$
is equal to the internal (angle) $AEC$  [Prop.~1.29]. And (angle)
$ACE$ was also shown (to be) equal to $BAD$. Thus, angle $ACE$ is also
equal to $AEC$. And, hence, side $AE$ is equal to side $AC$  [Prop.~1.6].
And since $AD$ has been drawn parallel to one of the sides $EC$ of triangle
$BCE$, thus, proportionally, as $BD$ is to $DC$, so $BA$ (is) to $AE$ [Prop.~6.2]. And $AE$ (is) equal to $AC$.
Thus, as $BD$ (is) to $DC$, so $BA$ (is) to $AC$.

And so, let $BD$ be to $DC$,  as $BA$ (is) to
$AC$. And let $AD$ have been joined. I say that angle $BAC$ has been cut in
half by the straight-line $AD$.

For, by the same construction, since as $BD$ is to $DC$, so $BA$ (is) to
$AC$, then also as $BD$ (is) to $DC$, so $BA$ is to $AE$. For $AD$ has been drawn
parallel to one (of the sides) $EC$ of triangle $BCE$ [Prop.~6.2]. Thus, also, as $BA$ (is) to $AC$,
so $BA$ (is) to $AE$  [Prop.~5.11]. Thus,
$AC$ (is) equal to $AE$ [Prop.~5.9].
And, hence, angle $AEC$ is equal to $ACE$  [Prop.~1.5]. But, 
$AEC$ [is] equal to the external (angle) $BAD$, and $ACE$ is equal to the alternate
(angle) $CAD$  [Prop.~1.29]. Thus, (angle) $BAD$ is
also equal to $CAD$. Thus, angle $BAC$ has been cut in half by the
straight-line $AD$.

Thus, if an angle of a triangle is cut in half, and the
straight-line cutting the angle also cuts the base, then the segments of the base
will have the same ratio as the remaining sides of the triangle. And if
the segments of the base have the same ratio as the remaining sides of the
triangle, then the straight-line joining the vertex to the cutting (point) will
cut the angle of the triangle in half. (Which is) the very thing it was required
to show.
{\footnotesize \noindent$^\dag$ The fact that the
two straight-lines meet follows because the sum of $ACE$ and $CAE$
is less than two right-angles, as can easily be demonstrated. See Post.~5.}

%%%%%%
% Prop 6.4
%%%%%%
\pdfbookmark[1]{Proposition 6.4}{pdf6.4}

\begin{center}
{\large Proposition 4}
\end{center}

In equiangular triangles the sides about the
equal angles are proportional, and those (sides) subtending equal angles
 correspond.
 
\epsfysize=2.2in
\centerline{\epsffile{Book06/fig04e.eps}}

Let $ABC$ and $DCE$ be equiangular triangles, having angle $ABC$ equal to
$DCE$, and (angle) $BAC$ to $CDE$, and, further,  (angle) $ACB$ to $CED$.
I say that  in triangles $ABC$ and $DCE$ the sides about the equal angles
are proportional, and those (sides) subtending equal angles
 correspond.
 
Let $BC$ be placed straight-on to $CE$. And since angles
$ABC$ and $ACB$ are less than two right-angles  [Prop~1.17], and
$ACB$ (is) equal to $DEC$, thus $ABC$ and $DEC$ are less than two right-angles.
Thus, $BA$ and $ED$, being produced, will meet  [C.N.~5]. Let them have been produced,
and let them meet at (point) $F$.

And since angle $DCE$ is equal to $ABC$, $BF$ is parallel to $CD$  [Prop.~1.28]. Again, since (angle) $ACB$ is equal to $DEC$,  $AC$ is
 parallel to $FE$  [Prop.~1.28]. Thus, $FACD$ is a parallelogram.
Thus, $FA$ is equal to $DC$, and $AC$ to $FD$  [Prop.~1.34]. And since
$AC$ has been drawn parallel to one (of the sides) $FE$ of triangle
$FBE$, thus as $BA$ is to $AF$, so $BC$ (is) to $CE$ [Prop.~6.2]. And $AF$ (is) equal to $CD$.
Thus, as $BA$ (is) to $CD$, so $BC$ (is) to $CE$, and, alternately, as $AB$
(is) to $BC$, so $DC$ (is) to $CE$ [Prop.~5.16].
Again, since $CD$ is parallel to $BF$, thus as $BC$ (is) to $CE$, so
$FD$ (is) to $DE$ [Prop.~6.2]. And $FD$ (is) equal to
$AC$. Thus, as $BC$ is to $CE$, so $AC$ (is) to $DE$, and, alternately, as $BC$ (is)
to $CA$, so $CE$ (is) to $ED$ [Prop.~6.2].
Therefore, since
it was shown that as $AB$ (is) to $BC$, so $DC$ (is) to $CE$, and
as $BC$ (is) to $CA$, so $CE$ (is) to $ED$, thus, via equality, as
$BA$ (is) to $AC$, so $CD$ (is) to $DE$ [Prop.~5.22].

Thus, in  equiangular triangles the sides about the
equal angles are proportional, and those (sides) subtending equal angles
 correspond. (Which is) the very thing it was required to show.

%%%%%%
% Prop 6.5
%%%%%%
\pdfbookmark[1]{Proposition 6.5}{pdf6.5}

\begin{center}
{\large Proposition 5}
\end{center}

If two triangles have proportional sides then the
triangles will be equiangular, and will have the angles which  corresponding sides
subtend  equal.

\epsfysize=2.2in
\centerline{\epsffile{Book06/fig05e.eps}}

Let $ABC$ and $DEF$ be two triangles having proportional sides, (so that) as
$AB$ (is) to $BC$, so $DE$ (is) to $EF$, and as $BC$ (is) to $CA$, so $EF$ (is) to 
$FD$, and, further, as $BA$ (is) to $AC$, so $ED$ (is) to $DF$. I say that triangle
$ABC$ is equiangular to triangle $DEF$, and (that the triangles) will have the angles which corresponding sides subtend equal. (That is), (angle) $ABC$ (equal) to $DEF$, $BCA$ to
$EFD$, and, further, $BAC$ to $EDF$.

For let (angle) $FEG$, equal to angle $ABC$, and (angle) $EFG$, equal to
$ACB$, have been constructed on the straight-line $EF$ at the points
$E$ and $F$ on it (respectively)   [Prop.~1.23]. 
Thus, the remaining (angle) at $A$ is equal to the remaining (angle) at
$G$ [Prop.~1.32].

Thus, triangle $ABC$ is equiangular to [triangle] $EGF$. Thus, for triangles
$ABC$ and $EGF$, the sides about the equal angles are proportional, and
(those) sides subtending equal angles correspond [Prop.~6.4]. Thus, as $AB$ is to $BC$, [so]
$GE$ (is) to $EF$. But, as $AB$ (is) to $BC$, so, it was assumed, (is) $DE$  to
$EF$. Thus, as $DE$ (is) to $EF$, so $GE$ (is) to $EF$ [Prop.~5.11]. Thus, $DE$ and $GE$ each have the
same ratio to $EF$. Thus, $DE$ is equal to $GE$ [Prop.~5.9]. So, for the same (reasons), 
$DF$ is also equal to $GF$. Therefore, since $DE$ is equal to $EG$, and $EF$ (is)
common, the two (sides) $DE$, $EF$ are equal to the two (sides)
$GE$, $EF$ (respectively). And base $DF$ [is] equal to base $FG$. Thus, angle $DEF$
is equal to angle $GEF$  [Prop.~1.8],  and triangle $DEF$ (is) equal to triangle $GEF$,  and the remaining angles (are) equal to the remaining
angles which the equal sides subtend  [Prop.~1.4]. Thus,
angle $DFE$ is also equal to $GFE$, and (angle) $EDF$ to $EGF$. And since 
(angle) $FED$ is equal to $GEF$, and (angle) $GEF$ to $ABC$, angle $ABC$ is
thus also equal to $DEF$. So, for the same (reasons), (angle) $ACB$ is
also equal to $DFE$, and, further, the (angle) at $A$ to the (angle) at $D$.
Thus, triangle $ABC$ is equiangular to triangle $DEF$.

Thus,  if two triangles have proportional sides then the
triangles will be equiangular, and will have the angles which  corresponding sides
subtend  equal. (Which is) the very thing it was required to show.

%%%%%%
% Prop 6.6
%%%%%%
\pdfbookmark[1]{Proposition 6.6}{pdf6.6}

\begin{center}
{\large Proposition 6}
\end{center}

If two triangles have one angle equal to one angle,
and the sides about the equal angles proportional, then the triangles will
be equiangular, and will have the angles which corresponding sides subtend equal.

\epsfysize=2.2in
\centerline{\epsffile{Book06/fig06e.eps}}

Let $ABC$ and $DEF$ be two triangles having one angle, $BAC$, equal to
one angle, $EDF$ (respectively), and the sides about the equal angles proportional,
(so that) as $BA$ (is) to $AC$, so $ED$ (is) to $DF$. I say that triangle $ABC$ is
equiangular to triangle $DEF$, and will have angle $ABC$ equal to
$DEF$, and (angle) $ACB$ to $DFE$.

For let (angle) $FDG$, equal to each of $BAC$ and $EDF$, and (angle) $DFG$, equal to $ACB$, have been constructed on the
straight-line $AF$ at the points $D$ and $F$ on it (respectively)   [Prop.~1.23]. Thus, the remaining angle at $B$
is equal to the remaining angle at $G$  [Prop.~1.32].

Thus, triangle $ABC$ is equiangular to triangle $DGF$. Thus, proportionally,
as $BA$ (is) to $AC$, so $GD$ (is) to $DF$ [Prop.~6.4].  And it was also assumed that as $BA$
is) to $AC$, so $ED$ (is) to $DF$. And, thus, as $ED$ (is) to $DF$, so $GD$ (is) to $DF$ [Prop.~5.11]. Thus, $ED$ (is) equal to $DG$ [Prop.~5.9]. And $DF$ (is) common. So, the two
(sides) $ED$, $DF$ are equal to the two (sides) $GD$, $DF$ (respectively). And 
angle $EDF$ [is] equal to  angle $GDF$. Thus,  base $EF$ is equal to
base $GF$, and triangle $DEF$ is equal to triangle $GDF$, and the remaining
angles will be equal to the remaining angles which the equal sides subtend
 [Prop.~1.4]. Thus, (angle) $DFG$ is equal to $DFE$, and (angle) $DGF$ to
$DEF$. But, (angle) $DFG$ is equal to $ACB$. Thus, (angle) $ACB$ is also equal
to $DFE$. And  (angle) $BAC$ was also assumed (to be) equal to $EDF$. 
Thus, the remaining (angle) at $B$ is equal to the remaining (angle) at $E$  [Prop.~1.32]. Thus, triangle $ABC$ is equiangular to triangle $DEF$.

Thus, if two triangles have one angle equal to one angle,
and the sides about the equal angles proportional, then the triangles will
be equiangular, and will have the angles which corresponding sides subtend equal. (Which is) the very thing it was required to show.

%%%%%%
% Prop 6.7
%%%%%%
\pdfbookmark[1]{Proposition 6.7}{pdf6.7}

\begin{center}
{\large Proposition 7}
\end{center}

If two triangles have one angle equal to one angle,
and the sides about  other angles proportional, and the remaining angles
either both less than, or both not less than, right-angles, then the triangles will
be equiangular, and will have the angles about which the sides are proportional equal.

\epsfysize=2.2in
\centerline{\epsffile{Book06/fig07e.eps}}

Let $ABC$ and $DEF$ be two triangles having one angle, $BAC$,  equal to one angle, $EDF$ (respectively), and  the sides about (some) other angles, $ABC$ and
$DEF$ (respectively), proportional, (so that) as $AB$ (is) to $BC$, so
$DE$ (is) to $EF$, and the remaining (angles) at $C$ and $F$, first of all, both
less than right-angles. I say that triangle $ABC$ is equiangular to triangle
$DEF$, and (that) angle $ABC$ will be equal to $DEF$, and (that) the remaining  (angle) at $C$
(will be) manifestly equal to the remaining (angle) at $F$.

For if angle $ABC$ is not equal to (angle) $DEF$ then one of them is greater.
Let $ABC$ be greater. And let (angle) $ABG$, equal to (angle) $DEF$, have been constructed on the straight-line $AB$  at the point $B$  on it [Prop.~1.23].

And since angle $A$ is equal to (angle) $D$, and (angle) $ABG$ to $DEF$, the remaining
(angle) $AGB$ is thus equal to the remaining (angle) $DFE$  [Prop.~1.32].
Thus, triangle $ABG$ is equiangular to triangle $DEF$. Thus, as $AB$ is to
$BG$, so $DE$ (is) to $EF$ [Prop.~6.4]. 
And as $DE$ (is) to $EF$, [so] it was assumed (is) $AB$ to $BC$. Thus, $AB$ has the
same ratio to each of $BC$ and $BG$ [Prop.~5.11].
Thus, $BC$ (is) equal to $BG$ [Prop.~5.9].
And, hence, the angle at $C$ is equal to angle $BGC$  [Prop.~1.5].
And the angle at $C$ was assumed (to be) less than a right-angle.
Thus, (angle) $BGC$ is also less than a right-angle. Hence, the adjacent
angle to it, $AGB$, is greater than a right-angle [Prop.~1.13].
And ($AGB$) was shown to be equal to the (angle) at $F$. Thus, the (angle)
at $F$ is also greater than a right-angle. But it was assumed (to be)
less than a right-angle. The very thing is absurd. Thus, angle $ABC$ is
not unequal to (angle) $DEF$. Thus, (it is) equal. And the (angle) at $A$ is also
equal to the (angle) at $D$. And thus the remaining (angle) at $C$ is equal
to the remaining (angle) at $F$  [Prop.~1.32]. Thus, triangle $ABC$
is equiangular to triangle $DEF$.

But, again, let each of the (angles) at $C$ and $F$ be assumed  (to be) not
less than a right-angle. I say, again, that triangle $ABC$
is equiangular to triangle $DEF$ in this case also.

For, with the same construction, we can similarly show that $BC$ is equal to $BG$.
Hence, also, the angle at $C$ is equal to (angle) $BGC$. And  the (angle) at $C$ (is)
not less than a right-angle. Thus, $BGC$ (is) not less than a right-angle
either. So, in triangle $BGC$ the (sum of) two angles is not less than two
right-angles. The very thing is impossible [Prop.~1.17]. Thus, again,
angle $ABC$ is not unequal to $DEF$. Thus, (it is) equal.  And the (angle) at $A$
is also equal to the (angle) at $D$. Thus, the remaining (angle) at $C$ is equal
to the remaining (angle) at $F$ [Prop.~1.32]. Thus, triangle $ABC$ is equiangular to triangle $DEF$.

Thus, if two triangles have one angle equal to one angle,
and the sides about  other angles proportional, and the remaining angles
both less than, or both not less than, right-angles, then the triangles will
be equiangular, and will have the angles about which the sides (are) proportional equal. (Which is) the very thing it was required to show.

%%%%%%
% Prop 6.8
%%%%%%
\pdfbookmark[1]{Proposition 6.8}{pdf6.8}

\begin{center}
{\large Proposition 8}
\end{center}

If, in a right-angled triangle, a (straight-line) is drawn from the right-angle perpendicular
to the base then the triangles around the perpendicular are similar to
the whole (triangle), and to one another.

Let $ABC$ be a right-angled triangle having the angle $BAC$ a right-angle, 
and let $AD$ have been drawn from $A$, perpendicular to $BC$  
[Prop.~1.12].  I say that
triangles $ABD$ and $ADC$ are each similar to the whole (triangle) $ABC$ and, further, to one another.

\epsfysize=2in
\centerline{\epsffile{Book06/fig08e.eps}}

For since (angle) $BAC$ is equal to $ADB$---for each (are) right-angles---and the (angle) at $B$ (is) common to the two triangles $ABC$ and
$ABD$, the remaining (angle) $ACB$ is thus equal to the remaining (angle)
$BAD$  [Prop.~1.32]. Thus, triangle $ABC$ is equiangular to triangle $ABD$. 
Thus, as $BC$, subtending  the right-angle in triangle $ABC$, is to
$BA$, subtending the right-angle in triangle $ABD$, so the
same $AB$, subtending the angle at $C$ in triangle $ABC$, (is) to
$BD$, subtending the equal (angle) $BAD$ in triangle $ABD$,
and, further, (so is) $AC$ to $AD$, (both) subtending the angle at $B$ common to
the two triangles [Prop.~6.4].
Thus, triangle $ABC$ is equiangular to triangle $ABD$, and has the
sides about the equal angles proportional. Thus, triangle $ABC$
[is] similar to triangle $ABD$ [Def.~6.1].
So, similarly, we can show that triangle $ABC$ is also  similar to triangle
$ADC$. Thus, [triangles] $ABD$ and $ADC$ are each similar to the whole
 (triangle) $ABC$.
 
 So I say that triangles $ABD$ and $ADC$ are also similar to one another.
 
 For since the right-angle $BDA$ is equal to the right-angle $ADC$, and, indeed,
 (angle) $BAD$ was also shown (to be) equal to the (angle) at $C$, thus the
 remaining (angle) at $B$ is also equal to the remaining (angle) $DAC$ 
 [Prop.~1.32].
  Thus, triangle $ABD$ is equiangular to triangle $ADC$.
 Thus, as $BD$, subtending (angle) $BAD$ in triangle $ABD$, is to
 $DA$, subtending the (angle) at $C$ in triangle $ADC$, (which is) equal to (angle) $BAD$, so (is) the
 same $AD$, subtending the angle at $B$ in triangle $ABD$, to $DC$, subtending
 (angle) $DAC$ in triangle $ADC$, (which is) equal to the (angle) at $B$,
 and, further, (so is) $BA$ to $AC$, (each) subtending  right-angles [Prop.~6.4]. Thus, triangle $ABD$ is similar to
 triangle $ADC$  [Def.~6.1].
 
 Thus,  if, in a right-angled triangle, a (straight-line) is drawn from the right-angle perpendicular
to the base then the triangles around the perpendicular are similar to
the whole (triangle), and to one another. [(Which is) the very thing it
was required to show.]\\

\begin{center}
{\large Corollary}
\end{center}\vspace*{-7pt}

So (it is) clear, from this, that if, in a right-angled triangle, a
(straight-line) is drawn from the right-angle perpendicular to the base then the (straight-line so)
drawn is in mean proportion to the pieces of the base.$^\dag$ (Which
is) the very thing it was required to show.
{\footnotesize \noindent$^\dag$ In other words, the perpendicular is the geometric mean of the pieces.} 

%%%%%%
% Prop 6.9
%%%%%%
\pdfbookmark[1]{Proposition 6.9}{pdf6.9}

\begin{center}
{\large Proposition 9}
\end{center}

To cut off a prescribed part from a given straight-line.

\epsfysize=1.8in
\centerline{\epsffile{Book06/fig09e.eps}}

Let $AB$ be the given straight-line. So it is required to cut off a prescribed
part from $AB$.

So let a third (part) have been prescribed. [And] let some straight-line $AC$
have been drawn from (point) $A$, encompassing a random angle with
$AB$.  And let a random point $D$ have been taken on $AC$. And let
$DE$ and $EC$ be made equal to $AD$ [Prop.~1.3]. And let $BC$ have been
joined. And let $DF$ have been drawn through $D$ parallel to it [Prop.~1.31].

Therefore, since $FD$ has been drawn parallel to one of the sides, $BC$,
of triangle $ABC$, then, proportionally, as $CD$ is to $DA$, so $BF$ (is) to
$FA$ [Prop.~6.2].
And $CD$ (is) double $DA$.  Thus, $BF$ (is) also double $FA$. Thus, $BA$ (is)
triple $AF$.

Thus, the prescribed third part, $AF$, has been cut off from the given
straight-line, $AB$. (Which is) the very thing it was required to do.

%%%%%%
% Prop 6.10
%%%%%%
\pdfbookmark[1]{Proposition 6.10}{pdf6.10}

\begin{center}
{\large Proposition 10}
\end{center}

To cut a given uncut  straight-line similarly
to a given cut (straight-line).

\epsfysize=2.2in
\centerline{\epsffile{Book06/fig10e.eps}}

Let $AB$ be the given uncut straight-line, and $AC$ a  (straight-line) cut at
points $D$ and $E$, and let ($AC$) be laid down so as to encompass a random angle (with $AB$).  And let $CB$ have been joined. And let $DF$ and $EG$ have been
drawn through (points) $D$ and $E$ (respectively), parallel to $BC$, and let $DHK$
have been drawn through (point) $D$, parallel to $AB$  [Prop.~1.31].

Thus, $FH$ and $HB$ are each parallelograms. Thus, $DH$ (is) equal to $FG$,
and $HK$ to $GB$ [Prop.~1.34]. 
And since the straight-line $HE$ has been drawn parallel to one of the
sides, $KC$, of triangle $DKC$, thus, proportionally, as $CE$ is to $ED$,  so $KH$
(is) to $HD$ [Prop.~6.2]. And $KH$ (is)
equal to $BG$, and $HD$ to $GF$. Thus, as $CE$ is to $ED$, so $BG$ (is) to $GF$. 
Again, since $FD$ has been drawn parallel to one of the sides, $GE$, of
triangle $AGE$, thus, proportionally, as $ED$ is to $DA$, so $GF$ (is) to $FA$
[Prop.~6.2]. And it was also shown
that as $CE$ (is) to $ED$, so $BG$ (is) to $GF$. Thus, as
$CE$ is to $ED$, so $BG$ (is) to $GF$, and as $ED$ (is) to $DA$, so $GF$ (is) to
$FA$.

Thus, the given uncut straight-line, $AB$, has been cut similarly
to the given cut straight-line, $AC$. (Which is) the very thing it was required to do.

%%%%%%
% Prop 6.11
%%%%%%
\pdfbookmark[1]{Proposition 6.11}{pdf6.11}

\begin{center}
{\large Proposition 11}
\end{center}

To find a third (straight-line) proportional to two
given straight-lines.

Let $BA$ and $AC$ be the [two] given [straight-lines], and let them be laid
down encompassing a random angle. So it is required to find
a third (straight-line) proportional to $BA$ and $AC$. For let ($BA$ and $AC$) have
been produced to points $D$ and $E$ (respectively), and let $BD$ be made equal
to $AC$  [Prop.~1.3]. And let $BC$ have been joined. And let $DE$ have been drawn
through (point) $D$ parallel to it [Prop.~1.31].

Therefore, since $BC$
has been drawn parallel to one of the sides $DE$ of triangle $ADE$, proportionally, as $AB$ is to $BD$, so $AC$ (is) to $CE$ [Prop.~6.2]. And $BD$ (is) equal to $AC$.
Thus, as $AB$ is to $AC$, so $AC$ (is) to $CE$.

\epsfysize=2.5in
\centerline{\epsffile{Book06/fig11e.eps}}

Thus, a third (straight-line), $CE$, has been found (which is) proportional to the two
given straight-lines, $AB$ and $AC$. (Which is) the very thing it was required to
do.

%%%%%%
% Prop 6.12
%%%%%%
\pdfbookmark[1]{Proposition 6.12}{pdf6.12}

\begin{center}
{\large Proposition 12}
\end{center}

To find a fourth (straight-line) proportional
to three given straight-lines.

\epsfysize=2.2in
\centerline{\epsffile{Book06/fig12e.eps}}

Let $A$, $B$, and $C$ be the three given straight-lines. So it is required
to find a fourth (straight-line) proportional to $A$, $B$, and $C$.

Let the two straight-lines $DE$ and $DF$ be set out encompassing the
[random] angle $EDF$. And let $DG$ be made equal to $A$, and $GE$ to $B$,
and, further, $DH$ to $C$  [Prop.~1.3]. And $GH$ being joined, 
let $EF$ have been drawn through (point) $E$ parallel to it  [Prop.~1.31].

Therefore, since $GH$ has been drawn parallel to one of the sides $EF$ of 
triangle $DEF$, thus as $DG$ is to $GE$, so $DH$ (is) to $HF$ [Prop.~6.2]. And $DG$ (is) equal to $A$, and
$GE$ to $B$, and $DH$ to $C$. Thus, as $A$ is to $B$, so $C$ (is) to $HF$.

Thus, a fourth (straight-line), $HF$, has been found (which is) proportional to the three given
straight-lines, $A$, $B$, and $C$. (Which is) the very thing it was required to
do.

%%%%%%
% Prop 6.13
%%%%%%
\pdfbookmark[1]{Proposition 6.13}{pdf6.13}

\begin{center}
{\large Proposition 13}
\end{center}

To find the (straight-line) in mean proportion to two
given straight-lines.$^\dag$

\epsfysize=1.5in
\centerline{\epsffile{Book06/fig13e.eps}}

Let $AB$ and $BC$ be the two given straight-lines. So it is required to
find the (straight-line) in mean proportion to $AB$ and $BC$.

Let  ($AB$ and $BC$) be laid down straight-on (with respect to one another), and
let the semi-circle $ADC$ have been drawn on $AC$  [Prop.~1.10]. And let $BD$ have been drawn
from (point) $B$, at right-angles to $AC$  [Prop.~1.11]. And let $AD$ and $DC$ have been joined.

And since $ADC$ is an angle in a semi-circle, it is a right-angle  [Prop.~3.31].
 And since, in the right-angled triangle $ADC$,  the (straight-line) $DB$ has been drawn from the right-angle perpendicular
  to the base, $DB$ is thus the mean proportional to the pieces of the base,  $AB$ and $BC$ 
[Prop.~6.8~corr.].

Thus, $DB$ has been found (which is) in mean proportion to the two given
straight-lines, $AB$ and $BC$. (Which is) the very thing it was required to
do.
{\footnotesize\noindent $^\dag$ In other words, to find the geometric mean of
two given straight-lines.}

%%%%%%
% Prop 6.14
%%%%%%
\pdfbookmark[1]{Proposition 6.14}{pdf6.14}

\begin{center}
{\large Proposition 14}
\end{center}

In equal and equiangular parallelograms the
sides about the equal angles are reciprocally proportional. And those
equiangular parallelograms in which  the sides about the equal
angles are reciprocally proportional are equal.

Let $AB$ and $BC$ be equal and equiangular parallelograms having the angles
at $B$ equal. And let $DB$ and $BE$ be laid down straight-on (with respect to one
another). Thus, $FB$ and $BG$ are also straight-on (with respect to one another)  [Prop.~1.14].
I say that the sides of $AB$ and $BC$ about the equal angles
are reciprocally proportional, that is to say, that as $DB$ is to $BE$,
so $GB$ (is) to $BF$.

\epsfysize=2.2in
\centerline{\epsffile{Book06/fig14e.eps}}

For let the parallelogram $FE$ have been completed. Therefore, since parallelogram $AB$ is equal to parallelogram $BC$, and $FE$ (is) some
other (parallelogram), thus as (parallelogram) $AB$ is to $FE$, so (parallelogram) $BC$ (is) to $FE$ [Prop.~5.7]. But, as (parallelogram) $AB$ (is) to $FE$, so $DB$
(is) to $BE$, and as (parallelogram) $BC$ (is) to $FE$, so $GB$ (is) to $BF$ [Prop.~6.1]. Thus, also, as $DB$ (is) to
$BE$, so $GB$ (is) to $BF$. Thus, in parallelograms $AB$ and $BC$
the sides about the equal angles are reciprocally proportional.

And so, let $DB$ be to $BE$, as $GB$ (is) to $BF$. I say that parallelogram
$AB$ is equal to parallelogram $BC$.

For since as $DB$ is to $BE$, so $GB$ (is) to $BF$, but as $DB$ (is) to $BE$, so
parallelogram $AB$ (is) to parallelogram $FE$, and as $GB$ (is) to $BF$,
so parallelogram $BC$ (is) to parallelogram $FE$  [Prop.~6.1], thus, also, as (parallelogram) $AB$ (is)
to $FE$, so (parallelogram) $BC$ (is) to $FE$ [Prop.~5.11].
Thus,  parallelogram $AB$ is equal to parallelogram $BC$ [Prop.~5.9].

Thus, in equal and equiangular parallelograms the
sides about the equal angles are reciprocally proportional. And those
equiangular parallelograms in which the sides about the equal
angles are reciprocally proportional are equal. (Which is) the very thing
it was required to show.

%%%%%%
% Prop 6.15
%%%%%%
\pdfbookmark[1]{Proposition 6.15}{pdf6.15}

\begin{center}
{\large Proposition 15}
\end{center}

In equal triangles also having one angle equal to
one (angle) the sides about the equal angles are reciprocally proportional.
And those triangles having one angle equal to one angle for which 
the sides about the equal angles (are) reciprocally proportional are equal.

Let $ABC$ and $ADE$ be equal triangles having one angle equal to
one (angle), (namely) $BAC$ (equal) to $DAE$. I say that, in triangles
$ABC$ and $ADE$, the sides about the equal angles are reciprocally
proportional, that is to say, that as $CA$ is to $AD$, so $EA$ (is) to $AB$.

\epsfysize=2.2in
\centerline{\epsffile{Book06/fig15e.eps}}

For let  $CA$ be laid down so as to be  straight-on (with respect) to $AD$. Thus,
$EA$ is also straight-on (with respect) to $AB$ 
[Prop.~1.14]. And let $BD$ have been joined.

Therefore, since triangle $ABC$ is equal to triangle $ADE$, and
$BAD$ (is) some other (triangle), thus as triangle $CAB$ is
to triangle $BAD$, so triangle $EAD$ (is) to triangle $BAD$  [Prop.~5.7]. But, as (triangle)
$CAB$ (is) to $BAD$, so $CA$ (is) to $AD$, and as (triangle) $EAD$ (is) to $BAD$, so
$EA$ (is) to $AB$
[Prop.~6.1]. And thus, as $CA$ (is) to $AD$,
so $EA$ (is) to $AB$.  
Thus,  in triangles $ABC$ and
$ADE$ the sides about the equal angles (are) reciprocally proportional.

And so, let the sides of triangles $ABC$ and $ADE$ be reciprocally
proportional, and (thus) let $CA$ be to $AD$, as $EA$ (is) to $AB$. I say
that triangle $ABC$ is equal to triangle $ADE$.

For, $BD$ again being  joined, since as $CA$ is to $AD$, so $EA$ (is) to $AB$,
but as $CA$ (is) to $AD$, so triangle $ABC$ (is) to triangle
$BAD$, and as $EA$ (is) to $AB$, so triangle $EAD$ (is) to triangle $BAD$ 
[Prop.~6.1], thus as triangle
$ABC$ (is) to triangle $BAD$, so triangle $EAD$ (is) to triangle $BAD$. 
Thus, (triangles) $ABC$ and $EAD$ each have the same ratio to $BAD$. 
Thus, [triangle] $ABC$ is equal to triangle $EAD$ [Prop.~5.9].

Thus, in equal triangles also having one angle equal to
one (angle) the sides about the equal angles (are) reciprocally proportional.
And those triangles having one angle equal to one angle for which 
the sides about the equal angles (are) reciprocally proportional are equal.
(Which is) the very thing it was required to show.

%%%%%%
% Prop 6.16
%%%%%%
\pdfbookmark[1]{Proposition 6.16}{pdf6.16}

\begin{center}
{\large Proposition 16}
\end{center}

If four straight-lines are proportional then the
rectangle contained by the (two) outermost is equal to the rectangle contained
by the middle (two). And if the rectangle contained by the (two) outermost
is
equal to the rectangle contained by the middle (two) then the four straight-lines
will be proportional.

\epsfysize=1.25in
\centerline{\epsffile{Book06/fig16e.eps}}

Let $AB$, $CD$, $E$, and $F$ be four proportional straight-lines, (such
that) as $AB$ (is) to $CD$, so $E$ (is) to $F$. I say that the rectangle contained
by  $AB$ and $F$ is equal to the rectangle contained by $CD$ and $E$.

\mbox{[}For] let $AG$ and $CH$ have been drawn from points $A$ and $C$
at right-angles to the straight-lines $AB$ and $CD$ (respectively)  [Prop.~1.11]. 
And let $AG$ be made equal to $F$, and $CH$ to $E$  [Prop.~1.3]. And
let the parallelograms $BG$ and $DH$ have been completed.

And since as $AB$ is to $CD$, so $E$ (is) to $F$, and $E$ (is) equal $CH$,
and $F$ to $AG$, thus as $AB$ is to $CD$, so $CH$ (is) to $AG$. Thus, in the parallelograms $BG$ and $DH$ the sides about the equal angles are
reciprocally proportional. And those equiangular parallelograms
in which the sides about the equal angles are reciprocally proportional
are equal [Prop.~6.14]. Thus, parallelogram
$BG$ is equal to parallelogram $DH$. And $BG$ is the (rectangle contained)
by $AB$ and $F$. For $AG$ (is) equal to $F$. And $DH$ (is) the (rectangle
contained) by $CD$ and $E$. For $E$ (is) equal to $CH$. Thus, the rectangle
contained by $AB$ and $F$ is equal to the rectangle contained by $CD$ and $E$.

And so, let the rectangle contained by $AB$ and $F$  be equal to the rectangle
contained by $CD$ and $E$. I say that the four straight-lines will be proportional,
(so that) as $AB$ (is) to $CD$, so $E$ (is) to $F$.

For,  with the same construction, since the (rectangle contained) by $AB$ and $F$
is equal to the (rectangle contained) by $CD$ and $E$. And $BG$ is the (rectangle
contained) by $AB$ and $F$. For $AG$ is equal to $F$. And $DH$ (is) the
(rectangle contained) by $CD$ and $E$. For $CH$ (is) equal to $E$. $BG$ is thus
equal to $DH$. And they are equiangular.  And in equal and equiangular parallelograms  the sides about the equal angles are reciprocally
proportional [Prop.~6.14]. Thus, as $AB$
is to $CD$, so $CH$ (is) to $AG$. And $CH$ (is) equal to $E$, and $AG$ to $F$.
Thus, as $AB$  is to $CD$, so $E$ (is) to $F$.

Thus, if four straight-lines are proportional then the
rectangle contained by the (two) outermost is equal to the rectangle contained
by the middle (two). And if the rectangle contained by the (two) outermost
is
equal to the rectangle contained by the middle (two) then the four straight-lines
will be proportional. (Which is) the very thing it was required to show.

%%%%%%
% Prop 6.17
%%%%%%
\pdfbookmark[1]{Proposition 6.17}{pdf6.17}

\begin{center}
{\large Proposition 17}
\end{center}

If three straight-lines are proportional then
the rectangle contained by the (two) outermost is equal to the square
on the middle (one). And if the rectangle contained by the (two) outermost
is equal to the square on the middle (one) then the three straight-lines
will be proportional.

\epsfysize=0.9in
\centerline{\epsffile{Book06/fig17e.eps}}

Let $A$, $B$ and $C$ be three proportional straight-lines, (such that) as $A$ (is) to
$B$, so $B$ (is) to $C$. I say that the rectangle contained by $A$ and $C$ is
equal to the square on $B$.

Let $D$ be made equal to $B$  [Prop.~1.3].

And since as $A$ is to $B$, so $B$ (is) to $C$, and $B$ (is) equal to $D$, thus
as $A$ is to $B$, (so) $D$ (is) to $C$. And if four straight-lines are proportional
then the [rectangle] contained by the (two) outermost is equal
to the rectangle contained by the middle (two) [Prop.~6.16]. Thus, the (rectangle contained)
by $A$ and $C$ is equal to the (rectangle contained) by $B$ and $D$. 
But, the (rectangle contained) by $B$ and $D$ is the (square) on $B$. For
$B$ (is) equal to $D$. Thus, the rectangle contained by $A$ and $C$ is equal
to the square on $B$.

And so, let the (rectangle contained) by $A$ and $C$ be equal to
the (square) on $B$. I say that as $A$ is to $B$, so $B$ (is) to $C$.

For, with the same construction, since the (rectangle contained)
by $A$ and $C$ is equal to the (square) on $B$. But, the (square) on $B$
is the (rectangle contained) by $B$ and $D$. For $B$ (is) equal to $D$.
The (rectangle contained) by $A$ and $C$ is thus equal to the (rectangle
contained) by $B$ and $D$. And if the (rectangle contained) by the (two)
outermost is equal to the (rectangle contained) by the middle (two) then
the four straight-lines are proportional [Prop.~6.16]. 
Thus, as $A$ is to $B$, so $D$ (is) to $C$.
And $B$ (is) equal to $D$. Thus, as $A$ (is) to $B$, so $B$ (is) to $C$.

Thus,  if three straight-lines are proportional then
the rectangle contained by the (two) outermost is equal to the square
on the middle (one). And if the rectangle contained by the (two) outermost
is equal to the square on the middle (one) then the three straight-lines
will be proportional. (Which is) the very thing it was required to show.

%%%%%%
% Prop 6.18
%%%%%%
\pdfbookmark[1]{Proposition 6.18}{pdf6.18}

\begin{center}
{\large Proposition 18}
\end{center}

To describe a rectilinear figure similar, and similarly laid down, to a
given rectilinear figure on a given straight-line.

\epsfysize=1.6in
\centerline{\epsffile{Book06/fig18e.eps}}

Let $AB$ be the given straight-line, and $CE$ the given rectilinear figure.
So it is required to describe a rectilinear figure similar, and similarly laid down,   to the rectilinear
figure $CE$ on the straight-line $AB$.

Let $DF$ have been joined, and let $GAB$, equal to the angle at $C$, and $ABG$,
equal to (angle) $CDF$, have been constructed  on the straight-line $AB$  
at the points $A$ and $B$ on it (respectively) [Prop.~1.23]. Thus, the
remaining (angle) $CFD$ is equal to $AGB$  [Prop.~1.32].
Thus, triangle $FCD$ is equiangular to triangle $GAB$. Thus, proportionally,
as $FD$ is to $GB$, so $FC$ (is) to $GA$, and $CD$ to $AB$ [Prop.~6.4]. Again, let $BGH$, equal to angle $DFE$,
and $GBH$ equal to (angle) $FDE$, have been constructed on the straight-line $BG$  at the points $G$ and
$B$ on it (respectively) [Prop.~1.23]. Thus, the
remaining (angle) at $E$ is equal to the remaining (angle) at $H$  [Prop.~1.32]. Thus, triangle $FDE$ is equiangular to triangle $GHB$.
Thus, proportionally, as $FD$ is to $GB$, so $FE$ (is) to $GH$, and $ED$ to 
$HB$ [Prop.~6.4]. And it was also
shown (that) as $FD$ (is) to $GB$, so $FC$ (is) to $GA$, and $CD$ to $AB$. Thus, also, as $FC$  (is) to $AG$, so  $CD$ (is) to $AB$, and $FE$ to $GH$, and, further, $ED$ to
$HB$. And since angle $CFD$ is equal to $AGB$, and $DFE$ to $BGH$, thus the
whole (angle) $CFE$ is equal to the whole (angle) $AGH$. So, for the same (reasons), (angle) $CDE$ is also equal to $ABH$. And the (angle) at $C$ is also equal to
the (angle) at $A$, and the (angle) at $E$ to the (angle) at $H$. Thus,  (figure) $AH$
is equiangular to $CE$. And (the two figures) have the sides about their equal angles
proportional. Thus, the rectilinear figure $AH$ is similar to the rectilinear figure
$CE$ [Def.~6.1].

Thus, the rectilinear figure $AH$, similar,  and similarly laid down, to the given rectilinear figure
$CE$ has been constructed on the given straight-line
$AB$. (Which is) the very thing it was required to do.

%%%%%%
% Prop 6.19
%%%%%%
\pdfbookmark[1]{Proposition 6.19}{pdf6.19}

\begin{center}
{\large Proposition 19}
\end{center}

Similar triangles are to one another in the squared$^\dag$
ratio of (their) corresponding sides.

\epsfysize=1.4in
\centerline{\epsffile{Book06/fig19e.eps}}

Let $ABC$ and $DEF$ be similar triangles having the angle at $B$ equal to
the (angle)  at $E$, and $AB$  to $BC$, as $DE$ (is) to $EF$, such that
$BC$ corresponds to $EF$. I say that triangle $ABC$ has a
squared ratio to triangle $DEF$ with respect to (that side) $BC$ (has) to $EF$.

For let a third (straight-line), $BG$, have been taken (which is)
proportional to $BC$ and $EF$,  so that as $BC$ (is) to $EF$, so $EF$ (is)
to $BG$ [Prop.~6.11]. And let $AG$
have been joined.

Therefore, since as $AB$ is to $BC$, so $DE$ (is) to $EF$, thus, alternately, 
as $AB$ is to $DE$, so $BC$ (is) to $EF$ [Prop.~5.16].
But, as $BC$ (is) to $EF$, so $EF$ is to $BG$. And, thus, as $AB$ (is) to $DE$,
so $EF$ (is) to $BG$. Thus, for triangles $ABG$ and $DEF$, the sides
about the equal angles are reciprocally proportional. And those triangles
 having one (angle) equal to one (angle) for which the sides
about the equal angles are reciprocally proportional are equal [Prop.~6.15]. Thus, triangle
$ABG$ is equal to triangle $DEF$. And since as $BC$ (is) to $EF$, so
$EF$ (is) to $BG$, and if three straight-lines are proportional then the
first has a squared ratio to the third with respect to the second  [Def.~5.9],  $BC$ thus has a squared ratio to $BG$ with
respect to  (that) $CB$ (has) to $EF$.  And as $CB$ (is) to
$BG$, so triangle $ABC$ (is) to triangle $ABG$ [Prop.~6.1].  Thus, triangle $ABC$ also has a squared
ratio to (triangle) $ABG$ with respect to (that side) $BC$ (has) to $EF$. And triangle
$ABG$ (is) equal to triangle $DEF$. Thus, triangle $ABC$ also
has a squared ratio to triangle $DEF$ with respect to (that side) $BC$ (has) to $EF$.

Thus, similar triangles are to one another in the squared
ratio of  (their) corresponding sides. [(Which is) the very thing it was required
to show].\\

\begin{center}
{\large Corollary}
\end{center}\vspace*{-7pt}

So it is clear, from this, that if three straight-lines are proportional, then
as the first is to the third, so  the figure (described) on the first (is) to the similar,
and similarly described, (figure) on the second. (Which is) the very thing it
was required to show.
{\footnotesize\noindent$^\dag$ Literally, ``double''.}

%%%%%%
% Prop 6.20
%%%%%%
\pdfbookmark[1]{Proposition 6.20}{pdf6.20}

\begin{center}
{\large Proposition 20}
\end{center}

Similar polygons can be divided into equal
numbers of similar
triangles corresponding (in proportion) to the wholes, and one
polygon has to the (other) polygon a squared ratio with respect to
(that) a corresponding side (has) to a corresponding side.

\epsfysize=1.8in
\centerline{\epsffile{Book06/fig20e.eps}}

Let $ABCDE$ and $FGHKL$ be similar polygons, and let $AB$ correspond to
$FG$. I say that polygons $ABCDE$ and $FGHKL$ can be divided into equal numbers of similar triangles corresponding (in proportion) to the wholes,
and (that) polygon $ABCDE$ has a squared ratio to polygon
$FGHKL$ with respect to that $AB$ (has) to $FG$.

Let $BE$, $EC$, $GL$, and $LH$ have been joined.

And since polygon $ABCDE$ is similar to polygon $FGHKL$, angle
$BAE$ is equal to angle $GFL$,
and as $BA$ is to $AE$, so $GF$ (is) to $FL$ [Def.~6.1].
Therefore, since $ABE$ and $FGL$ are two triangles having one angle equal
to one angle  and the sides about the equal angles proportional, triangle
$ABE$ is thus equiangular to triangle $FGL$ [Prop.~6.6]. Hence, (they are) also similar
 [Prop.~6.4, Def.~6.1]. Thus, angle $ABE$ is equal to
(angle) $FGL$.  And the whole (angle) $ABC$ is equal to the whole (angle)
$FGH$, on  account of the similarity of the polygons. Thus, the remaining
angle $EBC$ is equal to $LGH$. And since, on account of the similarity of
triangles $ABE$ and $FGL$, as $EB$ is to $BA$, so $LG$ (is) to $GF$, but 
also, on account of the similarity of the polygons, as $AB$ is to $BC$, so
$FG$ (is) to $GH$, thus, via equality, as $EB$ is to $BC$, so $LG$ (is) to $GH$ [Prop.~5.22], and the sides about the
equal angles, $EBC$ and $LGH$, are  proportional. Thus, triangle
$EBC$ is equiangular to triangle $LGH$ [Prop.~6.6]. Hence, triangle $EBC$ is also
similar to triangle $LGH$  [Prop.~6.4, Def.~6.1]. So, for the same (reasons), triangle
$ECD$ is also similar to triangle $LHK$. Thus, the similar
polygons $ABCDE$ and $FGHKL$ have been divided into
equal numbers of similar triangles.

I also say that (the triangles) correspond (in proportion) to the wholes.
That is to say,  the triangles are proportional:  $ABE$, $EBC$, and
$ECD$ are the leading (magnitudes), and their (associated) following (magnitudes are) 
$FGL$, $LGH$, and $LHK$ (respectively). (I) also (say) that polygon
$ABCDE$ has a squared ratio to polygon $FGHKL$ with respect to
(that) a corresponding side (has) to a corresponding side---that is to
say, (side) $AB$ to $FG$.

For let $AC$ and $FH$ have been joined. And since angle $ABC$ is
equal to $FGH$,  and
as $AB$ is to $BC$, so $FG$ (is) to $GH$, on account of the similarity of the polygons,  triangle $ABC$ is equiangular to triangle
$FGH$ [Prop.~6.6]. Thus, angle $BAC$ is equal 
to $GFH$, and (angle) $BCA$ to $GHF$. And since angle $BAM$ is equal to 
$GFN$, and (angle) $ABM$ is also equal to $FGN$ (see earlier), the remaining (angle) $AMB$ is
thus also equal to the remaining (angle) $FNG$  [Prop.~1.32]. 
Thus, triangle $ABM$ is equiangular to triangle $FGN$. So, similarly, we can
show that triangle $BMC$ is also equiangular to triangle $GNH$. Thus, proportionally,
as $AM$ is to $MB$, so $FN$ (is) to $NG$, and as $BM$ (is) to $MC$, so $GN$ (is) to
$NH$ [Prop.~6.4]. Hence, also, via equality,
as $AM$ (is) to $MC$, so $FN$ (is) to $NH$ 
[Prop.~5.22]. But, as $AM$ (is) to $MC$,
so [triangle] $ABM$ is to $MBC$, and $AME$ to $EMC$. For they are to one another
 as their bases  [Prop.~6.1]. And as one
 of the leading (magnitudes) is to one of the following (magnitudes), so
    (the sum of) all the leading (magnitudes) is to (the sum of) all the following (magnitudes)
[Prop.~5.12]. Thus, as triangle $AMB$
 (is) to $BMC$, so (triangle) $ABE$ (is) to $CBE$. But, as (triangle) $AMB$ (is) to
 $BMC$, so $AM$ (is) to $MC$. Thus, also, as $AM$ (is) to $MC$, so 
triangle $ABE$
 (is) to triangle $EBC$. And so, for the same (reasons), as $FN$ (is) to $NH$,
 so triangle $FGL$  (is) to triangle $GLH$. And as $AM$ is to $MC$, so $FN$ (is) to
 $NH$. Thus, also, as triangle $ABE$ (is) to triangle $BEC$, so triangle $FGL$
 (is) to triangle $GLH$, and, alternately, as triangle $ABE$ (is) to triangle
 $FGL$, so triangle $BEC$ (is) to triangle $GLH$  [Prop.~5.16]. So, similarly, we can also show, by
 joining $BD$ and $GK$, that  as triangle $BEC$ (is) to triangle $LGH$,
 so triangle $ECD$ (is) to triangle $LHK$. And since as triangle $ABE$ is
 to triangle $FGL$, so (triangle) $EBC$ (is) to $LGH$, and, further, (triangle)
 $ECD$ to $LHK$, and also as one of the leading (magnitudes is) to one
 of the following, so (the sum of) all the leading (magnitudes is) to (the sum of) all the
 following [Prop.~5.12], thus as triangle
 $ABE$ is to triangle $FGL$, so polygon $ABCDE$ (is) to polygon
 $FGHKL$. But, triangle $ABE$ has a squared ratio to triangle $FGL$ with
 respect to (that) the corresponding side $AB$ (has) to the corresponding side
 $FG$. For, similar triangles are in the squared ratio of corresponding sides 
[Prop.~6.14]. Thus, polygon
$ABCDE$ also has a squared ratio to polygon $FGHKL$ with respect to
(that) the corresponding side $AB$ (has) to the corresponding side
$FG$.

Thus, similar polygons can be divided into equal
numbers of similar
triangles corresponding (in proportion) to the wholes, and one
polygon has to the (other) polygon a squared ratio with respect to
(that) a corresponding side (has) to a corresponding side. [(Which is) the
very thing it was required to show].\\

\begin{center}
{\large Corollary}
\end{center}\vspace*{-7pt}

And, in the same manner, it can also be shown for [similar] quadrilaterals
that they are in the squared ratio of (their) corresponding sides. And it was also
shown for triangles. Hence, in general, similar rectilinear figures are also to one
another in the squared ratio of (their) corresponding sides. (Which is) the
very thing it was required to show.

%%%%%%
% Prop 6.21
%%%%%%
\pdfbookmark[1]{Proposition 6.21}{pdf6.21}

\begin{center}
{\large Proposition 21}
\end{center}

(Rectilinear figures) similar to the same rectilinear
figure are also similar to one another.

\epsfysize=2.5in
\centerline{\epsffile{Book06/fig21e.eps}}

Let each of the rectilinear figures $A$ and $B$ be similar to (the rectilinear figure)
$C$.  I say that $A$ is also similar to $B$.

For since $A$ is similar to $C$, ($A$) is equiangular to ($C$), and has the sides about
the equal angles proportional [Def.~6.1].
Again, since $B$ is similar to $C$, ($B$) is equiangular to ($C$), and has the sides about the equal angles proportional [Def.~6.1].
Thus, $A$ and $B$ are each equiangular to $C$, and have the sides about the
equal angles proportional [hence, $A$ is also equiangular to $B$, and has the sides
about the equal angles proportional]. Thus, $A$ is similar to $B$ [Def.~6.1]. (Which is) the very thing it was required
to show.

%%%%%%
% Prop 6.22
%%%%%%
\pdfbookmark[1]{Proposition 6.22}{pdf6.22}

\begin{center}
{\large Proposition 22}
\end{center}

If four straight-lines are proportional then
similar, and similarly described, rectilinear figures (drawn) on them will also
be proportional. And if similar, and similarly described, rectilinear figures (drawn) on them are proportional then the  straight-lines themselves will also be proportional.

\epsfysize=3in
\centerline{\epsffile{Book06/fig22e.eps}}

Let $AB$, $CD$, $EF$, and $GH$ be four proportional straight-lines, (such
that) as $AB$ (is) to $CD$, so $EF$ (is) to $GH$. And  let the similar, and similarly
laid out, rectilinear figures $KAB$ and $LCD$ have been described on
$AB$ and $CD$ (respectively), and the similar, and similarly
laid out, rectilinear figures $MF$ and $NH$ on $EF$ and $GH$ (respectively). 
I say that as $KAB$ is to $LCD$, so $MF$ (is) to $NH$.

For let a third (straight-line) $O$ have been taken (which is) proportional to
$AB$ and $CD$, and a third (straight-line) $P$ proportional to $EF$ and $GH$
[Prop.~6.11]. And since as $AB$ is to
$CD$, so $EF$ (is) to $GH$, and as $CD$ (is) to $O$, so $GH$ (is) to $P$, thus, via equality,
as $AB$ is to $O$, so $EF$ (is) to $P$ [Prop.~5.22].
But, as $AB$ (is) to $O$, so [also] $KAB$ (is)  to $LCD$, and as $EF$ (is) to $P$,
so $MF$ (is) to $NH$ [Prop.~5.19~corr.].
And, thus, as $KAB$ (is) to $LCD$, so $MF$ (is) to $NH$.

And so let $KAB$ be to $LCD$, as $MF$ (is) to $NH$. I say also that as
$AB$ is to $CD$, so $EF$ (is) to $GH$. For if as $AB$ is to $CD$, so $EF$ (is) not to $GH$, 
let $AB$ be to $CD$, as $EF$ (is) to $QR$ [Prop.~6.12]. And let the rectilinear figure $SR$, similar, and similarly laid down, to either  of $MF$ or $NH$, have been described on
$QR$  [Props.~6.18, 6.21].

Therefore, since as $AB$ is to $CD$, so $EF$ (is) to $QR$, and the similar, and
similarly laid out, (rectilinear figures) $KAB$ and $LCD$ have been described
on $AB$  and $CD$ (respectively), and the similar, and similarly laid out, (rectilinear figures) $MF$ and $SR$ on $EF$ and $QR$ (resespectively), thus
as $KAB$ is to $LCD$, so $MF$ (is) to $SR$  (see above). And it was also assumed
that as $KAB$ (is) to $LCD$, so $MF$ (is) to $NH$. Thus, also, as $MF$ (is) to $SR$,
so $MF$ (is) to $NH$ [Prop.~5.11]. Thus, $MF$ has the same ratio to each of $NH$ and $SR$. 
Thus, $NH$ is equal to $SR$ [Prop.~5.9].
And it is also similar, and similarly laid out, to it. Thus,  $GH$ (is) equal to $QR$.$^\dag$
And since $AB$ is to $CD$, as $EF$ (is) to $QR$, and $QR$ (is) equal to $GH$, thus
as $AB$ is to $CD$, so $EF$ (is) to $GH$.

Thus, if four straight-lines are proportional, then
similar, and similarly described, rectilinear figures (drawn) on them will also
be proportional. And if similar, and similarly described, rectilinear figures (drawn) on them are proportional then the  straight-lines themselves will also be proportional. (Which is) the very thing it was required to show.
{\footnotesize \noindent$^\dag$ Here, Euclid assumes, without proof, that
if two similar figures are equal then any pair of corresponding sides is also
equal.}

%%%%%%
% Prop 6.23
%%%%%%
\pdfbookmark[1]{Proposition 6.23}{pdf6.23}

\begin{center}
{\large Proposition 23}
\end{center}

Equiangular parallelograms have to one another the ratio compounded$^\dag$ out of  (the ratios of) their sides.

Let $AC$ and $CF$ be equiangular parallelograms having angle $BCD$ equal to
$ECG$. I say that parallelogram $AC$ has to parallelogram $CF$ the ratio compounded out of (the ratios of) their sides.

For let $BC$ be laid down so as to be straight-on to $CG$. Thus, $DC$ is also straight-on
to $CE$ [Prop.~1.14]. And let the parallelogram $DG$ have been completed. And let
some straight-line $K$ have been laid down. And let it be contrived that as $BC$ (is) to $CG$, so $K$ (is) to $L$, and as $DC$ (is) to $CE$,
so $L$ (is) to $M$ [Prop.~6.12].

Thus, the ratios of $K$ to $L$ and of $L$ to $M$ are the same as the ratios
of the sides, (namely), $BC$ to $CG$ and $DC$ to $CE$ (respectively). 
But, the ratio of $K$ to $M$ is compounded out of the ratio of $K$ to $L$
and (the ratio) of $L$ to $M$. Hence, $K$ also has to $M$ the ratio compounded out of
(the ratios of)  the sides
(of the parallelograms).
And since as $BC$ is to $CG$, so parallelogram $AC$ (is) to $CH$ [Prop.~6.1], but as $BC$ (is) to $CG$, so 
$K$ (is) to $L$, thus, also, as $K$ (is) to $L$, so (parallelogram) $AC$ (is) to $CH$.
Again, since as $DC$ (is) to $CE$, so parallelogram $CH$ (is) to $CF$
[Prop.~6.1], but 
as $DC$ (is) to $CE$, so $L$ (is) to $M$, thus, also, as $L$ (is) to $M$, so parallelogram
$CH$ (is) to parallelogram $CF$. Therefore, since it was shown that as $K$
(is) to $L$, so parallelogram $AC$ (is) to parallelogram $CH$, and as $L$ (is)
to $M$, so parallelogram $CH$ (is) to parallelogram $CF$, thus, via equality, 
as $K$ is to $M$, so (parallelogram) $AC$ (is) to parallelogram $CF$ [Prop.~5.22]. And $K$ has to $M$ the ratio compounded
out of (the ratios of) the  sides (of the parallelograms). Thus, (parallelogram) $AC$
also has to (parallelogram) $CF$ the ratio compounded out of (the ratio of) their sides.\\

\epsfysize=2in
\centerline{\epsffile{Book06/fig23e.eps}}

Thus, equiangular parallelograms have to one another the ratio compounded out of (the ratio  of)  their sides. (Which is) the very thing it was required to show.
{\footnotesize\noindent$^\dag$ In modern terminology, if two
ratios are ``compounded'' then they are multiplied together.}

%%%%%%
% Prop 6.24
%%%%%%
\pdfbookmark[1]{Proposition 6.24}{pdf6.24}

\begin{center}
{\large Proposition 24}
\end{center}

In any parallelogram the parallelograms
about the diagonal are similar to the whole, and to one another.

Let $ABCD$ be a parallelogram, and $AC$ its diagonal. And let
$EG$ and $HK$ be parallelograms about $AC$. I say that the
parallelograms $EG$ and $HK$ are each similar to the whole (parallelogram) $ABCD$, and
to one another.

For since $EF$ has been drawn parallel to one of the sides $BC$ of triangle
$ABC$, proportionally, as $BE$ is to $EA$, so $CF$ (is) to $FA$ [Prop.~6.2]. Again, since $FG$ has been drawn
parallel to one (of the sides) $CD$ of triangle $ACD$, proportionally, 
as $CF$ is to $FA$, so $DG$ (is) to $GA$ [Prop.~6.2].
But, as $CF$ (is) to $FA$, so it was also shown (is) $BE$ to $EA$. And thus as
$BE$ (is) to $EA$, so $DG$ (is) to $GA$. And, thus, compounding, as $BA$ (is) to
$AE$, so $DA$ (is) to $AG$ [Prop.~5.18].
And, alternately, as $BA$ (is) to $AD$, so $EA$ (is) to $AG$ [Prop.~5.16]. Thus, in parallelograms
$ABCD$ and $EG$ the sides about the common angle $BAD$ are proportional.
And since $GF$ is parallel to $DC$, angle $AFG$ is equal to $DCA$  [Prop.~1.29]. And angle $DAC$ (is) common to the two triangles $ADC$ and
$AGF$. Thus, triangle $ADC$ is equiangular to triangle $AGF$  [Prop.~1.32]. So, for the same (reasons), triangle $ACB$ is equiangular to triangle $AFE$,
and the whole parallelogram $ABCD$ is equiangular to parallelogram $EG$. Thus,
proportionally, as $AD$ (is) to $DC$, so $AG$ (is) to $GF$, and as $DC$ (is) to 
$CA$, so $GF$ (is) to $FA$, and as $AC$ (is) to $CB$, so $AF$ (is) to $FE$, and, further, 
as $CB$ (is) to $BA$, so $FE$ (is) to $EA$ [Prop.~6.4].
And since it was shown that as $DC$  is to $CA$, so $GF$ (is) to $FA$, and
as $AC$ (is) to $CB$, so $AF$ (is) to $FE$, thus, via equality, as $DC$ is to $CB$,
so $GF$ (is) to $FE$ [Prop.~5.22]. 
Thus, in parallelograms $ABCD$ and $EG$ the sides about the equal angles are
proportional. Thus, parallelogram $ABCD$ is similar to parallelogram $EG$ [Def.~6.1]. So, for the same (reasons), 
parallelogram $ABCD$ is also similar to parallelogram $KH$. Thus, parallelograms $EG$ and $HK$ are each similar to [parallelogram] $ABCD$.
And (rectilinear figures) similar to the same rectilinear figure are also
similar to one another [Prop.~6.21].
Thus, parallelogram $EG$ is also similar to parallelogram $HK$.\\

\epsfysize=2in
\centerline{\epsffile{Book06/fig24e.eps}}

Thus, in any parallelogram the parallelograms
about the diagonal are similar to the whole, and to one another.
(Which is) the very thing it was required to show.

%%%%%%
% Prop 6.25
%%%%%%
\pdfbookmark[1]{Proposition 6.25}{pdf6.25}

\begin{center}
{\large Proposition 25}
\end{center}

To
 construct a single (rectilinear figure) similar to a given rectilinear figure, and equal to
a different given rectilinear figure.

\epsfysize=1.35in
\centerline{\epsffile{Book06/fig25e.eps}}

Let $ABC$ be the given rectilinear figure to which it is required to construct
a similar (rectilinear figure), and $D$ the (rectilinear figure) to which (the constructed figure) is required (to be) equal. So it is required to
construct a single (rectilinear figure) similar to $ABC$, and equal to $D$.

For  let the parallelogram $BE$, equal to triangle $ABC$, have been applied to
(the straight-line) $BC$  [Prop.~1.44], and the parallelogram $CM$, equal to $D$, (have been applied) to (the straight-line) $CE$,  in the angle $FCE$, which is equal to $CBL$  [Prop.~1.45]. Thus, $BC$ is
straight-on to $CF$, and $LE$ to $EM$ [Prop.~1.14]. And let the mean proportion $GH$ have
been taken of $BC$ and $CF$  [Prop.~6.13].
And let $KGH$, similar, and similarly laid out, to  $ABC$ have been described
on $GH$ [Prop.~6.18].

And since as $BC$ is to $GH$, so $GH$ (is) to $CF$, and if three straight-lines are
proportional then as the first is to the third, so the figure (described) on the first
(is) to the similar, and similarly described, (figure) on the second 
[Prop.~6.19~corr.],  thus as $BC$ is to
$CF$, so triangle $ABC$ (is) to triangle $KGH$. But, also, as $BC$ (is) to
$CF$, so parallelogram $BE$ (is) to parallelogram $EF$ [Prop.~6.1]. And, thus,  as triangle $ABC$ (is) to triangle $KGH$, so parallelogram $BE$ (is) to parallelogram $EF$. 
Thus, alternately, as triangle
$ABC$ (is) to parallelogram $BE$, so triangle $KGH$ (is) to
parallelogram $EF$ [Prop.~5.16]. 
And triangle $ABC$ (is) equal to parallelogram $BE$. Thus, triangle $KGH$
(is) also equal to parallelogram $EF$. But, parallelogram $EF$ is equal to $D$. 
Thus, $KGH$ is also equal to $D$. And $KGH$ is also similar to $ABC$.

Thus, a single (rectilinear figure) $KGH$ has been constructed (which is)
similar to the given rectilinear figure $ABC$, and equal to a different
given (rectilinear figure) $D$. (Which is) the very thing it was required to do.

%%%%%%
% Prop 6.26
%%%%%%
\pdfbookmark[1]{Proposition 6.26}{pdf6.26}

\begin{center}
{\large Proposition 26}
\end{center}

If from a parallelogram a(nother) parallelogram
is subtracted (which is) similar, and similarly laid out, to the whole,
having a common angle with it, then (the subtracted parallelogram) is about the same diagonal
as the whole. 

For, from parallelogram $ABCD$, let (parallelogram) $AF$ have been
subtracted (which is) similar, and similarly laid out,  to $ABCD$,
having  the common angle $DAB$ with it. I say that $ABCD$ is about
the same diagonal as $AF$.

\epsfysize=2in
\centerline{\epsffile{Book06/fig26e.eps}}

For (if) not, then, if possible, let $AHC$ be [$ABCD$'s] diagonal. And producing
$GF$, let it have been drawn through to (point) $H$. And let $HK$ have
been drawn through (point) $H$, parallel to either of $AD$ or $BC$  [Prop.~1.31].

Therefore, since $ABCD$ is about the same diagonal as $KG$, thus as
$DA$ is to $AB$, so $GA$ (is) to $AK$ [Prop.~6.24].
And, on account of the similarity of $ABCD$ and $EG$, also, as $DA$ (is)
to $AB$, so $GA$ (is) to $AE$. Thus, also, as $GA$ (is) to $AK$, so $GA$ (is)
to $AE$. Thus, $GA$ has the same ratio  to each of $AK$ and $AE$. Thus, 
$AE$ is equal to $AK$ [Prop.~5.9], the
lesser to the greater. The very thing is impossible. Thus, $ABCD$ is not not
about the same diagonal as $AF$. Thus, parallelogram $ABCD$ is
about the same diagonal as parallelogram $AF$.

Thus, if from a parallelogram a(nother) parallelogram
is subtracted (which is) similar, and similarly laid out, to the whole,
having a common angle with it, then (the subtracted parallelogram) is about the same diagonal
as the whole. (Which is) the very thing it was required to show.

%%%%%%
% Prop 6.27
%%%%%%
\pdfbookmark[1]{Proposition 6.27}{pdf6.27}

\begin{center}
{\large Proposition 27}
\end{center}

Of all the parallelograms applied to the
same straight-line, and falling short by parallelogrammic figures
similar, and similarly laid out, to the (parallelogram)
described on half (the straight-line), the greatest is the [parallelogram]
applied to half (the straight-line) which (is) similar to (that parallelogram)
by which it falls short.

Let $AB$ be a straight-line, and let it have been cut in half at
(point) $C$ [Prop.~1.10]. And let the parallelogram $AD$ have been applied to the straight-line
$AB$, 
falling short by the parallelogrammic figure $DB$ (which is) applied to
half of $AB$---that is to say, $CB$. I say that of all  the parallelograms applied to $AB$, and
falling short by  [parallelogrammic] figures similar, and similarly
laid out, to $DB$, the greatest is $AD$. For let the parallelogram
$AF$ have been applied to the straight-line $AB$, falling short by
the parallelogrammic figure $FB$ (which is) similar, and similarly laid out,
to $DB$. I say that $AD$ is greater than $AF$.

\epsfysize=2in
\centerline{\epsffile{Book06/fig27e.eps}}

For since parallelogram $DB$ is similar to parallelogram $FB$, they are
about the same diagonal [Prop.~6.26].
Let their (common) diagonal $DB$ have been drawn, and let the (rest of the)  figure have been
described.

Therefore, since (complement) $CF$ is equal to (complement) $FE$  [Prop.~1.43],
and (parallelogram) $FB$ is common, the whole (parallelogram) $CH$ is thus
equal to the whole (parallelogram) $KE$. But, (parallelogram)
$CH$ is equal to $CG$, since $AC$ (is) also (equal) to $CB$ [Prop.~6.1]. Thus, (parallelogram) $GC$ is also equal to $EK$. 
Let (parallelogram) $CF$ have been added to both. Thus, the whole (parallelogram) $AF$
is equal to the gnomon $LMN$. Hence, parallelogram $DB$---that is
to say, $AD$---is greater than parallelogram $AF$.

Thus, for all parallelograms applied to the
same straight-line, and falling short by a parallelogrammic figure
similar, and similarly laid out, to the (parallelogram)
described on half (the straight-line), the greatest is the [parallelogram]
applied to half (the straight-line). (Which is) the very thing it was required
to show.

%%%%%%
% Prop 6.28
%%%%%%
\pdfbookmark[1]{Proposition 6.28}{pdf6.28}

\begin{center}
{\large Proposition 28}$^\dag$
\end{center}

To apply a parallelogram, equal to a given
rectilinear figure,  to a given
straight-line, (the applied parallelogram) falling short by a parallelogrammic figure similar to a given (parallelogram). It is necessary for the given rectilinear figure [to which it
is required to apply an equal (parallelogram)] not to be  greater than the (parallelogram)
described on half (of the straight-line) and similar to the deficit.

Let $AB$ be the given straight-line, and $C$ the given rectilinear figure to which
the (parallelogram) applied to $AB$ is required (to be) equal, [being] not
greater than the (parallelogram) described on half of $AB$ and
similar to the deficit, and $D$ the (parallelogram) to which the deficit is
required (to be) similar. So it is required to apply a parallelogram, equal to
the given rectilinear figure $C$, 
to the straight-line $AB$, falling short by a parallelogrammic figure which
is similar to $D$.

\epsfysize=1.8in
\centerline{\epsffile{Book06/fig28e.eps}}

Let $AB$ have been cut in half at point $E$  [Prop.~1.10], and let (parallelogram) $EBFG$, (which is)
similar, and similarly laid out, to (parallelogram) $D$, have been described on $EB$ 
[Prop.~6.18]. And let parallelogram
$AG$ have been completed.

Therefore, if $AG$ is equal to $C$ then the thing prescribed has happened. For
a parallelogram  $AG$, equal to the given rectilinear figure $C$, has been
applied to the given straight-line $AB$, falling short by a parallelogrammic
figure $GB$ which is similar to $D$. And if not,  let $HE$ be greater than $C$. 
And $HE$ (is) equal to $GB$ [Prop.~6.1]. 
Thus, $GB$ (is) also greater than $C$.
So, let (parallelogram) $KLMN$ have been constructed (so as to be) both similar, and similarly laid out, to $D$,
and equal to the excess by which $GB$ is greater than $C$ [Prop.~6.25]. But, $GB$ [is] similar to
$D$. Thus, $KM$ is also similar to $GB$
[Prop.~6.21]. Therefore, let $KL$ correspond to
$GE$, and $LM$ to $GF$. And since (parallelogram) $GB$ is equal to
(figure) $C$ and (parallelogram) $KM$, $GB$ is thus greater than $KM$. Thus,
$GE$ is also greater than $KL$, and $GF$ than $LM$. Let $GO$ be made equal to
$KL$, and $GP$ to $LM$  [Prop.~1.3]. And let the parallelogram
$OGPQ$ have been completed. Thus, [$GQ$] is equal and similar to
$KM$ [but, $KM$ is similar to $GB$]. Thus, $GQ$ is also similar to $GB$ [Prop.~6.21]. Thus, $GQ$ and $GB$ are about the
same diagonal [Prop.~6.26]. 
Let $GQB$ be their (common) diagonal, and let the (remainder of the) figure have been described.

Therefore, since $BG$ is equal to $C$ and $KM$, of which $GQ$ is equal
to $KM$, the remaining gnomon $UWV$ is thus equal to the remainder $C$. 
And since (the complement) $PR$ is equal to (the complement) $OS$  [Prop.~1.43], let (parallelogram) $QB$ have been added to both. Thus, the whole
(parallelogram) $PB$ is equal to the whole (parallelogram) $OB$. But,
$OB$ is equal to $TE$, since side $AE$ is equal to side $EB$ [Prop.~6.1]. Thus, $TE$ is also equal to $PB$. 
Let (parallelogram) $OS$ have been added to both. Thus, the whole (parallelogram) $TS$ is equal to the gnomon $VWU$. But, gnomon $VWU$ was shown (to be)
equal to $C$.  Therefore, (parallelogram) $TS$ is also equal to (figure) $C$.

Thus, the parallelogram $ST$, equal to the given rectilinear figure
$C$, has been applied to the given straight-line $AB$, falling
short by the parallelogrammic figure $QB$, which is similar to $D$ [inasmuch as
$QB$ is similar to $GQ$ [Prop.~6.24]\,]. 
(Which is) the very thing it was required to do.
{\footnotesize\noindent$^\dag$ This proposition is a geometric
solution of the quadratic equation $x^2 - \alpha\,x +\beta = 0$. Here,
$x$ is the ratio of a side of the deficit to the corresponding side of figure $D$, $\alpha$
is the ratio of the length of $AB$ to the length of that side of figure $D$ which corresponds to the side of the deficit running along $AB$, and $\beta$ is the
ratio of the areas of figures $C$ and $D$. The constraint corresponds to the
condition $\beta < \alpha^2/4$ for the equation to have real roots. Only the
smaller root of the equation is found. The larger root can be found by a
similar method.}

%%%%%%
% Prop 6.29
%%%%%%
\pdfbookmark[1]{Proposition 6.29}{pdf6.29}

\begin{center}
{\large Proposition 29}$^\dag$
\end{center}

To apply a parallelogram, equal to a given
rectilinear figure,  to a given
straight-line, (the applied parallelogram) overshooting by a parallelogrammic figure similar to a given (parallelogram).

\epsfysize=1.8in
\centerline{\epsffile{Book06/fig29e.eps}}

Let $AB$ be the given straight-line, and $C$ the given rectilinear figure to which the (parallelogram) applied to $AB$ is required (to be) equal, and $D$ the (parallelogram) to which the excess is required (to be) similar. So
it is required to apply a parallelogram, equal to the given rectilinear figure
$C$, to the given straight-line $AB$, overshooting by a parallelogrammic
figure  similar to $D$.

Let $AB$ have been cut in half at (point) $E$  [Prop.~1.10], and
let the parallelogram $BF$, (which is) similar, and similarly laid out, to $D$,
have been described on $EB$ [Prop.~6.18].
And let (parallelogram) $GH$ have been constructed (so as to be) both similar, and similarly laid out, to $D$, and equal to the sum of $BF$ and $C$ [Prop.~6.25]. And let $KH$ correspond to
$FL$, and $KG$ to $FE$. And since (parallelogram) $GH$ is greater than 
(parallelogram) $FB$, $KH$ is thus
also greater than $FL$, and $KG$ than $FE$.  Let $FL$ and $FE$ have been produced,
and let $FLM$ be (made) equal to $KH$, and $FEN$ to $KG$  [Prop.~1.3]. 
And let (parallelogram) $MN$ have been completed. Thus, $MN$ is equal and
similar  to $GH$. But, $GH$ is similar to $EL$. Thus, $MN$  is also similar to
$EL$ [Prop.~6.21]. $EL$ is thus about the
same diagonal as $MN$ [Prop.~6.26]. 
Let their (common) diagonal $FO$ have been drawn, and let the (remainder
of the) figure have been described.

And since (parallelogram) $GH$ is equal to (parallelogram) $EL$ and (figure) $C$,
but $GH$ is equal to (parallelogram) $MN$, $MN$ is thus also equal to $EL$ and
$C$. Let $EL$ have been subtracted from both. Thus, the remaining gnomon
$XWV$ is equal to (figure) $C$. And since $AE$ is equal to $EB$, (parallelogram)
$AN$ is also equal to (parallelogram) $NB$ [Prop.~6.1], that is to say, (parallelogram) $LP$  [Prop.~1.43]. Let
(parallelogram) $EO$ have been added to both. Thus, the whole (parallelogram)
$AO$ is equal to the gnomon $VWX$. But, the gnomon $VWX$ is equal to (figure)
$C$. Thus, (parallelogram) $AO$ is also equal to (figure) $C$.

Thus, the parallelogram $AO$, equal to the given rectilinear figure $C$, has been applied to the given straight-line
$AB$, overshooting by the parallelogrammic figure $QP$ which is similar to $D$,
since $PQ$ is also similar to $EL$ [Prop.~6.24].
(Which is) the very thing it was required to do.
{\footnotesize\noindent$^\dag$ This proposition is a geometric
solution of the quadratic equation $x^2 + \alpha\,x -\beta = 0$. Here,
$x$ is the ratio of a side of the excess to the corresponding side of figure $D$, $\alpha$
is the ratio of the length of $AB$ to the length of that side of figure $D$ which corresponds to the side of the excess running along $AB$, and $\beta$ is the
ratio of the areas of figures $C$ and $D$. Only the positive root of the equation is found.}

%%%%%%
% Prop 6.30
%%%%%%
\pdfbookmark[1]{Proposition 6.30}{pdf6.30}

\begin{center}
{\large Proposition 30}$^\dag$
\end{center}

To cut a given finite straight-line in extreme and
mean ratio.

\epsfysize=2.2in
\centerline{\epsffile{Book06/fig30e.eps}}

Let $AB$ be the given finite straight-line. So it is required to cut the straight-line $AB$ in extreme and mean ratio.

Let the square $BC$ have been described on $AB$  [Prop. 1.46],
and let the parallelogram $CD$, equal to $BC$, have been applied to $AC$,
overshooting by the figure $AD$ (which is) similar to $BC$ [Prop.~6.29].

And $BC$ is a square. Thus, $AD$ is also a square. And since $BC$ is equal to
$CD$, let (rectangle) $CE$ have been subtracted from both. Thus, the remaining (rectangle) $BF$
is equal to the remaining (square) $AD$. And it is also equiangular to it. 
Thus, the sides of $BF$ and $AD$ about the equal angles are reciprocally proportional [Prop.~6.14]. 
Thus, as $FE$ is to $ED$, so $AE$ (is) to $EB$. And $FE$ (is) equal to $AB$, and $ED$ to
$AE$. Thus, as $BA$ is to $AE$, so $AE$ (is) to $EB$.  And $AB$ (is)
 greater than $AE$. Thus, $AE$ (is) also greater than $EB$ [Prop.~5.14].
 
 Thus, the straight-line $AB$ has been cut in extreme and mean ratio at $E$,
 and $AE$ is its greater piece. (Which is) the very thing it was required to
 do.
{\footnotesize\noindent$^\dag$ This method of cutting a straight-line
is sometimes called the ``Golden Section''---see Prop.~2.11.}

%%%%%%
% Prop 6.31
%%%%%%
\pdfbookmark[1]{Proposition 6.31}{pdf6.31}

\begin{center}
{\large Proposition 31}
\end{center}

In right-angled triangles, the figure (drawn)
on the side subtending the right-angle is equal to the (sum of the) similar, and similarly
described, figures on the sides surrounding the right-angle.

\epsfysize=2in
\centerline{\epsffile{Book06/fig31e.eps}}

Let $ABC$ be a right-angled triangle having the angle $BAC$ a right-angle. I say
that the figure (drawn) on $BC$ is equal to the (sum of the) similar,
and similarly described, figures on $BA$ and $AC$.

Let the perpendicular $AD$ have been drawn  [Prop. 1.12].

Therefore, since, in the right-angled triangle  $ABC$, the (straight-line) $AD$ has been drawn from the right-angle
at $A$ perpendicular to the base $BC$, the triangles $ABD$
and $ADC$ about the perpendicular are similar to the whole (triangle) $ABC$,
and to one another  [Prop.~6.8].
And since $ABC$ is similar to $ABD$, thus as $CB$ is to $BA$, so $AB$ (is) to
$BD$ [Def.~6.1]. And since three straight-lines
are proportional, as the first is to the third, so the figure (drawn) on the
first is to the similar, and similarly described, (figure) on the second 
[Prop.~6.19~corr.]. Thus, as $CB$ (is) to
$BD$, so the figure (drawn) on $CB$ (is) to the similar, and similarly
described, (figure) on $BA$. And so, for the same (reasons), as $BC$ (is) to
$CD$, so the figure  (drawn) on $BC$ (is) to the (figure) on $CA$. Hence,
also, as $BC$ (is) to $BD$ and $DC$, so the figure (drawn) on $BC$ (is) to
the (sum of the) similar, and similarly described, (figures) on $BA$ and $AC$ [Prop.~5.24].
And $BC$ is equal to $BD$ and $DC$. Thus, the figure (drawn) on $BC$
(is) also equal to the (sum of the) similar, and similarly described, figures
on $BA$ and $AC$ [Prop.~5.9].

Thus, in right-angled triangles, the figure (drawn)
on the side subtending the right-angle is equal to the (sum of the) similar, and similarly
described, figures on the sides surrounding the right-angle. (Which is)
the very thing it was required to show.

%%%%%%
% Prop 6.32
%%%%%%
\pdfbookmark[1]{Proposition 6.32}{pdf6.32}

\begin{center}
{\large Proposition 32}
\end{center}

If two triangles,
having two sides proportional to two sides, are placed together at a single angle such that the corresponding 
sides are also parallel, then the remaining sides of the triangles will
be straight-on (with respect to one another).

\epsfysize=2in
\centerline{\epsffile{Book06/fig32e.eps}}

Let $ABC$ and $DCE$ be two triangles having the two sides $BA$ and $AC$
proportional to the two sides $DC$ and $DE$---so that as $AB$ (is) to $AC$, so
$DC$ (is) to $DE$---and (having side) $AB$ parallel to $DC$, and $AC$ to $DE$. I say that (side) $BC$
is straight-on to $CE$.

For since $AB$ is parallel to $DC$, and the straight-line $AC$ has fallen across them, the alternate angles $BAC$ and $ACD$ are equal to one another  [Prop.~1.29]. So, for the same (reasons), $CDE$ is also equal to $ACD$. 
And, hence, $BAC$ is equal to $CDE$. And since $ABC$ and $DCE$ are two triangles
having the one angle at $A$ equal to the one angle at $D$, and the sides about the
equal angles proportional, (so that) as $BA$ (is) to $AC$, so $CD$ (is) to $DE$, triangle $ABC$ is thus equiangular to triangle $DCE$ [Prop.~6.6].  Thus, angle $ABC$ is equal to $DCE$.
And (angle) $ACD$ was also shown (to be) equal to $BAC$. Thus, the whole
(angle) $ACE$ is equal to the two (angles) $ABC$ and $BAC$. Let $ACB$ have
been added to both. Thus, $ACE$ and $ACB$ are equal to $BAC$, $ACB$,  and
$CBA$. But, $BAC$, $ABC$, and $ACB$ are equal to two right-angles  [Prop.~1.32]. Thus, $ACE$ and $ACB$ are also equal to two right-angles.
Thus, the two straight-lines $BC$ and $CE$, not lying on the same side, 
make  adjacent angles $ACE$ and $ACB$ (whose sum is) equal to two right-angles with some straight-line $AC$,  at the
point $C$ on it. Thus, $BC$ is straight-on to
$CE$  [Prop.~1.14].

Thus, if  two triangles,
having two sides proportional to two sides, are placed together at a single angle such that the corresponding 
sides are also parallel, then the remaining sides of the triangles will
be straight-on (with respect to one another). (Which is) the very thing it
was required to show.

%%%%%%
% Prop 6.33
%%%%%%
\pdfbookmark[1]{Proposition 6.33}{pdf6.33}

\begin{center}
{\large Proposition 33}
\end{center}

In equal circles,  angles have the same ratio
as the (ratio of the) circumferences on which they stand, whether they are standing at
the centers (of the circles) or at the circumferences.

\epsfysize=1.5in
\centerline{\epsffile{Book06/fig33e.eps}}

Let $ABC$ and $DEF$ be equal circles, and let $BGC$ and $EHF$ be angles
at their centers, $G$ and $H$ (respectively), and $BAC$ and $EDF$ (angles)
at their circumferences. I say that as circumference $BC$ is to circumference
$EF$, so angle $BGC$ (is) to $EHF$, and (angle) $BAC$ to $EDF$.

For let any number whatsoever of consecutive (circumferences), $CK$ and
$KL$, be made equal to circumference $BC$, and any number whatsoever, $FM$ and $MN$, to circumference $EF$. And let $GK$, $GL$, $HM$, and $HN$ have been
joined.

Therefore, since circumferences $BC$, $CK$, and $KL$ are equal to one
another, angles $BGC$, $CGK$, and $KGL$ are also equal to
one another [Prop.~3.27].  Thus, as many times as circumference $BL$
is (divisible) by $BC$, so many times is angle $BGL$ also
(divisible) by $BGC$. And so, for the same (reasons), as many times as
circumference $NE$ is (divisible) by $EF$, so many times is
angle $NHE$ also (divisible) by $EHF$. Thus, if circumference $BL$
is equal to circumference $EN$ then angle $BGL$ is also equal to $EHN$ [Prop.~3.27], and
if circumference $BL$ is greater than circumference $EN$ then angle
$BGL$ is also greater than $EHN$,$^\dag$ and if ($BL$ is) less (than $EN$ then $BGL$ is also) less (than $EHN$). So there are four magnitudes, two circumferences
$BC$ and $EF$, and two angles $BGC$ and $EHF$. And equal multiples
have been taken of circumference $BC$ and angle $BGC$, (namely)
circumference $BL$ and angle $BGL$, and of circumference $EF$ and angle
$EHF$, (namely) circumference $EN$ and angle $EHN$. And it has been
shown that if circumference $BL$ exceeds circumference
$EN$ then angle $BGL$ also exceeds angle $EHN$, and if ($BL$ is) equal (to $EN$
then $BGL$ is also) equal (to $EHN$), and if ($BL$ is) less (than $EN$
then $BGL$ is also) less (than $EHN$). Thus, as circumference
$BC$ (is) to $EF$, so angle $BGC$ (is) to $EHF$ [Def.~5.5]. But as
angle $BGC$ (is) to $EHF$, so  (angle) $BAC$ (is) to $EDF$ [Prop.~5.15]. For the former (are) double the
latter (respectively) [Prop.~3.20]. Thus, also, as circumference $BC$ (is)
to circumference $EF$, so  angle $BGC$ (is) to $EHF$, and $BAC$ to $EDF$.

Thus, in equal circles,  angles have the same ratio
as the (ratio of the) circumferences on which they stand, whether they are standing at
the centers (of the circles) or at the circumferences. (Which is) the very thing it was required to show.
{\footnotesize\noindent$^\dag$ This is a straight-forward generalization of Prop.~3.27}
