%%%%%%
% BOOK 9
%%%%%%
\pdfbookmark[0]{Book 9}{book9}
\pagestyle{plain}
\begin{center}
{\Huge ELEMENTS BOOK 9}\\
\spa\spa\spa
{\huge\it Applications of Number Theory\symbolfootnote[2]{The propositions contained in Books 7--9 are generally attributed to the
school of Pythagoras.}}
\end{center}\newpage

%%%%%%
% Prop 9.1
%%%%%%
\pdfbookmark[1]{Proposition 9.1}{pdf9.1}
\pagestyle{fancy}
\cfoot{\gr{\thepage}}
\lhead{\large\gr{STOIQEIWN \ggn{9}.}}
\rhead{\large ELEMENTS BOOK 9}

\begin{center}
{\large Proposition 1}
\end{center}

If two similar plane numbers make some (number by) multiplying one another then the created (number) will be  square.

\epsfysize=1.2in
\centerline{\epsffile{Book09/fig01e.eps}}

Let $A$ and $B$ be two similar plane numbers, and let $A$ make $C$
(by) multiplying $B$. I say that $C$ is square.

For let $A$ make $D$ (by) multiplying itself. $D$ is thus square. Therefore, since $A$ has made $D$ (by) multiplying itself, and
has made $C$ (by) multiplying $B$, thus as $A$ is to $B$, so $D$ (is) to $C$ [Prop.~7.17]. And since $A$ and $B$ are similar
plane numbers, one number thus falls (between) $A$ and $B$ in mean
proportion [Prop.~8.18]. And if (some) numbers fall between two  numbers in continued proportion then, as many (numbers)
as fall in  (between) them (in continued proportion), so many also (fall) in (between numbers) having
the same ratio (as them in continued proportion)  [Prop.~8.8].
And hence one number falls (between) $D$ and $C$ in mean proportion.
And $D$ is  square. Thus, $C$ (is) also  square [Prop.~8.22]. (Which is)
the very thing it was required to show.

%%%%%%
% Prop 9.2
%%%%%%
\pdfbookmark[1]{Proposition 9.2}{pdf9.2}

\begin{center}
{\large Proposition 2}
\end{center}

If two numbers make a square (number by)
multiplying one another then they are similar plane numbers.

\epsfysize=1.3in
\centerline{\epsffile{Book09/fig02e.eps}}

Let $A$ and $B$ be two numbers, and let $A$ make the square (number)
$C$ (by) multiplying $B$. I say that $A$ and $B$ are similar plane numbers.

For let $A$ make $D$ (by) multiplying itself. Thus, $D$ is square. And since $A$ has made $D$ (by) multiplying itself, and has made $C$ (by)
multiplying $B$, thus as $A$ is to $B$, so $D$ (is) to $C$ [Prop.~7.17]. And since $D$ is  square, and $C$ (is) also, $D$ and $C$ are thus similar plane numbers. Thus, 
one (number) falls (between) $D$ and $C$ in mean proportion [Prop.~8.18]. And as $D$ is to $C$, so $A$ (is) to $B$. Thus, one (number) also falls (between) $A$ and $B$ in mean proportion [Prop.~8.8].  And if one (number)
falls (between) two numbers in mean proportion then [the] numbers
are similar plane (numbers)  [Prop.~8.20].
Thus, $A$ and $B$ are similar plane (numbers). (Which is) the
very thing it was required to show.

%%%%%%
% Prop 9.3
%%%%%%
\pdfbookmark[1]{Proposition 9.3}{pdf9.3}

\begin{center}
{\large Proposition 3}
\end{center}

If a cube number makes some (number by) multiplying itself then the created (number) will be cube.

\epsfysize=1.2in
\centerline{\epsffile{Book09/fig03e.eps}}

For let the cube number $A$ make $B$ (by) multiplying itself. I say
that $B$ is  cube.

For let the side $C$ of $A$ have been taken. And let $C$ make
$D$ by multiplying itself. So it is clear that $C$ has made $A$ (by)
multiplying $D$. And since $C$ has made $D$ (by) multiplying itself,
$C$ thus measures $D$ according to the units in it [Def.~7.15]. But, in fact, a unit also measures
$C$ according to the units in it [Def.~7.20]. 
Thus, as a unit is to $C$, so $C$ (is) to $D$. Again, since $C$ has
made $A$ (by) multiplying $D$, $D$ thus measures $A$ according to
the units in $C$. And a unit also measures $C$ according to the units in it. Thus, as a unit is to $C$, so $D$ (is) to $A$. But, as a unit (is) to $C$, so
$C$ (is) to $D$. And thus as a unit (is) to $C$, so $C$ (is) to $D$, and
$D$ to $A$. Thus, two numbers, $C$ and $D$, have fallen (between) a unit
and the number $A$ in continued  mean proportion. Again, since $A$ has made $B$ (by) multiplying itself, $A$ thus measures $B$ according to the units in it.
And a unit also measures $A$ according to the units in it. Thus, as a
unit is to $A$, so $A$ (is) to $B$. And two numbers have
fallen (between) a unit and $A$ in mean proportion. Thus two
numbers will also fall (between) $A$ and $B$ in mean proportion [Prop.~8.8].
And if two
(numbers) fall (between) two numbers in mean proportion, and the
first (number) is cube, then the second will also be  cube 
[Prop.~8.23]. And $A$ is cube.
Thus, $B$ is also  cube. (Which is) the very thing it was required to show.

%%%%%%
% Prop 9.4
%%%%%%
\pdfbookmark[1]{Proposition 9.4}{pdf9.4}

\begin{center}
{\large Proposition 4}
\end{center}

If a cube number makes some (number by) multiplying a(nother) cube number then the created (number) will be  cube.

\epsfysize=1.2in
\centerline{\epsffile{Book09/fig04e.eps}}

For let the cube number $A$ make $C$ (by) multiplying the cube number $B$. I say that $C$ is cube.

For let $A$ make $D$ (by) multiplying itself.  Thus, $D$ is  cube 
[Prop.~9.3]. And since $A$ has made
$D$ (by) multiplying itself, and has made $C$ (by) multiplying $B$, 
thus as $A$ is to $B$, so $D$ (is) to $C$ [Prop.~7.17].  And since $A$ and $B$ are cube, $A$ and $B$ are similar solid (numbers). Thus, two numbers fall
(between) $A$ and $B$ in mean proportion [Prop.~8.19]. Hence, two numbers will
also fall (between) $D$ and $C$ in mean proportion [Prop.~8.8]. And $D$ is  cube. Thus,
$C$ (is) also cube  [Prop.~8.23]. 
(Which is) the very thing it was required to show.

%%%%%%
% Prop 9.5
%%%%%%
\pdfbookmark[1]{Proposition 9.5}{pdf9.5}

\begin{center}
{\large Proposition 5}
\end{center}

If a cube number makes a(nother) cube number
(by) multiplying some (number) then the (number) multiplied will also be 
cube.

\epsfysize=1.2in
\centerline{\epsffile{Book09/fig05e.eps}}

For let the cube number $A$ make the cube (number) $C$ (by) multiplying
some number $B$. I say that $B$ is cube.

For let $A$ make $D$ (by) multiplying itself. $D$ is thus  cube [Prop.~9.3]. And since $A$ has made $D$ (by)
multiplying itself, and has made $C$ (by) multiplying $B$, thus as $A$ is to
$B$, so $D$ (is) to $C$ [Prop.~7.17]. And since
$D$ and $C$ are (both) cube, they are similar solid (numbers). Thus,
two numbers fall (between) $D$ and $C$ in mean proportion [Prop.~8.19]. And as $D$  is to $C$, so $A$ (is) to $B$.  Thus, two numbers also fall (between) $A$ and $B$ in mean proportion
[Prop.~8.8]. And $A$ is  cube. Thus,
 $B$ is also  cube  [Prop.~8.23]. (Which is) the very thing it was required to show.

%%%%%%
% Prop 9.6
%%%%%%
\pdfbookmark[1]{Proposition 9.6}{pdf9.6}

\begin{center}
{\large Proposition 6}
\end{center}

If a number makes a cube (number by) multiplying itself then it itself will also be  cube.

\epsfysize=0.8in
\centerline{\epsffile{Book09/fig06e.eps}}

For let the number $A$ make the cube (number) $B$ (by) multiplying itself.
I say that $A$ is also cube.

For let $A$ make $C$ (by) multiplying $B$. Therefore, since $A$
has made $B$ (by) multiplying itself, and has made $C$ (by) multiplying 
$B$, $C$ is thus cube. And since $A$ has made $B$ (by)
multiplying itself, $A$ thus measures $B$ according to the units in ($A$).
And a unit also measures $A$ according to the units in it. Thus, as
a unit is to $A$, so $A$ (is) to $B$. And since $A$ has made $C$ (by)
multiplying $B$, $B$ thus measures $C$ according to the units in $A$.
And a unit also measures $A$ according to the units in it. Thus, as
a unit is to $A$, so $B$ (is) to $C$. But, as a unit (is) to $A$, so $A$ (is)
to $B$. And thus as $A$ (is) to $B$, (so) $B$ (is) to $C$. And since $B$
and $C$ are cube, they are similar solid (numbers). Thus,
there exist two numbers in mean proportion (between) $B$ and $C$ [Prop.~8.19]. And as $B$ is to $C$, (so) $A$ (is)
to $B$. Thus, there also exist two numbers in mean proportion (between) $A$ and
$B$ [Prop.~8.8].  And $B$ is  cube. Thus, $A$
is also  cube   [Prop.~8.23]. (Which is)
the very thing it was required to show.

%%%%%%
% Prop 9.7
%%%%%%
\pdfbookmark[1]{Proposition 9.7}{pdf9.7}

\begin{center}
{\large Proposition 7}
\end{center}

If a composite number makes some (number by)
multiplying some (other) number then the created (number) will be  solid.

\epsfysize=1.5in
\centerline{\epsffile{Book09/fig07e.eps}}

For let the composite number $A$ make $C$ (by) multiplying some number
$B$. I say that $C$ is  solid.

For since $A$ is a composite (number), it will be measured by some
number. Let it be measured by $D$.  And, as many times as $D$ measures
$A$, so many units let there be in $E$. Therefore, since $D$ measures $A$
according to the units in $E$, $E$ has thus made $A$ (by) multiplying
$D$ [Def.~7.15]. And since $A$ has made $C$ (by) multiplying $B$, and $A$ is
the (number created) from (multiplying) $D$, $E$, the (number created)
from (multiplying) $D$, $E$ has thus made $C$ (by) multiplying $B$. Thus,
$C$ is solid, and its sides are $D$, $E$, $B$. (Which is) the
very thing it was required to show.

%%%%%%
% Prop 9.8
%%%%%%
\pdfbookmark[1]{Proposition 9.8}{pdf9.8}

\begin{center}
{\large Proposition 8}
\end{center}

If any multitude whatsoever of numbers is continuously proportional, (starting) from a unit, then the third from the unit will be 
square, and (all) those (numbers after that) which leave an interval of  one (number), and the fourth (will
be) cube, and all those (numbers after that) which leave an interval of two (numbers), and the seventh
(will be) both cube and square, and (all) those (numbers after that) which leave an interval of five (numbers).

\epsfysize=1.8in
\centerline{\epsffile{Book09/fig08e.eps}}

Let any multitude whatsoever of numbers, $A$, $B$, $C$, $D$, $E$, $F$, be continuously proportional, (starting) from a unit. I say that the third from the unit, $B$, is square, and all those (numbers after that) which leave an interval of one (number). And the fourth (from the unit), $C$,
(is)  cube, and all those (numbers after that) which leave an
interval of two (numbers). And the seventh (from the unit), $F$, (is)
both cube and square,  and all those (numbers after that) which leave an interval of five (numbers).

For since as the unit is to $A$, so $A$ (is) to $B$, the unit thus measures the number $A$
the same number of times as $A$ (measures) $B$ [Def.~7.20]. And the unit measures the number $A$
according to the units in it. Thus, $A$ also measures $B$ according to the units
in $A$. $A$ has thus made $B$ (by) multiplying itself [Def.~7.15].  Thus, $B$ is  square. And
since $B$, $C$, $D$ are continuously proportional, and $B$ is  square, $D$ is thus also  square  
[Prop.~8.22]. So, for the same (reasons), $F$ is also
 square. So, similarly, we can also show that all those (numbers after that) which leave an interval of  one (number) are square. 
So I also say that the fourth (number) from the unit, $C$,  is  cube, and
all those (numbers after that) which leave an interval of two (numbers).
For since as the unit is to $A$, so $B$ (is) to $C$, the unit thus measures the
number $A$ the same number of times that $B$ (measures) $C$. And the
unit measures the number $A$ according to the units in $A$. And thus
$B$ measures $C$ according to the units in $A$.  $A$ has thus made $C$
(by) multiplying $B$. Therefore, since $A$ has made $B$ (by) multiplying
itself, and has made $C$ (by) multiplying $B$, $C$ is thus  cube. And since $C$, $D$, $E$, $F$ are continuously proportional, and
$C$ is cube, $F$ is thus also  cube  [Prop.~8.23]. And it was also shown (to be) 
square. Thus, the seventh (number) from the unit is (both) 
cube and  square. So, similarly, we can show that  all those (numbers after that) which leave an interval of five (numbers) are (both)
cube and square. (Which is) the very thing it was required to show.

%%%%%%
% Prop 9.9
%%%%%%
\pdfbookmark[1]{Proposition 9.9}{pdf9.9}

\begin{center}
{\large Proposition 9}
\end{center}

If  any multitude whatsoever of numbers is continuously proportional, (starting) from a unit, and the (number) after the
unit is square, then all the remaining (numbers) will also 
be square. And if the (number) after the unit is  cube, then
all the remaining (numbers) will also be cube.

\epsfysize=1.8in
\centerline{\epsffile{Book09/fig08e.eps}}

Let any multitude whatsoever of numbers, $A$, $B$, $C$, $D$, $E$, $F$, be continuously proportional, (starting) from a unit. And let the (number) after the
unit, $A$, be square. I say that all the remaining (numbers)
will also be square.

In fact, it has (already) been shown that the third (number) from the unit, $B$, is
square, and all those (numbers after that) which leave an interval of one (number) [Prop.~9.8]. [So] I say that all the remaining (numbers) are also square. For since $A$, $B$, $C$
are continuously proportional, and $A$ (is)  square, $C$
is [thus] also  square [Prop.~8.22]. Again, 
since $B$, $C$, $D$ are [also] continuously proportional, and $B$ is
 square, $D$ is [thus] also square [Prop.~8.22]. So, similarly,
we can show that all the remaining (numbers) are also square.

And so let $A$ be  cube. I say that all the remaining (numbers)
are also cube.

In fact, it has (already) been shown that the fourth (number) from the unit, $C$, is
cube, and all those (numbers after that) which leave an interval of two (numbers) [Prop.~9.8]. [So] I say
that all the remaining (numbers) are also cube. For since as the
unit is to $A$, so $A$ (is) to $B$, the unit thus measures $A$ the same number of times as $A$ (measures) $B$. And the unit measures $A$ according to the units in it. Thus, $A$ also measures $B$ according to the
units in ($A$).
$A$ has thus made $B$ (by) multiplying itself. 
And $A$ is  cube. And if a cube number makes some (number by)
multiplying itself then the created (number) is cube  [Prop.~9.3]. Thus, $B$ is also cube.
And since the four numbers $A$, $B$, $C$, $D$ are continuously proportional, and $A$ is  cube, $D$ is thus also  cube 
 [Prop.~8.23]. So, for the same (reasons), $E$ is also
 cube, and, similarly, all the remaining (numbers) are cube. (Which is) the very thing it was required to show.

%%%%%%
% Prop 9.10
%%%%%%
\pdfbookmark[1]{Proposition 9.10}{pdf9.10}

\begin{center}
{\large Proposition 10}
\end{center}

If any multitude whatsoever of numbers is [continuously] proportional, (starting) from a unit, and the (number) after the
unit is not  square, then no other (number) will be  square
either, apart from the third from the unit, and all those (numbers after that) which leave an interval of  one (number). And if the (number) after the unit is not cube, then no other (number) will be cube
 either, apart from the fourth from the unit, and all those (numbers after that) which leave an interval of  two (numbers).
 
\epsfysize=1.8in
\centerline{\epsffile{Book09/fig08e.eps}}
 
Let any multitude whatsoever of numbers, $A$, $B$, $C$, $D$, $E$, $F$, be continuously proportional, (starting) from a unit. And let the (number) after the unit, $A$, not be  square. I say that  no other (number)  will
be square either, apart from the third from the unit [and (all) those (numbers after that) which leave an interval of  one (number)].

For, if possible, let $C$ be square.  And $B$ is also square [Prop.~9.8]. Thus, $B$ and $C$ have to one another (the) ratio which (some) square number (has) to (some other) square number. And as $B$ is to $C$, (so) $A$ (is) to $B$. Thus, $A$ and $B$ have
to one another (the) ratio which (some) square number has to (some other)
square number.  Hence, $A$ and $B$ are similar plane (numbers) [Prop.~8.26]. And $B$ is square.
Thus, $A$ is also  square. The very opposite thing was assumed.
$C$ is thus not square. So, similarly, we can show that no other
(number is) square either, apart from the  third from the unit, and (all) those (numbers after that) which leave an interval of  one (number).

And so let $A$ not be  cube. I say that  no other (number)  will
be cube either, apart from the fourth from the unit, and (all) those (numbers after that) which leave an interval of  two (numbers).

For, if possible, let $D$ be  cube. And $C$ is also  cube 
[Prop.~9.8]. For it is the fourth (number) from the unit.
And as $C$ is to $D$,  (so) $B$ (is) to $C$. And $B$ thus has to $C$ the
ratio which (some) cube (number has) to (some other) cube (number). And $C$
is  cube. Thus, $B$ is also cube [Props.~7.13, 8.25].
And since as the unit is to $A$,  (so) $A$ (is) to $B$, and the unit measures
$A$ according to the units in it, $A$ thus also measures $B$ according to the units in ($A$). Thus, $A$ has made the cube  (number) $B$ (by) multiplying
itself. And if a number makes a cube (number by) multiplying itself then
it itself will be  cube  [Prop.~9.6]. Thus, $A$
(is) also  cube. The very opposite thing was assumed. Thus, $D$ is not
cube. So, similarly, we can show that no other (number) is cube
either, apart from the fourth from the unit, and (all) those (numbers after that) which leave an interval of  two (numbers). (Which is) the very thing it was required to show.

%%%%%%
% Prop 9.11
%%%%%%
\pdfbookmark[1]{Proposition 9.11}{pdf9.11}

\begin{center}
{\large Proposition 11}
\end{center}

If any multitude whatsoever of numbers is continuously proportional, (starting) from a unit, then a lesser (number)
measures a greater according to some existing (number)  among
the proportional numbers.

\epsfysize=1.6in
\centerline{\epsffile{Book09/fig11e.eps}}

Let any multitude whatsoever of numbers,  $B$, $C$, $D$, $E$, be continuously proportional, (starting) from the unit $A$. I say that, for $B$, $C$, $D$, $E$, the least (number), $B$, measures $E$ according to some (one) of
$C$, $D$.

For since as the unit $A$ is to $B$, so $D$ (is) to $E$,  the unit $A$ thus measures the number $B$ the same number of times as $D$ (measures) $E$. Thus,
alternately, the unit $A$ measures $D$ the same number of times as $B$
(measures) $E$ [Prop.~7.15]. And the unit $A$ measures $D$ according to the units in it. Thus, $B$ also measures $E$ according to the units in $D$. Hence, the lesser (number) $B$
measures the greater $E$ according to some existing number among the
proportional numbers (namely, $D$). \\

\begin{center}
{\large Corollary}
\end{center}\vspace*{-7pt}

And (it is) clear that what(ever relative) place the measuring (number) has from the unit,
the  (number) according to which it measures has the same (relative) place from the measured (number), in (the direction of the number)
before it. (Which is) the very thing it was required to show.

%%%%%%
% Prop 9.12
%%%%%%
\pdfbookmark[1]{Proposition 9.12}{pdf9.12}

\begin{center}
{\large Proposition 12}
\end{center}

If  any multitude whatsoever of numbers is continuously proportional, (starting) from a unit, then  however  many prime numbers  the
last (number) is measured by, the (number) next to the unit will also be measured by the same (prime numbers).

\epsfysize=1.1in
\centerline{\epsffile{Book09/fig12e.eps}}

Let any multitude whatsoever of numbers,  $A$, $B$, $C$, $D$, be (continuously) proportional, (starting) from a unit. I say that however many prime
numbers $D$ is measured by, $A$ will also be measured by the same
(prime numbers).

For let $D$ be measured by some prime number $E$. I say that $E$ measures $A$. For (suppose it does) not. $E$ is prime, and every prime number is prime to
every number which it does not measure [Prop.~7.29]. Thus, $E$ and $A$ are prime to one another. And since $E$ measures $D$, let it measure it according to $F$. 
Thus, $E$ has made $D$ (by) multiplying $F$.  Again, since $A$ measures $D$ according to the units in $C$ [Prop.~9.11~corr.],
$A$ has thus made $D$ (by) multiplying $C$. But, in fact, $E$ has also
made $D$ (by) multiplying $F$. Thus, the (number created) from (multiplying)
$A$, $C$ is equal to  the (number created) from (multiplying) $E$, $F$.
Thus, as $A$ is to $E$, (so) $F$ (is) to $C$ [Prop.~7.19]. And $A$ and $E$ (are)  prime (to one another), and (numbers) prime (to one another are) also the least (of those
numbers having the same ratio as them) [Prop.~7.21],
and the least (numbers) measure those (numbers) having the same ratio as them an equal number of times, the leading (measuring) the leading, and the
following the following [Prop.~7.20]. Thus, $E$
measures $C$. Let it measure it according to $G$. Thus, $E$ has made $C$
(by) multiplying $G$. But, in fact, via the (proposition) before this, $A$
has also made $C$ (by) multiplying $B$ [Prop.~9.11~corr.].
 Thus, the (number created) from (multiplying) $A$, $B$ is equal to the
 (number created) from (multiplying) $E$, $G$. Thus, as $A$ is to $E$, (so) $G$ (is) to $B$ [Prop.~7.19]. And $A$ and $E$
 (are) prime (to one another), and (numbers) prime (to one another are) also the least (of those
numbers having the same ratio as them) [Prop.~7.21],
and the least (numbers) measure those (numbers) having the same ratio as them an equal number of times, the leading (measuring) the leading, and the
following the following [Prop.~7.20]. Thus, $E$
measures $B$. Let it measure it according to $H$. Thus, $E$ has made $B$ (by) multiplying $H$. But, in fact, $A$ has also made $B$
(by) multiplying itself [Prop.~9.8].  Thus, the (number created) from (multiplying) $E$, $H$ is equal to the (square) on $A$.
Thus, as $E$ is to $A$, (so) $A$ (is) to $H$ [Prop.~7.19]. And $A$ and $E$
 are prime (to one another), and (numbers) prime (to one another are) also the least (of those
numbers having the same ratio as them) [Prop.~7.21],
and the least (numbers) measure those (numbers) having the same ratio  as them an equal number of times, the leading (measuring) the leading, and the
following the following
[Prop.~7.20]. Thus, $E$
measures $A$, as the leading (measuring the) leading. But, in fact, 
($E$) also does not measure ($A$). The very thing (is) impossible. Thus,
$E$ and $A$ are not prime to one another. Thus, (they are) composite (to one another). And (numbers) composite (to one another) are (both) measured
by some [prime] number [Def.~7.14]. And since $E$
is assumed (to be) prime, and a prime (number) is not measured by  another number (other) than itself
[Def.~7.11], $E$ thus measures (both) $A$ and $E$. Hence, $E$ measures $A$. And it also measures $D$. Thus, $E$ measures (both) $A$ and $D$. So, similarly, we can show that however many prime
numbers $D$ is measured by, $A$ will also be measured by the same
(prime numbers). (Which is) the very thing it was required to show.

%%%%%%
% Prop 9.13
%%%%%%
\pdfbookmark[1]{Proposition 9.13}{pdf9.13}

\begin{center}
{\large Proposition 13}
\end{center}

If  any multitude whatsoever of numbers is continuously proportional, (starting) from a unit, and the (number) after the
unit is prime, then the greatest (number) will be measured by no [other] (numbers) except (numbers) existing  among the proportional numbers.

Let any multitude whatsoever of numbers,  $A$, $B$, $C$, $D$, be continuously proportional, (starting) from a unit. And let the (number)
after the unit, $A$, be prime. I say that the greatest of them, $D$, will be
measured by no other (numbers) except $A$, $B$, $C$. 

\epsfysize=1.1in
\centerline{\epsffile{Book09/fig13e.eps}}

For, if possible, let it be measured by $E$, and let $E$ not be
the same as one of $A$, $B$, $C$. So it is clear that $E$ is not prime. 
For if $E$ is prime, and measures $D$, then it will also measure $A$, (despite $A$) being prime (and) not being  the same as it [Prop.~9.12]. The very thing is impossible. Thus, $E$
is not prime. Thus, (it is) composite. And every composite
number is measured by some prime number [Prop.~7.31]. Thus, $E$ is measured by some prime number. So I say that it will be measured by no other prime number than $A$.
For if $E$ is measured by another (prime number), and $E$ measures $D$, then this (prime number) will thus also measure $D$. Hence, it will also
measure $A$,  (despite $A$) being prime (and) not being  the same as it [Prop.~9.12]. The very thing is impossible. Thus,
$A$ measures $E$. And since $E$ measures $D$, let it measure it according to $F$. I say that $F$ is not the same as one of $A$, $B$, $C$. For if $F$ is
the same as one of $A$, $B$, $C$, and measures $D$ according to $E$,
then one of $A$, $B$, $C$ thus also measures $D$ according to $E$.
But one of $A$, $B$, $C$ (only) measures $D$ according to some (one) of
$A$, $B$, $C$
[Prop.~9.11]. And thus $E$ is the same as one of
$A$, $B$, $C$. The very opposite thing was assumed. Thus, $F$ is not the
same as one of $A$, $B$, $C$. Similarly, we can show that $F$ is measured by $A$,  (by) again showing that $F$ is not prime. For if ($F$ is prime), and measures $D$,  then it will also measure $A$, (despite $A$) being prime (and) not being  the same as it [Prop.~9.12]. The very thing
is impossible. Thus, $F$ is not prime. Thus, (it is) composite.  And every composite
number is measured by some prime number [Prop.~7.31]. Thus, $F$ is measured by some prime
number. So I say that it will be measured by no other prime number than $A$.
For if  some other prime (number) measures $F$, and $F$ measures $D$, then this (prime number) will thus also measure $D$.  Hence, it will also
measure $A$,  (despite $A$) being prime (and) not being  the same as it [Prop.~9.12]. The very thing is impossible.
Thus, $A$ measures $F$. And since $E$ measures $D$ according to 
$F$, $E$ has thus made $D$ (by) multiplying $F$.  But, in fact, $A$ has
also made $D$ (by) multiplying $C$ [Prop.~9.11~corr.]. Thus, the (number created) from (multiplying)
$A$, $C$ is equal to  the (number created) from (multiplying) $E$, $F$. Thus, proportionally, as $A$ is to $E$, so $F$ (is) to $C$ [Prop.~7.19]. And $A$ measures $E$. Thus, $F$ also
measures $C$. Let it
 measure it according to $G$. So, similarly, we can show
that $G$ is not the same as one of $A$, $B$, and that it is measured by $A$. 
And since $F$ measures $C$ according to $G$, $F$ has thus made
$C$ (by) multiplying $G$.  But, in fact, $A$ has
also made $C$ (by) multiplying $B$ [Prop.~9.11~corr.]. Thus, the (number created) from (multiplying)
$A$, $B$ is equal to  the (number created) from (multiplying) $F$, $G$.
Thus, proportionally, as $A$ (is) to $F$, so $G$ (is) to $B$ [Prop.~7.19]. And $A$ measures $F$. 
Thus, $G$ also measures $B$.
Let it measure
it according to $H$. So, similarly, we can show that $H$ is not the same as $A$. And since $G$ measures $B$ according to $H$, $G$ has thus made $B$
(by) multiplying $H$.  But, in fact, $A$ has
 also made $B$ (by) multiplying itself [Prop.~9.8]. Thus, the (number created) from (multiplying)
$H$, $G$ is equal to  the square on $A$. Thus, as $H$ is to $A$, (so) $A$ (is) to $G$ [Prop.~7.19]. And $A$ measures $G$.
Thus, $H$ also measures $A$, (despite $A$) being prime (and) not being  the same as it. The very thing (is) absurd. Thus, the greatest (number) $D$ cannot be
measured by  another (number) except (one of) $A$, $B$, $C$. (Which is) the
very thing it was required to show.

%%%%%%
% Prop 9.14
%%%%%%
\pdfbookmark[1]{Proposition 9.14}{pdf9.14}

\begin{center}
{\large Proposition 14}
\end{center}

If  a least number is measured by (some) prime numbers then it will not be measured by
any other prime number except (one of) the original measuring (numbers).

\epsfysize=0.9in
\centerline{\epsffile{Book09/fig14e.eps}}

For let $A$ be the least number measured by the prime numbers $B$, $C$, $D$. I say that $A$ will not be measured by any other prime number except
(one of) $B$, $C$, $D$.

For, if possible, let it be measured by the prime (number) $E$. And
let $E$ not be the same as one of $B$, $C$, $D$. And since $E$ measures
$A$, let it measure it according to $F$.  Thus, $E$ has made $A$ (by)
multiplying $F$. And $A$ is measured by the prime numbers $B$, $C$, $D$.
And if two numbers make some (number by) multiplying one another,
and some prime number  measures the number created from them, then (the prime number)
will also measure one of the original (numbers) [Prop.~7.30]. Thus, $B$, $C$, $D$ will measure one of
$E$, $F$. In fact, they do not measure $E$. For $E$ is prime, and not the
same as one of $B$, $C$, $D$. Thus, they (all) measure $F$, which is less than $A$. The very thing (is) impossible. For $A$ was assumed
(to be) the least (number) measured by $B$, $C$, $D$. Thus, no prime number can measure $A$ except (one of) $B$, $C$, $D$. (Which is) the very thing it was required to show.

%%%%%%
% Prop 9.15
%%%%%%
\pdfbookmark[1]{Proposition 9.15}{pdf9.15}

\begin{center}
{\large Proposition 15}
\end{center}

If three continuously proportional
numbers are the least of those (numbers) having the same ratio as them then
two (of them) added together in any way are prime to the remaining (one).

\epsfysize=1.in
\centerline{\epsffile{Book09/fig15e.eps}}

Let $A$, $B$, $C$ be three continuously proportional numbers (which are)
the least of those (numbers) having the same ratio as them. I say that two
of $A$, $B$, $C$ added together in any way are prime to the remaining
(one), (that is) $A$ and $B$ (prime) to $C$, $B$ and $C$ to
$A$, and, further, $A$ and $C$ to $B$.

Let the two least numbers, $DE$ and $EF$, having the same ratio as
$A$, $B$, $C$, have been taken [Prop.~8.2]. So
it is clear that $DE$ has made $A$ (by) multiplying itself, and
has made $B$ (by) multiplying $EF$, and, further, $EF$ has made $C$ (by)
multiplying itself [Prop.~8.2]. And since $DE$, $EF$ are the least (of those numbers having the same ratio as them), 
they are prime to one another [Prop.~7.22]. 
And if two numbers are prime to one another then the sum (of them)
is also prime to each [Prop.~7.28]. Thus, $DF$
is also prime to each of $DE$, $EF$. But, in fact, $DE$ is  also prime to $EF$.
Thus, $DF$, $DE$ are (both) prime to $EF$. And if two numbers are (both)
prime to some number then the (number) created from (multiplying) them
is also prime to the remaining (number) [Prop.~7.24].
Hence, the (number created) from (multiplying) $FD$, $DE$ is prime
to $EF$. Hence, the (number created) from (multiplying) $FD$, $DE$ is also prime
to the (square) on $EF$ [Prop.~7.25]. 
[For if two numbers are prime to one another then the (number) created from
(squaring) one of them is prime to the remaining (number).]
But the (number created) from (multiplying) $FD$, $DE$ is the
(square) on $DE$ plus the (number created) from (multiplying) $DE$, $EF$
[Prop.~2.3]. Thus, the (square) on $DE$ plus
the (number created) from (multiplying) $DE$, $EF$ is prime to the
(square) on $EF$. 
And the (square)
on $DE$ is $A$, and the (number created) from (multiplying) $DE$, $EF$
(is) $B$, and the (square) on $EF$ (is) $C$. Thus,  $A$, $B$ summed is prime to $C$. So,  similarly, we can show that $B$, $C$ (summed) is also
prime to $A$. So I say that $A$, $C$ (summed) is also prime to $B$.
For since $DF$ is prime to each of $DE$, $EF$ then the (square) on
$DF$ is also prime to the (number created) from (multiplying) $DE$, $EF$
[Prop.~7.25]. But, the (sum of the squares) on $DE$, $EF$
plus twice the (number created) from (multiplying) $DE$, $EF$ is equal to
the (square) on $DF$ [Prop.~2.4]. And thus
the (sum of the squares) on $DE$, $EF$ plus twice  the (rectangle contained) by $DE$, $EF$ [is] prime to the (rectangle contained) by $DE$, $EF$.
By separation, the (sum of the squares) on $DE$, $EF$ plus  once the (rectangle contained) by $DE$, $EF$  is prime to the (rectangle contained)  by $DE$, $EF$.$^\dag$ Again, by separation,  the (sum of the squares) on $DE$, $EF$ is prime to the (rectangle contained)  by $DE$, $EF$. And the (square) on $DE$   is $A$,  and the (rectangle contained) by $DE$, $EF$ (is) $B$, and the (square) on $EF$ (is) $C$. Thus, $A$, $C$
summed is prime to $B$. (Which is) the very thing it was required to show.
{\footnotesize\noindent$^\dag$ Since if $\alpha\,\beta$ measures
$\alpha^2+\beta^2+2\,\alpha\,\beta$ then it also measures $\alpha^2+\beta^2+\alpha\,\beta$, and {\em vice versa}.}

%%%%%%
% Prop 9.16
%%%%%%
\pdfbookmark[1]{Proposition 9.16}{pdf9.16}

\begin{center}
{\large Proposition 16}
\end{center}

If two numbers are prime to one another then as the
first is to the second, so the second (will) not (be) to some other (number).

\epsfysize=0.9in
\centerline{\epsffile{Book09/fig16e.eps}}

For let the two numbers $A$ and $B$ be prime to one another. I say that
as $A$ is to $B$, so $B$ is not to some other (number).

For, if possible, let it be that as $A$ (is) to $B$, (so) $B$ (is) to $C$. And $A$ and $B$
(are) prime (to one another). And (numbers) prime (to one another are)
also the least (of those numbers having the same ratio as them) [Prop.~7.21]. And the least numbers measure those
(numbers) having the same ratio (as them) an equal number of times,
the leading (measuring) the leading, and the following the following [Prop.~7.20]. Thus, $A$ measures $B$, 
as the leading (measuring) the leading. And ($A$) also measures itself.
Thus, $A$ measures $A$ and $B$, which are prime to one another. The very
thing (is) absurd. Thus, as $A$  (is) to $B$, so $B$ cannot be to $C$.
(Which is) the very thing it was required to show.

%%%%%%
% Prop 9.17
%%%%%%
\pdfbookmark[1]{Proposition 9.17}{pdf9.17}

\begin{center}
{\large Proposition 17}
\end{center}

If   any multitude whatsoever of numbers is continuously proportional, and the outermost of them are prime to one another, then as the first (is) to the second, so the last will not be
to some other (number).

Let $A$, $B$, $C$, $D$ be any multitude whatsoever of continuously
proportional numbers. And let the outermost of them, $A$ and $D$,
be prime to one another. I say that as $A$ is to $B$, so $D$ (is) not to some
other (number).

\epsfysize=1.3in
\centerline{\epsffile{Book09/fig17e.eps}}

For, if possible, let it be that as $A$ (is) to $B$, so $D$ (is) to $E$. 
Thus, alternately, as $A$ is to $D$, (so) $B$ (is) to $E$ [Prop.~7.13]. And $A$ and $D$ are prime (to one another).  And (numbers) prime (to one another are)
also the least (of those numbers having the same ratio as them) [Prop.~7.21]. And the least numbers measure those
(numbers) having the same ratio (as them) an equal number of times,
the leading (measuring) the leading, and the following the following [Prop.~7.20]. Thus, $A$ measures $B$. And as $A$
is to $B$, (so) $B$ (is) to $C$.  Thus, $B$ also measures $C$.  And hence
$A$ measures $C$ [Def.~7.20]. And since as $B$ is to
$C$, (so) $C$ (is) to $D$, and $B$ measures $C$, $C$ thus also measures
$D$ [Def.~7.20].  But, $A$ was (found to be) measuring $C$. And 
hence $A$ also measures $D$. And ($A$) also measures itself.  Thus, $A$
measures $A$ and $D$, which are prime to one another. The very thing is impossible. Thus, as $A$ (is) to $B$, so $D$ cannot be to some other (number). (Which is) the very thing it was required to show.

%%%%%%
% Prop 9.18
%%%%%%
\pdfbookmark[1]{Proposition 9.18}{pdf9.18}

\begin{center}
{\large Proposition 18}
\end{center}

For  two given numbers, to investigate whether it
is possible to find a third (number) proportional to them.

\epsfysize=0.4in
\centerline{\epsffile{Book09/fig18e.eps}}

Let $A$ and $B$ be the two given numbers. And let it be required to investigate whether it is possible to find a third (number) proportional
to them.

So $A$ and $B$ are either prime to one another, or not. 
And if they are prime to one another then it has (already) been show that it is impossible
to find a third (number) proportional to them [Prop.~9.16].

And so let $A$ and $B$ not be prime to one another. And let $B$ make $C$
(by) multiplying itself. So $A$ either measures, or does not measure, $C$. 
Let it first of all measure ($C$) according to $D$. Thus, $A$ has made
$C$ (by) multiplying $D$. But, in fact, $B$ has also made
$C$ (by) multiplying itself. Thus, the (number created) from (multiplying)
$A$, $D$ is equal to the (square) on $B$. Thus, as $A$ is to $B$, (so)
$B$ (is) to $D$ [Prop.~7.19]. Thus, a third number
has been found proportional to $A$, $B$, (namely) $D$. 

And so let $A$ not measure $C$. I say that it is impossible to find a third
number proportional to $A$, $B$. For, if possible, let it have been found,
(and let it be) $D$. Thus, the (number created) from (multiplying) $A$, $D$
is equal to the (square) on $B$ [Prop.~7.19]. 
And the (square) on $B$ is $C$. Thus, the (number created) from (multiplying) $A$, $D$
is equal to $C$. Hence, $A$ has made $C$ (by) multiplying $D$. 
Thus, $A$ measures $C$ according to $D$. But ($A$) was, in fact, also assumed
(to be) not measuring ($C$). The very thing (is) absurd. Thus, it is not
possible to find a third number proportional to $A$, $B$ when $A$ does not
measure $C$. (Which is) the very thing it was required to show.

%%%%%%
% Prop 9.19
%%%%%%
\pdfbookmark[1]{Proposition 9.19}{pdf9.19}

\begin{center}
{\large Proposition 19}$^\dag$
\end{center}

For three given numbers, to investigate when it is
possible to find a fourth (number) proportional to them.

\epsfysize=1.3in
\centerline{\epsffile{Book09/fig19e.eps}}

Let $A$, $B$, $C$ be the three given numbers. And let it be required to
investigate when it is possible to find a fourth (number) proportional to them.

In fact, ($A$, $B$, $C$) are either not continuously proportional and the outermost
of them are prime to one another, or are continuously proportional
and the outermost of them are not prime to one another, or are neither
continuously proportional nor are the outermost of them prime to one another, or are continuously proportional and the outermost of them are
prime to one another.

In fact, if $A$, $B$, $C$ are continuously proportional, and the outermost
of them, $A$ and $C$, are prime to one another, (then) it has (already) been shown that it is impossible to find a fourth number proportional to them
[Prop.~9.17]. So let $A$, $B$, $C$
not be continuously proportional, (with) the outermost of them again being
prime to one another. I say that, in this case, it is also impossible to find a fourth (number)
proportional to them. For, if possible, let it have been found,
(and let it be) $D$. Hence, it will be that as $A$ (is) to $B$, (so)
$C$ (is) to $D$. And let it be contrived that as $B$ (is) to $C$, (so) $D$ (is) to $E$. And since as $A$ is to $B$, (so) $C$ (is) to $D$, and as $B$ (is) to $C$, (so)
$D$ (is) to $E$, thus, via equality, as $A$ (is) to $C$, (so) $C$ (is) to $E$
[Prop.~7.14]. And $A$ and $C$ (are) prime (to one another). And (numbers) prime (to one another are) also the least (numbers
having the same ratio as them) [Prop.~7.21]. 
And the least (numbers) measure those numbers having the same ratio
as them (the same number of times), the leading (measuring) the leading,
and the following the following [Prop.~7.20]. 
Thus, $A$ measures $C$, (as) the leading (measuring) the leading.
And it also measures itself. Thus, $A$ measures $A$ and $C$, which are
prime to one another. The very thing is impossible. Thus, it is not possible to find a fourth (number) proportional to $A$, $B$, $C$.

And so let $A$, $B$, $C$  again be continuously proportional, and let $A$ and $C$ not be prime to one another. I say that it is possible to find
a fourth (number) proportional to them. For let $B$ make $D$ (by) multiplying $C$. Thus, $A$ either measures or does not measure $D$. 
Let it, first of all, measure ($D$) according to $E$. Thus, $A$ has made $D$ (by) multiplying $E$. But, in fact, $B$ has also made $D$ (by) multiplying $C$. Thus, the (number created) from (multiplying) $A$, $E$
is equal to the (number created) from (multiplying) $B$, $C$. Thus, 
proportionally, as $A$ [is] to $B$, (so) $C$ (is) to $E$ [Prop.~7.19]. Thus, a fourth (number) proportional
to $A$, $B$, $C$ has been found, (namely) $E$.

And so let $A$ not measure $D$. I say that it is impossible to find a fourth
number proportional to $A$, $B$, $C$. For, if possible, let it have
been found, (and let it be) $E$. Thus, the (number created) from
(multiplying) $A$, $E$ is equal to the (number created) from (multiplying)
$B$, $C$. But, the (number created) from (multiplying) $B$, $C$ is $D$.
And thus the (number created) from (multiplying) $A$, $E$ is equal to $D$.
Thus, $A$ has made $D$ (by) multiplying $E$. Thus, $A$ measures
$D$ according to $E$. Hence, $A$ measures $D$. But, it also does not
measure ($D$). The very thing (is) absurd. Thus, it is not possible
to find a fourth number proportional to $A$, $B$, $C$ when $A$ does not
measure $D$. And so (let) $A$, $B$, $C$ (be) neither continuously
proportional, nor (let) the outermost of them (be) prime to one another.
And let $B$ make $D$ (by) multiplying $C$. So, similarly, it can be show that
if $A$ measures $D$ then it is possible to find a fourth (number)
proportional to ($A$, $B$, $C$), and impossible if ($A$) does not measure ($D$).
(Which is) the very thing it was required to show.
{\footnotesize\noindent$^\dag$ The proof of this proposition is incorrect.
There are, in fact, only two cases. Either $A$, $B$, $C$ are continuously
proportional,  with $A$ and $C$ prime to one another, or not. In the first case, 
it is impossible to find a fourth proportional number. In the second case,
it is possible to find a fourth proportional number provided that $A$
measures $B$ times $C$. Of the four cases considered by Euclid,
the proof given in the second case is incorrect, since it only demonstrates that
if $A:B::C:D$ then a number $E$ cannot be found such that $B:C::D:E$. The proofs given in the other three cases are correct.}

%%%%%%
% Prop 9.20
%%%%%%
\pdfbookmark[1]{Proposition 9.20}{pdf9.20}

\begin{center}
{\large Proposition 20}
\end{center}

The (set of all) prime numbers is more numerous than any assigned
multitude of prime numbers.

\epsfysize=1.3in
\centerline{\epsffile{Book09/fig20e.eps}}

Let $A$, $B$, $C$ be the assigned prime numbers. I say that the (set of all)
primes numbers is more numerous than $A$, $B$, $C$.

For let the least number measured by  $A$, $B$, $C$ have been taken, and let it be $DE$ [Prop.~7.36]. And let the unit $DF$ have been added to $DE$. So $EF$ is either prime, or not. Let it, first of all,
be prime. Thus, the (set of) prime numbers $A$, $B$, $C$, $EF$, (which is) more numerous than $A$, $B$, $C$, has been found.

And so let $EF$ not be prime. Thus, it is measured by some prime number [Prop.~7.31]. Let it be measured by the prime
(number) $G$. I say that $G$ is not the same as any of $A$, $B$, $C$. 
For, if possible, let it be (the same). And $A$, $B$, $C$ (all) measure
$DE$. Thus, $G$ will also measure $DE$. And it also measures $EF$. (So)
$G$ will also measure the remainder, unit $DF$, (despite) being a number
[Prop.~7.28]. The very thing (is) absurd. Thus, $G$
is not the same as one of $A$, $B$, $C$. And it was assumed (to be) prime.
Thus, the (set of) prime numbers $A$, $B$, $C$, $G$, (which is) more numerous than the assigned multitude (of prime numbers), $A$, $B$, $C$, has been found. (Which is) the very thing it
was required to show.

%%%%%%
% Prop 9.21
%%%%%%
\pdfbookmark[1]{Proposition 9.21}{pdf9.21}

\begin{center}
{\large Proposition 21}
\end{center}

If any multitude whatsoever of even numbers
is added together then the whole is even.

\epsfysize=0.3in
\centerline{\epsffile{Book09/fig21e.eps}}

For let any multitude  whatsoever of even numbers, $AB$, $BC$, $CD$, $DE$, lie together. I say that the whole, $AE$, is even.

For since everyone  of $AB$, $BC$, $CD$, $DE$ is even, it has a half part
[Def.~7.6]. And hence the whole $AE$ has
a half part. And an even number is one (which can be) divided in half [Def.~7.6]. Thus, $AE$ is even. (Which is) the very thing it was required to show.

%%%%%%
% Prop 9.22
%%%%%%
\pdfbookmark[1]{Proposition 9.22}{pdf9.22}

\begin{center}
{\large Proposition 22}
\end{center}

If any multitude whatsoever of odd numbers
is added together, and the multitude of them is even,  then the whole will be even.

\epsfysize=0.3in
\centerline{\epsffile{Book09/fig21e.eps}}

For let any even multitude whatsoever of odd numbers, 
$AB$, $BC$, $CD$, $DE$, lie together. I say that the whole, $AE$, is even.

For since everyone of $AB$, $BC$, $CD$, $DE$ is odd then,  a unit being subtracted from
each, everyone of the remainders will be (made) even [Def.~7.7]. And hence the sum of them will be even [Prop.~9.21]. And the multitude  of the units is even.
Thus, the whole $AE$ is also even [Prop.~9.21]. (Which is) the very thing it was required to show.

%%%%%%
% Prop 9.23
%%%%%%
\pdfbookmark[1]{Proposition 9.23}{pdf9.23}

\begin{center}
{\large Proposition 23}
\end{center}

If  any multitude whatsoever of odd numbers
is added together, and the multitude of them is odd,  then the whole will also be odd.

\epsfysize=0.3in
\centerline{\epsffile{Book09/fig23e.eps}}

For let any  multitude whatsoever of odd numbers, $AB$, $BC$, $CD$,  lie together, and let the multitude of them be odd. I say that the whole, $AD$, is also odd.
 
For let the unit $DE$ have been subtracted from $CD$. The remainder $CE$
is thus even [Def.~7.7]. And $CA$ is also even
[Prop.~9.22]. Thus, the whole $AE$ is also even
[Prop.~9.21]. And $DE$ is a unit. Thus,
$AD$ is odd [Def.~7.7]. (Which is) the very thing it
was required to show.

%%%%%%
% Prop 9.24
%%%%%%
\pdfbookmark[1]{Proposition 9.24}{pdf9.24}

\begin{center}
{\large Proposition 24}
\end{center}

If  an even (number) is subtracted from an(other) even
number then the remainder will be even.

\epsfysize=0.3in
\centerline{\epsffile{Book09/fig24e.eps}}

For let the even (number) $BC$ have been subtracted from the even number
$AB$. I say that the remainder $CA$ is even.

For since $AB$ is even, it has a half part [Def.~7.6]. 
So, for the same (reasons), $BC$ also has a half part. And hence the remainder [$CA$ has a half part]. [Thus,] $AC$ is even. (Which is) the very thing it was required to show.~\\

%%%%%%
% Prop 9.25
%%%%%%
\pdfbookmark[1]{Proposition 9.25}{pdf9.25}

\begin{center}
{\large Proposition 25}
\end{center}

If an odd (number) is subtracted from an even
number then the remainder will be odd.

\epsfysize=0.3in
\centerline{\epsffile{Book09/fig25e.eps}}

For let the odd (number) $BC$ have been subtracted from the even number
$AB$. I say that the remainder $CA$ is odd.

For let the unit $CD$ have been subtracted from $BC$. $DB$ is thus even
[Def.~7.7]. And $AB$ is also even. And thus the remainder $AD$ is even [Prop.~9.24]. And $CD$ is a unit. 
Thus, $CA$ is odd [Def.~7.7]. (Which is) the very thing it was required to show.

%%%%%%
% Prop 9.26
%%%%%%
\pdfbookmark[1]{Proposition 9.26}{pdf9.26}

\begin{center}
{\large Proposition 26}
\end{center}

If an odd (number) is subtracted from an odd
number then the remainder will be even.

\epsfysize=0.3in
\centerline{\epsffile{Book09/fig26e.eps}}

For let the odd (number) $BC$ have been subtracted from the odd
(number) $AB$. I say that the remainder $CA$ is even.

For since $AB$ is odd, let the unit $BD$ have been subtracted (from it). 
Thus, the remainder $AD$ is even [Def.~7.7]. So, for the same (reasons), 
$CD$ is also even. And hence the remainder $CA$ is even [Prop.~9.24]. (Which is) the very thing it was required to show.

%%%%%%
% Prop 9.27
%%%%%%
\pdfbookmark[1]{Proposition 9.27}{pdf9.27}

\begin{center}
{\large Proposition 27}
\end{center}

If an even (number) is subtracted from an odd
number then the remainder will be odd.

\epsfysize=0.3in
\centerline{\epsffile{Book09/fig27e.eps}}

For let the even (number) $BC$ have been subtracted from the odd (number)
$AB$. I say that the remainder $CA$ is odd.

$[$For$]$ let the unit $AD$ have been subtracted (from $AB$). $DB$ is thus
even [Def.~7.7]. And $BC$ is also even. Thus, the
remainder $CD$ is also even [Prop.~9.24]. 
$CA$ (is) thus odd [Def.~7.7]. (Which is) the
very thing it was required to show.

%%%%%%
% Prop 9.28
%%%%%%
\pdfbookmark[1]{Proposition 9.28}{pdf9.28}

\begin{center}
{\large Proposition 28}
\end{center}

If an odd number makes some (number by)
multiplying an even (number) then the created (number) will be even.

\epsfysize=0.8in
\centerline{\epsffile{Book09/fig28e.eps}}

For let the odd number $A$ make $C$ (by) multiplying the even (number)
$B$. I say that $C$ is even.

For since $A$ has made $C$ (by) multiplying $B$, $C$ is thus composed  out of so many (magnitudes) equal to $B$, as many as (there) are  units in $A$ [Def.~7.15]. And $B$ is even. Thus, $C$ is composed
out of even (numbers). And if any multitude whatsoever of even numbers
is added together then the whole is even [Prop.~9.21]. 
Thus, $C$ is even. (Which is) the very thing it was required to show.

%%%%%%
% Prop 9.29
%%%%%%
\pdfbookmark[1]{Proposition 9.29}{pdf9.29}

\begin{center}
{\large Proposition 29}
\end{center}

If  an odd number makes some (number by)
multiplying an odd (number) then the created (number) will be odd.

\epsfysize=0.8in
\centerline{\epsffile{Book09/fig28e.eps}}

For let the odd number $A$ make $C$ (by) multiplying the odd (number)
$B$. I say that $C$ is odd.

For since $A$ has made $C$ (by) multiplying $B$, $C$ is thus composed  out of so many (magnitudes) equal to $B$, as many as (there) are  units in $A$ [Def.~7.15]. And each of $A$, $B$ is odd. Thus, $C$ is composed
out of odd (numbers), (and) the multitude of them is odd.   
Hence $C$ is odd [Prop.~9.23]. (Which is) the very thing it was required to show.

%%%%%%
% Prop 9.30
%%%%%%
\pdfbookmark[1]{Proposition 9.30}{pdf9.30}

\begin{center}
{\large Proposition 30}
\end{center}

If an odd number measures an even number then it
will also measure (one) half of it.

\epsfysize=0.8in
\centerline{\epsffile{Book09/fig30e.eps}}

For let the odd number $A$ measure the even (number) $B$. I say that
($A$) will also measure (one) half of ($B$).

For since $A$ measures $B$, let it measure it according to $C$. I say that
$C$ is not odd. For, if possible, let it be (odd). And since $A$ measures $B$ according to $C$, $A$ has thus made $B$ (by) multiplying $C$. Thus, $B$
is composed out of odd numbers, (and) the multitude of them is odd. $B$
is thus odd [Prop.~9.23]. The very thing (is)
absurd. For ($B$) was assumed (to be) even. Thus, $C$ is not odd.
Thus, $C$ is even. Hence, $A$ measures $B$ an even number of times.
So, on account of this, ($A$) will also measure (one) half of ($B$).
(Which is) the very thing it was required to show.

%%%%%%
% Prop 9.31
%%%%%%
\pdfbookmark[1]{Proposition 9.31}{pdf9.31}

\begin{center}
{\large Proposition 31}
\end{center}

If an odd number is prime to some number then it
will also be prime to its double.

\epsfysize=1.1in
\centerline{\epsffile{Book09/fig31e.eps}}

For let the odd number $A$ be prime to some number $B$. And let $C$ be
double $B$. I say that $A$ is [also] prime to $C$.

For if [$A$ and $C$] are not prime (to one another) then some number
will measure them. Let it measure (them), and let it be $D$. And
$A$ is odd. Thus, $D$ (is) also odd. And since $D$, which is odd, measures
$C$, and $C$ is even, [$D$] will thus also measure half of $C$ [Prop.~9.30]. 
And $B$ is half of $C$.  Thus, $D$ measures $B$. And it also measures
$A$. Thus, $D$ measures (both) $A$ and $B$, (despite) them being prime to one another. The very thing is impossible. Thus, $A$ is not unprime 
to $C$. Thus, $A$ and $C$ are prime to one another. (Which is) the very thing it
was required to show.

%%%%%%
% Prop 9.32
%%%%%%
\pdfbookmark[1]{Proposition 9.32}{pdf9.32}

\begin{center}
{\large Proposition 32}
\end{center}

Each of the numbers (which is continually) doubled,
(starting) from a dyad, is an even-times-even (number) only.

\epsfysize=1.1in
\centerline{\epsffile{Book09/fig32e.eps}}

For let any multitude of numbers whatsoever, $B$, $C$, $D$, have been
(continually) doubled, (starting) from the dyad $A$. I say that
$B$, $C$, $D$ are  even-times-even (numbers) only.

In fact, (it is) clear that each [of $B$, $C$, $D$] is an even-times-even (number). For it is doubled from a dyad [Def.~7.8]. 
I also say that (they are even-times-even numbers) only. For let a unit
be laid down. Therefore, since any multitude of numbers whatsoever
are continuously proportional, starting from a unit, and the (number) $A$ after the unit is prime, the greatest of $A$, $B$, $C$, $D$, (namely) $D$,
will not be measured by any other (numbers) except $A$, $B$, $C$
[Prop.~9.13].
And each of $A$, $B$, $C$ is even. Thus, $D$ is an
even-time-even (number) only [Def.~7.8]. So, similarly,
we can show that each of $B$, $C$ is [also]  an even-time-even (number) only. (Which is) the very thing it was required to show.

%%%%%%
% Prop 9.33
%%%%%%
\pdfbookmark[1]{Proposition 9.33}{pdf9.33}

\begin{center}
{\large Proposition 33}
\end{center}

If a number has an odd half then it is an
even-time-odd (number) only.

\epsfysize=0.175in
\centerline{\epsffile{Book09/fig33e.eps}}

For let the number $A$ have an odd half. I say that $A$ is  an
even-times-odd (number) only.

In fact, (it is) clear that ($A$) is an even-times-odd (number). For its
half, being odd, measures it an even number of times [Def.~7.9]. So I also say that (it is an even-times-odd number) only. For if $A$ is also  an even-times-even
(number) then it will be measured by an even (number) according to
an even number [Def.~7.8]. Hence, its half
will also be measured by an even number, (despite)  being odd. The very
thing is absurd. Thus, $A$ is  an even-times-odd (number) only. (Which is)
the very thing it was required to show.

%%%%%%
% Prop 9.34
%%%%%%
\pdfbookmark[1]{Proposition 9.34}{pdf9.34}

\begin{center}
{\large Proposition 34}
\end{center}

If a number is neither (one) of the (numbers) doubled from a dyad, nor
has an odd half, then it is (both) an even-times-even and an even-times-odd
(number).

\epsfysize=0.175in
\centerline{\epsffile{Book09/fig33e.eps}}

For let the number $A$   neither be (one) of the (numbers) doubled from a
dyad, nor let it have an odd half. I say that $A$ is (both) an even-times-even
and an even-times-odd (number).

In fact, (it is) clear that $A$ is an even-times-even (number) [Def.~7.8]. For it does
not have an odd half. So I say that it is also an even-times-odd (number). 
For if we cut $A$ in half, and (then cut) its half in half, and we do this continually, then we will arrive at some odd number which will measure
$A$ according to an even number. For if not, we will arrive at a dyad,
and $A$ will be (one) of the (numbers) doubled from a dyad. The very opposite thing (was) assumed. Hence, $A$ is an even-times-odd (number)
[Def.~7.9]. And it was also shown (to be) an
even-times-even (number). Thus, $A$ is (both) an even-times-even and
an even-times-odd (number). (Which is) the very thing it was required to
show.

%%%%%%
% Prop 9.35
%%%%%%
\pdfbookmark[1]{Proposition 9.35}{pdf9.35}

\begin{center}
{\large Proposition 35}$^\dag$
\end{center}

If there is any multitude whatsoever of continually
proportional numbers, and (numbers) equal to the first are subtracted from (both) the
second and the last, then as the excess of the second (number is) to the first, so
the excess of the last will be to (the sum of) all   those (numbers) before it.

\epsfysize=1.6in
\centerline{\epsffile{Book09/fig35e.eps}}

Let $A$, $BC$, $D$, $EF$ be any multitude whatsoever of continuously proportional numbers, beginning from the least $A$.  And let $BG$ and
$FH$, each equal to $A$, have been subtracted from $BC$ and $EF$ (respectively). I say that as $GC$ is to $A$, so $EH$ is to $A$, $BC$, $D$.

For let $FK$ be made equal to $BC$, and $FL$ to  $D$. And since $FK$
is equal to $BC$, of which $FH$ is equal to $BG$, the remainder
$HK$ is thus equal to the remainder $GC$. And since as $EF$ is to $D$, so
$D$ (is) to $BC$, and $BC$ to $A$ [Prop.~7.13],
and $D$ (is) equal to $FL$, and $BC$ to $FK$, and $A$ to  $FH$, thus
as $EF$ is to $FL$, so $LF$ (is) to $FK$, and $FK$ to $FH$. By
separation, as $EL$ (is) to $LF$, so $LK$ (is) to $FK$, and $KH$ to $FH$
[Props.~7.11, 7.13]. And thus
as one of the leading (numbers) is to one of the following, so (the sum of) all of the leading (numbers is) to (the sum of) all of the following [Prop.~7.12]. Thus, as $KH$ is to $FH$, so
$EL$, $LK$, $KH$ (are) to $LF$, $FK$, $HF$.  And $KH$ (is) equal to
$CG$, and $FH$ to $A$, and $LF$, $FK$, $HF$ to $D$, $BC$, $A$.
Thus, as $CG$ is to $A$, so $EH$ (is) to $D$, $BC$, $A$. Thus,
as the excess of the second (number) is to the first, so the excess of the last
(is) to (the sum of) all  those (numbers) before it. (Which is) the very thing it was required to show.
{\footnotesize\noindent$^\dag$ This proposition allows us to sum a geometric series of the form $a$, $a\,r$, $a\,r^2$, $a\,r^3,\cdots a\,r^{n-1}$. 
According to Euclid, the sum $S_n$  satisfies
$(a\,r-a)/a = (a\,r^n-a)/S_n$. Hence, $S_n= a\,(r^n-1)/(r-1)$.}

%%%%%%
% Prop 9.36
%%%%%%
\pdfbookmark[1]{Proposition 9.36}{pdf9.36}

\begin{center}
{\large Proposition 36}$^\dag$
\end{center}

If any multitude whatsoever of numbers is set out
continuously 
in a double  proportion, (starting) from a  unit,
until the whole sum added together becomes  prime, and the sum multiplied
into the last (number) makes some (number), then the (number so) created
will be perfect.

For let any multitude of numbers, $A$, $B$, $C$, $D$, be set out (continuouly) in a
double proportion, until the whole sum added together is made  prime. And let $E$ be equal to the sum. And let $E$ make $FG$ (by) multiplying $D$. I say that $FG$ is a perfect (number).

\epsfysize=1.3in
\centerline{\epsffile{Book09/fig36e.eps}}
~\\


\epsfysize=1.4in
\centerline{\epsffile{Book09/fig36ae.eps}}

For as many as is the multitude of $A$, $B$, $C$, $D$, let so many (numbers), $E$, $HK$, $L$, $M$, have been taken in a double
proportion, (starting) from $E$. Thus, via equality, as $A$ is to $D$, so
$E$ (is) to $M$ [Prop.~7.14].
Thus, the (number created) from (multiplying) $E, D$ is equal to the (number
created) from (multiplying) $A$, $M$.
 And $FG$ is the
(number created) from (multiplying) $E$, $D$. Thus, $FG$ is also
the (number created) from (multiplying) $A$, $M$ [Prop.~7.19]. Thus, $A$ has made $FG$ (by) multiplying $M$. Thus, $M$ measures $FG$ according to the units in $A$.
And $A$ is a dyad. Thus, $FG$ is double $M$.  And $M$, $L$, $HK$,
$E$ are also continuously double one another. Thus, $E$, $HK$, $L$, $M$,
$FG$ are continuously proportional in a double proportion. So
let $HN$ and $FO$, each equal to the first (number) $E$, have been subtracted from the second (number) $HK$ and the last $FG$ (respectively). 
Thus, as the excess of the second number is to the first, so the excess
of the last (is) to (the sum of) all  those (numbers) before it [Prop.~9.35]. Thus, as $NK$ is to $E$, so
$OG$ (is) to $M$, $L$, $KH$, $E$. And $NK$ is equal to $E$. And thus
$OG$ is equal to $M$, $L$, $HK$, $E$. And $FO$ is also equal to $E$,
and $E$ to $A$, $B$, $C$, $D$, and a unit. Thus, the whole of $FG$ is equal to
$E$, $HK$, $L$, $M$, and $A$, $B$, $C$, $D$, and a unit. And it is
measured by them. I also say that $FG$ will be measured by no other (numbers)
except $A$, $B$, $C$, $D$, $E$, $HK$, $L$, $M$, and a unit. For, if possible, let some (number) $P$ measure $FG$, and let $P$ not be the same
as any of $A$, $B$, $C$, $D$, $E$, $HK$, $L$, $M$. And as many times
as $P$ measures $FG$, so many units let there be in $Q$. Thus, $Q$ has
made $FG$ (by) multiplying $P$. But, in fact, $E$ has  also made $FG$
(by) multiplying $D$. Thus, as $E$ is to $Q$, so $P$ (is) to $D$
[Prop.~7.19]. And since $A$, $B$, $C$, $D$
are continually proportional, (starting) from a unit, $D$ will thus not be measured
by any other numbers except $A$, $B$, $C$ [Prop.~9.13]. And $P$ was assumed not (to be) the same as any of $A$, $B$, $C$. Thus, $P$ does not measure $D$. But, as $P$ (is) to
$D$, so $E$ (is) to $Q$. Thus, $E$ does not measure $Q$ either
 [Def.~7.20]. And $E$ is a prime (number).
And every prime number [is] prime to every  (number) which it does not
measure [Prop.~7.29]. Thus, $E$ and $Q$ are
prime to one another. And  (numbers) prime (to one another are) also the least (of those
numbers having the same ratio as them) [Prop.~7.21],
and the least (numbers) measure those (numbers) having the same ratio as them an equal number of times, the leading (measuring) the leading, and the
following the following [Prop.~7.20].
And as $E$ is to $Q$, (so) $P$ (is) to $D$. Thus, $E$ measures $P$ the same number of times as $Q$ (measures) $D$. And $D$ is not measured by any
other (numbers) except $A$, $B$, $C$.  Thus, $Q$ is the same as one
of $A$, $B$, $C$. Let it be the same as $B$. And as many as is the
multitude of $B$, $C$, $D$, let so many (of the set out numbers) have been taken, (starting) from $E$, 
(namely) $E$, $HK$, $L$. And $E$, $HK$, $L$ are in the same ratio as
$B$, $C$, $D$. Thus, via equality, as $B$ (is) to $D$, (so) $E$ (is) to $L$
[Prop.~7.14]. Thus, the (number created) from
(multiplying) $B$, $L$ is equal to the (number created) from multiplying
$D$, $E$ [Prop.~7.19]. But, the (number created)
from (multiplying) $D$, $E$ is equal to the (number created) from (multiplying)
$Q$, $P$. Thus, the (number created) from (multiplying) $Q$, $P$
is equal to the (number created) from (multiplying) $B$, $L$. Thus, as
$Q$ is to $B$, (so) $L$ (is) to $P$ [Prop.~7.19]. 
And $Q$ is the same as $B$. Thus, $L$ is also the same as $P$. The
very thing (is) impossible. For $P$ was assumed not (to be) the same
as any of the (numbers) set out. Thus, $FG$ cannot be measured by any number except $A$, $B$, $C$, $D$, $E$, $HK$, $L$, $M$, and a unit.
And $FG$ was shown (to be) equal to (the sum of) $A$, $B$, $C$, $D$, $E$, $HK$,
$L$, $M$, and a unit. And a perfect number is one which is equal to (the sum of) its own
parts [Def.~7.22]. Thus, $FG$ is a perfect (number).
(Which is) the very thing it was required to show.
{\footnotesize\noindent$^\dag$ This proposition demonstrates that perfect
numbers take the form $2^{n-1}\,(2^n-1)$ provided that $2^n-1$ is a prime
number. The ancient Greeks knew of four perfect numbers: 6, 28, 496, and
8128, which correspond to $n= 2$, 3, 5, and 7, respectively.}

\newpage
\thispagestyle{plain}
~\\