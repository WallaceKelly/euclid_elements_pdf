%%%%%%
% BOOK 12
%%%%%%
\pdfbookmark[0]{Book 12}{book12}
\pagestyle{plain}
\begin{center}
{\Huge ELEMENTS BOOK 12}\\
\spa\spa\spa
{\huge\it Proportional Stereometry\symbolfootnote[2]{The novel feature of this book is
the use of the so-called {\em method of exhaustion}\/ (see Prop.~10.1), a precursor to integration which is generally attributed to Eudoxus of Cnidus.}}
\end{center}
\newpage

%%%%
%12.1
%%%%
\pdfbookmark[1]{Proposition 12.1}{pdf12.1}
\pagestyle{fancy}
\cfoot{\gr{\thepage}}
\lhead{\large\gr{STOIQEIWN \ggn{12}.}}
\rhead{\large ELEMENTS BOOK 12}

\begin{center}
{\large Proposition 1}
\end{center}

Similar polygons (inscribed) in circles are to one another
as the squares on the diameters (of the circles).

\epsfysize=1.5in
\centerline{\epsffile{Book12/fig01e.eps}}

Let $ABC$ and $FGH$ be circles, and let $ABCDE$ and $FGHKL$ be
similar polygons (inscribed) in them (respectively), and let $BM$ and
$GN$ be the diameters of the circles (respectively). I say that as the square on
$BM$ is to the square on $GN$, so polygon $ABCDE$ (is) to polygon
$FGHKL$.

For let $BE$, $AM$, $GL$, and $FN$ have been joined. And since polygon
$ABCDE$ (is) similar to polygon $FGHKL$, angle $BAE$ is also equal to
(angle) $GFL$, and as $BA$ is to $AE$, so $GF$ (is) to  $FL$ [Def.~6.1]. So, $BAE$ and $GFL$ are two triangles having one angle
equal to one angle, (namely), $BAE$ (equal) to $GFL$, and the sides
around the equal angles proportional. Triangle $ABE$ is thus equiangular with
triangle $FGL$ [Prop.~6.6]. Thus, angle $AEB$ is equal to
(angle) $FLG$. But, $AEB$ is equal to $AMB$, and $FLG$ to $FNG$, for they stand on the same
circumference [Prop.~3.27].
Thus, $AMB$ is also equal to $FNG$. And the right-angle $BAM$ is also
equal to the right-angle $GFN$ [Prop.~3.31]. 
Thus, the remaining (angle) is also equal to the remaining (angle) [Prop.~1.32]. Thus,
triangle $ABM$ is equiangular with triangle $FGN$. Thus, proportionally, as $BM$
is to $GN$, so $BA$ (is) to $GF$ [Prop.~6.4]. But, 
the (ratio) of the square on $BM$ to the square
on $GN$ is the square of the
ratio of $BM$ to $GN$, and  the (ratio) of
polygon $ABCDE$ to polygon $FGHKL$ is the square of the (ratio) of $BA$ to $GF$ [Prop.~6.20].  
And, thus, as the square on $BM$ (is) to the square on $GN$, so polygon $ABCDE$
(is) to polygon $FGHKL$. 

Thus, similar polygons (inscribed) in circles are to one another
as the squares on the diameters (of the circles). (Which is) the very thing it was required to
show.

%%%%
%12.2
%%%%
\pdfbookmark[1]{Proposition 12.2}{pdf12.2}

\begin{center}
{\large Proposition 2}
\end{center}

Circles are to one another as the squares on (their) diameters.

Let $ABCD$ and $EFGH$ be circles, and [let] $BD$ and $FH$ [be] their diameters.
I say that as circle $ABCD$ is to circle $EFGH$, so the square on $BD$ (is)
to the square on $FH$.

\epsfysize=3in
\centerline{\epsffile{Book12/fig02e.eps}}

For if the circle $ABCD$ is not to the (circle) $EFGH$, as the square on
$BD$ (is) to the (square) on $FH$, then  as the (square) on $BD$ (is) to
the (square) on $FH$, so circle $ABCD$ will be to some area either less than, or greater
than, circle $EFGH$. Let it, first of all, be (in that ratio) to (some) lesser (area), $S$.
And let the square $EFGH$ have been inscribed in circle $EFGH$ [Prop.~4.6]. So the inscribed
square is greater than half of circle $EFGH$, inasmuch as if we draw tangents to the circle through the points $E$, $F$, $G$, and $H$,
then square $EFGH$ is half of the square circumscribed about the circle [Prop.~1.47], and the circle is less than the
circumscribed square. Hence, the inscribed square $EFGH$ is greater than
half of circle $EFGH$. Let the circumferences $EF$, $FG$, $GH$,
and $HE$ have been cut in half at points $K$, $L$, $M$, and $N$ (respectively), and let $EK$, $KF$, $FL$, $LG$, $GM$, $MH$, $HN$,
and $NE$ have been joined. And, thus, each of the triangles
$EKF$, $FLG$, $GMH$, and $HNE$ is greater than half of the
segment of the circle about it, inasmuch as if we draw tangents to the circle through points $K$, $L$, $M$, and $N$, and complete the parallelograms on the straight-lines $EF$, $FG$, $GH$, and $HE$, then
each of the triangles $EKF$, $FLG$, $GMH$, and $HNE$ will be
half of the parallelogram about it, but the segment about
it is less than the parallelogram. Hence, each of the triangles $EKF$, 
$FLG$, $GMH$, and $HNE$ is greater than half of the segment of
the circle about it. So, by cutting the  circumferences remaining behind in half,
and joining straight-lines, and doing this continually, we will
(eventually) leave behind some segments of the circle whose (sum) will be
less than the excess by which circle $EFGH$ exceeds the area $S$.
For we showed in the first theorem of the tenth book that if two unequal
magnitudes are laid out, and if (a part) greater than a half is subtracted
from the greater, and (if from) the remainder (a part) greater than a
half (is subtracted), and this happens continually, then some magnitude
will (eventually) be left which will be less than the lesser laid out magnitude [Prop.~10.1]. Therefore, let the (segments) have been left, and let 
 the (sum of the) segments of the circle $EFGH$ on $EK$, $KF$, $FL$, $LG$, $GM$, $MH$, $HN$, and $NE$ be less than the excess by which  circle
 $EFGH$ exceeds  area $S$. Thus, the remaining polygon $EKFLGMHN$ is greater than area $S$. 
 And let the polygon $AOBPCQDR$,
 similar to the polygon $EKFLGMHN$, have been inscribed in circle
 $ABCD$. Thus, as the square on $BD$ is to the square on $FH$, so
polygon $AOBPCQDR$ (is) to polygon $EKFLGMHN$ [Prop.~12.1]. But, also, as the square on $BD$ (is) to the square on $FH$, so circle $ABCD$ (is) to area $S$. And, thus, as
circle $ABCD$ (is) to area $S$, so polygon $AOBPGQDR$ (is) to
polygon $EKFLGMHN$ [Prop.~5.11]. Thus, alternately, as circle $ABCD$
(is) to the polygon (inscribed) within it, so area $S$ (is) to polygon $EKFLGMHN$
[Prop.~5.16]. And circle $ABCD$ (is)
greater than the polygon (inscribed) within it. Thus, area $S$ is also greater than polygon 
$EKFLGMHN$. But, (it is) also less. The very thing
is impossible. Thus, the square on $BD$ is not to the (square) on $FH$, as
circle $ABCD$ (is) to some area less than circle $EFGH$. So, similarly,
we can show that the (square) on $FH$ (is) not to the (square) on 
$BD$ as circle $EFGH$ (is) to some area less than circle $ABCD$ either.

So, I say that neither (is) the (square) on $BD$ to the (square) on $FH$,
as circle $ABCD$ (is) to some area greater than circle $EFGH$.

For, if possible, let it be (in that ratio) to (some)  greater (area), $S$. Thus, inversely, 
as the square on $FH$ [is] to the (square) on $DB$, so area $S$ (is) to circle $ABCD$ [Prop.~5.7~corr.]. But, as area $S$ (is) to circle $ABCD$, so circle $EFGH$
(is) to some area less than circle $ABCD$ (see lemma). And, thus, as the (square) on $FH$ (is) to the (square) on $BD$, so circle $EFGH$ (is) to some area
less than  circle $ABCD$ [Prop.~5.11]. The very thing was shown (to be) impossible.
Thus, as the square on $BD$ is to the (square) on $FH$, so circle $ABCD$
(is) not to some area greater than circle $EFGH$. And it was shown that neither (is it in that ratio) to (some) lesser (area). Thus, as the square on $BD$ is to the (square) on $FH$, so circle $ABCD$ (is) to circle $EFGH$.

Thus, circles are to one another as the squares on (their) diameters. (Which is) the very thing it was required to show.


\begin{center}
\large{Lemma}
\end{center}

So, I say that, area $S$ being greater than circle $EFGH$, as area $S$ is to
circle $ABCD$, so circle $EFGH$ (is) to some area less
than circle $ABCD$.

For let it have been contrived that as area $S$ (is) to circle $ABCD$,
so circle $EFGH$ (is) to area $T$. I say that area $T$ is less than circle
$ABCD$. For since as area $S$ is to circle $ABCD$, so circle $EFGH$
(is) to area $T$, alternately, as area $S$ is to circle $EFGH$, so circle
$ABCD$ (is) to area $T$ [Prop.~5.16]. And area $S$ (is) greater than circle $EFGH$.
Thus, circle $ABCD$ (is) also greater than area $T$ [Prop.~5.14]. Hence,
as area $S$ is to circle $ABCD$, so circle $EFGH$ (is) to some area less
than circle $ABCD$. (Which is) the very thing it was required to show.

%%%%
%12.3
%%%%
\pdfbookmark[1]{Proposition 12.3}{pdf12.3}

\begin{center}
{\large Proposition 3}
\end{center}

Any pyramid having a triangular base is divided into two
pyramids having triangular bases (which are) equal, similar to one another, and [similar] to the whole, and into two equal prisms. And the (sum of the) two prisms is greater than half of the
whole pyramid.

\epsfysize=2.2in
\centerline{\epsffile{Book12/fig03e.eps}}

Let there be a pyramid whose base is triangle $ABC$, and (whose) apex (is) point $D$. I say that
pyramid $ABCD$ is divided into two pyramids having triangular bases (which are) equal
to one another, and similar to the whole, and into two equal prisms. And the (sum of the)
two prisms is greater than half of the whole pyramid.

For let $AB$, $BC$, $CA$, $AD$, $DB$, and $DC$ have been cut in half at points $E$, $F$, $G$,
$H$, $K$, and $L$ (respectively). And let $HE$, $EG$, $GH$, $HK$, $KL$, $LH$, $KF$, and $FG$
have been joined. Since $AE$ is equal to $EB$, and $AH$ to $DH$, $EH$ is thus parallel to
$DB$ [Prop.~6.2]. So, for the same (reasons), $HK$ is also parallel
to $AB$. Thus, $HEBK$ is a parallelogram. Thus, $HK$ is equal to $EB$ [Prop.~1.34]. But, $EB$ is equal to $EA$. Thus, $AE$ is also equal to $HK$. And $AH$
is also equal to $HD$. So the two (straight-lines) $EA$ and $AH$ are equal to the two
(straight-lines) $KH$ and $HD$, respectively. And angle $EAH$ (is) equal to angle
$KHD$ [Prop.~1.29]. Thus, base $EH$ is equal to base $KD$ [Prop.~1.4]. Thus, triangle $AEH$ is equal and similar to triangle $HKD$
[Prop.~1.4]. So, for the same (reasons), triangle $AHG$
is also equal and similar to triangle $HLD$. And since $EH$ and $HG$ are two
straight-lines joining one another (which are respectively) parallel to two straight-lines
joining one another, $KD$ and $DL$, not being in the same plane, they will contain
equal angles [Prop.~11.10]. Thus, angle $EHG$ is equal to angle
$KDL$. And since the two straight-lines $EH$ and $HG$ are equal to the two straight-lines
$KD$ and $DL$, respectively, and angle $EHG$ is equal to angle $KDL$, base $EG$
[is] thus equal to base $KL$ [Prop.~1.4]. Thus, triangle $EHG$ is equal
and similar to triangle $KDL$. So, for the same (reasons), triangle $AEG$ is also equal and
similar to triangle $HKL$. Thus, the pyramid whose base is triangle $AEG$,
and apex the point $H$, is equal and similar to the pyramid whose base is triangle $HKL$,
and apex the point $D$ [Def.~11.10]. And since $HK$ has been drawn
parallel to one of the sides, $AB$, of triangle $ADB$, triangle $ADB$
is equiangular to triangle $DHK$ [Prop.~1.29], and they have proportional sides. Thus, triangle $ADB$ is similar to triangle $DHK$ [Def.~6.1].
So, for the same (reasons), triangle $DBC$ is also similar to triangle $DKL$, and
$ADC$ to $DLH$. And since two straight-lines joining one another, $BA$ and $AC$, are parallel
to two straight-lines joining one another, $KH$ and $HL$, not in the same plane, they will
contain equal angles [Prop.~11.10]. Thus, angle $BAC$ is equal
to (angle) $KHL$. And as $BA$ is to $AC$, so $KH$ (is) to $HL$. Thus, triangle $ABC$ is
similar to triangle $HKL$ [Prop.~6.6]. And, thus, the pyramid whose base is triangle $ABC$, and apex
the point $D$, is similar to the pyramid whose base is triangle $HKL$, and apex the point $D$ [Def.~11.9]. But,
the pyramid whose base [is] triangle $HKL$, and apex the point $D$, was shown (to be) similar
to the pyramid whose base is triangle $AEG$, and apex the point $H$. Thus, each of the pyramids
$AEGH$ and $HKLD$ is similar to the whole pyramid $ABCD$.

And since $BF$ is equal to $FC$, parallelogram $EBFG$ is double triangle $GFC$ [Prop.~1.41]. And since, if two prisms (have) 
equal heights, and the former has a parallelogram as a base, and the latter a triangle, and the parallelogram (is) double
the triangle, then the prisms are equal [Prop.~11.39],  the prism
contained by the two triangles $BKF$ and $EHG$, and the three parallelograms $EBFG$,
$EBKH$, and $HKFG$, is thus  equal to the prism contained by the two triangles $GFC$ and
$HKL$, and the three parallelograms $KFCL$, $LCGH$, and $HKFG$. And (it is) clear that
each of the prisms whose base (is) parallelogram $EBFG$, and opposite (side) straight-line
$HK$, and whose base (is) triangle $GFC$, and opposite (plane) triangle $HKL$, is greater than
each of the pyramids whose bases are triangles $AEG$ and $HKL$, and apexes the points
$H$ and $D$ (respectively), inasmuch as, if we [also] join the straight-lines
$EF$ and $EK$ then the prism whose base (is) parallelogram $EBFG$, and opposite
(side) straight-line $HK$, is greater than the pyramid whose base (is) triangle $EBF$,
and apex the point $K$. But the pyramid whose base (is) triangle $EBF$, and apex the point $K$,
is equal to the pyramid whose base is triangle $AEG$, and apex point $H$. For
they are contained by equal and similar planes. And, hence, the prism whose base (is)
parallelogram $EBFG$, and opposite (side) straight-line $HK$, is greater than the pyramid
whose base (is) triangle $AEG$, and apex the  point $H$. And the prism 
whose base is
parallelogram $EBFG$, and opposite (side) straight-line $HK$, (is) equal to the prism
whose base (is) triangle $GFC$, and opposite (plane) triangle $HKL$. And the pyramid whose
base (is) triangle $AEG$, and apex the  point $H$, is equal to the pyramid whose base (is) triangle
$HKL$, and apex the point $D$. Thus, the (sum of the) aforementioned
two prisms is greater than the (sum of the) aforementioned two pyramids, whose bases (are)
triangles $AEG$ and $HKL$, and apexes the points $H$ and $D$ (respectively).

Thus, the whole pyramid, whose base (is) triangle $ABC$, and apex the point $D$, has been
divided into two pyramids (which are) equal to one another [and similar to the whole],
and into two equal prisms. And the (sum of the) two prisms is greater than half of the
whole pyramid. (Which is) the very thing it was required to show.

%%%%
%12.4
%%%%
\pdfbookmark[1]{Proposition 12.4}{pdf12.4}

\begin{center}
{\large Proposition 4}
\end{center}

If there are two pyramids with the same height, having trianglular
bases, and each of them is divided into two pyramids equal to one another, and
similar to the whole, and into two equal prisms then as the base of one pyramid
(is) to the base of the other pyramid, so  (the sum of) all the prisms in one pyramid
will be to (the sum of all) the equal  number of prisms in the other pyramid.

Let there be two pyramids with the same height, having the triangular
bases $ABC$ and $DEF$, (with)
apexes the points $G$ and $H$ (respectively). And let each of them have been divided into
two pyramids equal to one another, and similar to the whole, and into two
equal prisms [Prop.~12.3]. I say that as base $ABC$ is to base
$DEF$, so (the sum of) all the prisms in pyramid $ABCG$ (is) to (the sum of) all the equal number of prisms in
pyramid $DEFH$.

\epsfysize=2.in
\centerline{\epsffile{Book12/fig04e.eps}}

For since $BO$ is equal to $OC$, and $AL$ to $LC$,
$LO$ is thus parallel to $AB$, and triangle $ABC$ similar to triangle $LOC$
[Prop.~12.3]. So, for the same (reasons), triangle $DEF$ is also similar
to triangle $RVF$. And since $BC$ is double $CO$, and $EF$ (double) $FV$, thus
as $BC$ (is) to $CO$, so $EF$ (is) to $FV$. And the similar, and similarly laid out, rectilinear
(figures) $ABC$ and $LOC$ have been described on $BC$ and $CO$ (respectively),
and the similar, and similarly laid out, [rectilinear] (figures) $DEF$ and $RVF$ on
$EF$ and $FV$ (respectively). Thus, as triangle $ABC$ is to triangle $LOC$, so triangle
$DEF$ (is) to triangle $RVF$ [Prop.~6.22]. Thus, alternately, as
triangle $ABC$ is to [triangle] $DEF$, so [triangle] $LOC$ (is) to triangle $RVF$ [Prop.~5.16]. But, as triangle $LOC$ (is) to triangle $RVF$, so the prism whose
base [is] triangle $LOC$, and opposite (plane) $PMN$, (is) to the prism whose base (is) triangle
$RVF$, and opposite (plane) $STU$ (see lemma). And, thus, as triangle $ABC$ (is) to triangle $DEF$,  so the prism whose base (is) triangle $LOC$, and opposite (plane) $PMN$, (is) to the
prism whose base (is) triangle $RVF$, and opposite (plane) $STU$. And as the aforementioned
prisms (are) to one another, so the prism whose base (is) parallelogram $KBOL$, and opposite
(side) straight-line $PM$, (is) to the prism whose
base (is) parallelogram $QEVR$, and opposite (side) straight-line $ST$ [Props.~11.39, 12.3]. Thus, also, (is) the (sum of the) two prisms---that whose base
(is) parallelogram $KBOL$, and opposite (side) $PM$, and that whose base (is) $LOC$,
and opposite (plane) $PMN$---to (the sum of) the  (two) prisms---that whose base (is) $QEVR$, and opposite
(side) straight-line $ST$, and that whose base (is) triangle $RVF$, and opposite (plane) $STU$
[Prop.~5.12]. And, thus, as base $ABC$ (is) to base $DEF$, so the 
(sum of the first) aforementioned two prisms (is) to the (sum  of the second) aforementioned two prisms.

And, similarly, if pyramids $PMNG$ and $STUH$ are divided into two prisms, and
two pyramids, as base $PMN$ (is) to base $STU$, so (the sum of) the  two prisms in pyramid
$PMNG$ will be to (the sum of) the  two prisms in pyramid $STUH$. But, as base $PMN$ (is) to
base $STU$, so base $ABC$ (is) to base $DEF$. For  the triangles $PMN$ and
$STU$ (are) equal to $LOC$ and $RVF$, respectively. And, thus, as base $ABC$ (is)
to base $DEF$, so (the sum of) the four prisms (is) to (the sum of) the four prisms [Prop.~5.12].
So, similarly, even if we  divide the pyramids left behind into two pyramids
and into two prisms, as base $ABC$ (is) to base $DEF$, so (the sum of) all the prisms in pyramid
$ABCG$ will be to (the sum of) all the equal number of prisms in pyramid $DEFH$. (Which is) the
very thing it was required to show.


\begin{center}
\large{Lemma}
\end{center}

And one may show, as follows, that as triangle $LOC$ is to triangle $RVF$, so the prism whose base (is) triangle $LOC$,
and opposite (plane) $PMN$, (is) to the prism whose base (is) [triangle] $RVF$, and opposite
(plane) $STU$. 

For, in the same figure, let perpendiculars have been conceived (drawn) from (points)
$G$ and $H$ to the planes $ABC$ and $DEF$ (respectively). These clearly turn
out to be equal, on account of the pyramids being assumed (to be) of equal height. And
since two straight-lines, $GC$ and the perpendicular from $G$, are cut by the parallel
planes $ABC$ and $PMN$ they will be cut in the same ratios [Prop.~11.17]. And $GC$ was cut in half by the plane $PMN$ at $N$. Thus, the perpendicular
from $G$ to the plane $ABC$ will also be cut in half by the plane $PMN$. So, for the same
(reasons), the perpendicular from $H$ to the plane $DEF$ will also be cut in half by the
plane $STU$. And  the perpendiculars from $G$ and $H$ to the planes $ABC$ and $DEF$
(respectively) are equal. Thus, the perpendiculars from the triangles $PMN$ and $STU$
to $ABC$ and $DEF$ (respectively, are) also equal. Thus, the prisms whose bases are triangles
$LOC$ and $RVF$, and opposite (sides) $PMN$ and $STU$ (respectively), [are] of equal height.
And, hence, the parallelepiped solids described on the aforementioned prisms
[are] of equal height and (are) to one another as their bases [Prop.~11.32]. Likewise, the halves (of
the solids) [Prop.~11.28].  Thus, as base $LOC$ is to base $RVF$, so the
aforementioned prisms (are) to one another. (Which is) the very thing it was required to show.

%%%%
%12.5
%%%%
\pdfbookmark[1]{Proposition 12.5}{pdf12.5}

\begin{center}
{\large Proposition 5}
\end{center}

Pyramids which are of the same height, and have triangular bases, are to one another
as their bases.

\epsfysize=1.4in
\centerline{\epsffile{Book12/fig05e.eps}}

Let there be pyramids of the same height whose bases (are) the triangles $ABC$ and $DEF$, and apexes the
points $G$ and $H$ (respectively). I say that as base $ABC$ is to base $DEF$, so pyramid $ABCG$
(is) to pyramid $DEFH$.

For if base $ABC$ is not to base $DEF$, as pyramid $ABCG$ (is) to pyramid $DEFH$, then base
$ABC$ will be to base $DEF$, as pyramid $ABCG$ (is) to some solid either less than, or greater
than, pyramid $DEFH$. Let it, first of all, be (in this ratio) to (some) lesser (solid), $W$.
And let pyramid $DEFH$ have been divided into two pyramids equal to one another, and
similar to the whole,  and into two equal prisms. So, the (sum of the) two prisms is greater than
half of the whole pyramid [Prop.~12.3].  And, again, let the pyramids generated
by the division have been similarly divided, and let this be done continually until some pyramids
are left from pyramid $DEFH$ which (when added together) are less than the excess by which pyramid
$DEFH$ exceeds the solid $W$ [Prop.~10.1]. Let them have been left, and, for the sake of argument, let them be $DQRS$ and $STUH$. Thus, the (sum of the) remaining prisms within pyramid $DEFH$ is greater than solid $W$. Let pyramid $ABCG$ also have been divided similarly, and a similar number of times,
as pyramid $DEFH$. Thus, as base $ABC$ is to base $DEF$, so the (sum of the) prisms within pyramid $ABCG$
(is) to the (sum of the) prisms within pyramid $DEFH$ [Prop.~12.4]. But, also, as base
$ABC$ (is) to base $DEF$, so pyramid $ABCG$ (is) to solid $W$. And, thus, as pyramid $ABCG$ (is)
to solid $W$, so the (sum of the) prisms within pyramid $ABCG$ (is) to the (sum of the)
prisms within pyramid $DEFH$ [Prop.~5.11]. Thus, alternately, as pyramid $ABCG$ (is) to the (sum of the) prisms
within it, so solid $W$ (is) to the (sum of the) prisms within pyramid $DEFH$ [Prop.~5.16]. 
And  pyramid $ABCG$ (is) greater than the (sum of the) prisms within it. Thus, solid $W$ (is)
also greater than the (sum of the) prisms within pyramid $DEFH$ [Prop.~5.14]. But,
(it is) also less. This very thing is impossible. Thus, as base $ABC$ is to base $DEF$, so pyramid
$ABCG$ (is) not to some solid less than pyramid $DEFH$. So, similarly, we can show that 
base $DEF$ is not to base $ABC$, as pyramid $DEFH$ (is) to some solid less than pyramid $ABCG$ either.

So, I say that neither is base $ABC$ to base $DEF$, as pyramid $ABCG$ (is) to 
some solid greater
than pyramid $DEFH$.

For, if possible, let it be (in this ratio) to some greater (solid), $W$. Thus, inversely, as base $DEF$ (is) to base
$ABC$, so solid $W$ (is) to pyramid $ABCG$ [Prop.~5.7.~corr.]. And as solid $W$ (is) to pyramid $ABCG$, so pyramid
$DEFH$ (is) to some (solid) less than pyramid $ABCG$, as shown before [Prop.~12.2~lem.]. And, thus, as base $DEF$ (is) to base
$ABC$, so pyramid $DEFH$ (is) to some (solid) less than pyramid 
$ABCG$ [Prop.~5.11]. The very thing
was shown (to be) absurd. Thus, base $ABC$ is not to base $DEF$, as pyramid $ABCG$ (is) to some
solid greater than pyramid $DEFH$. And, it was shown that neither (is it in this ratio) to a lesser (solid). Thus, as base
$ABC$ is to base $DEF$, so pyramid $ABCG$ (is) to pyramid $DEFH$. (Which is) the very
thing it was required to show.

%%%%
%12.6
%%%%
\pdfbookmark[1]{Proposition 12.6}{pdf12.6}

\begin{center}
{\large Proposition 6}
\end{center}

Pyramids which are of the same height, and have polygonal bases, are to one
another as their bases.\\

\epsfysize=2.in
\centerline{\epsffile{Book12/fig06e.eps}}

Let there be pyramids of the same height whose bases (are) the polygons $ABCDE$ and $FGHKL$,
and apexes the points $M$ and $N$ (respectively). I say that as base $ABCDE$ is to base $FGHKL$,
so pyramid $ABCDEM$ (is) to pyramid $FGHKLN$.

For let $AC$, $AD$, $FH$, and $FK$ have been joined. Therefore, since $ABCM$ and
$ACDM$ are two pyramids having triangular bases and equal height, they are to one another
as their bases [Prop.~12.5]. Thus, as base $ABC$ is to base $ACD$, so pyramid 
$ABCM$ (is) to pyramid $ACDM$.  And, via composition, as base $ABCD$ (is) to base $ACD$, so
pyramid $ABCDM$ (is) to pyramid $ACDM$ [Prop.~5.18]. But, as base
$ACD$ (is) to base $ADE$, so pyramid $ACDM$ (is) also to pyramid $ADEM$ [Prop.~12.5]. Thus, via equality, 
as base $ABCD$ (is) to base $ADE$, so pyramid $ABCDM$ (is) to  pyramid $ADEM$ [Prop.~5.22]. And, again, via composition, as base $ABCDE$ (is) to base $ADE$,
so pyramid $ABCDEM$ (is) to pyramid $ADEM$ [Prop.~5.18]. 
So, similarly, it can also be shown that as base $FGHKL$ (is) to base $FGH$, so pyramid 
$FGHKLN$ (is) also to pyramid $FGHN$. And since $ADEM$ and $FGHN$ are two
pyramids having triangular bases and equal height,  thus as base $ADE$ (is) to base $FGH$, so
pyramid $ADEM$ (is) to pyramid $FGHN$ [Prop.~12.5]. 
But, as base $ADE$ (is) to base $ABCDE$,  so pyramid $ADEM$ (was) to 
pyramid $ABCDEM$. Thus, via equality,  as base $ABCDE$ (is) to base $FGH$,
so pyramid $ABCDEM$ (is) also to pyramid $FGHN$ [Prop.~5.22]. But,
furthermore, as base $FGH$ (is) to base $FGHKL$, so pyramid $FGHN$ was also to pyramid
$FGHKLN$. Thus, via equality, as base $ABCDE$ (is) to base $FGHKL$, so
pyramid $ABCDEM$ (is) also to pyramid $FGHKLN$  [Prop.~5.22]. (Which is) the very thing it was required
to show.

%%%%
%12.7
%%%%
\pdfbookmark[1]{Proposition 12.7}{pdf12.7}

\begin{center}
{\large Proposition 7}
\end{center}

Any prism having a triangular base is divided into three pyramids having triangular bases (which are) equal to
one another.

\epsfysize=2.5in
\centerline{\epsffile{Book12/fig07e.eps}}

Let there be a prism whose base (is) triangle $ABC$, and opposite (plane) $DEF$. I say that
prism $ABCDEF$ is divided into three pyramids having
triangular bases (which are) equal to one another.

For let $BD$, $EC$, and $CD$ have been joined. Since $ABED$ is a parallelogram, and $BD$ is its diagonal, triangle $ABD$ is thus equal to triangle $EBD$ [Prop.~1.34]. And, thus,
the pyramid whose base (is) triangle $ABD$, and apex the point $C$, is equal to the pyramid
whose base is triangle $DEB$, and apex the point $C$ [Prop.~12.5]. But,
the pyramid whose base is triangle $DEB$, and apex the point $C$, is the same as the pyramid whose
base is triangle $EBC$, and apex the point $D$. For they are contained
by the same planes. And, thus, the pyramid whose base is $ABD$, and
apex the point $C$, is equal to the pyramid whose base is $EBC$ and
apex the point $D$. 
Again, since $FCBE$ is a parallelogram, and
$CE$ is its diagonal, triangle $CEF$ is equal to triangle $CBE$ [Prop.~1.34].
And, thus, the pyramid whose base is triangle $BCE$, and apex the point $D$, is equal to the
pyramid whose base is triangle $ECF$, and apex the point $D$ [Prop.~12.5]. And the pyramid whose base
is triangle $BCE$, and apex the point $D$, was shown (to be) equal to the pyramid whose base is
triangle $ABD$, and apex the point $C$. Thus, the pyramid whose base is triangle $CEF$,
and apex the point $D$, is also equal to the pyramid whose base [is] triangle $ABD$, and apex
the point $C$. Thus, the prism $ABCDEF$ has been divided into three pyramids having triangular bases
(which are) equal to one another.

And since the pyramid whose base is triangle $ABD$, and apex the point $C$, is the
same as the pyramid whose base is triangle $CAB$, and apex the point $D$. For they are contained
by the same planes. And the pyramid whose base (is) triangle $ABD$, and apex the point
$C$, was shown (to be) a third of the prism whose base is triangle $ABC$, and opposite
(plane) $DEF$, thus the pyramid whose base is triangle $ABC$, and apex the point $D$, is  also
a third of the pyramid having the same base, triangle $ABC$, and opposite (plane)  $DEF$.


\begin{center}
\large{Corollary}
\end{center}

And, from this, (it is) clear that any pyramid is the third part of the prism having the same
base as it, and an equal height. (Which is) the very thing it was required to show.

%%%%
%12.8
%%%%
\pdfbookmark[1]{Proposition 12.8}{pdf12.8}

\begin{center}
{\large Proposition 8}
\end{center}

Similar pyramids which also have triangular bases are in the cubed ratio of their
corresponding sides.

Let there be similar, and similarly laid out, pyramids whose bases are triangles $ABC$ and $DEF$, and apexes
the points $G$ and $H$ (respectively). I say that pyramid $ABCG$ has to pyramid $DEFH$ the cubed ratio
of that $BC$ (has) to $EF$. 

\epsfysize=2.in
\centerline{\epsffile{Book12/fig08e.eps}}

For let the parallelepiped solids $BGML$ and $EHQP$ have been completed. And since pyramid $ABCG$
is similar to pyramid $DEFH$, angle $ABC$ is thus equal to angle $DEF$, and $GBC$ to $HEF$, and $ABG$
to $DEH$. And as $AB$ is to $DE$, so $BC$ (is) to $EF$, and $BG$ to $EH$ [Def.~11.9]. 
And since as $AB$ is to $DE$, so $BC$ (is) to $EF$, and (so) the sides around equal angles are proportional, parallelogram
$BM$ is thus similar to paralleleogram $EQ$. So, for the same (reasons), $BN$ is also similar to $ER$, and $BK$
to $EO$. Thus, the three (parallelograms) $MB$, $BK$, and $BN$ are similar to the three (parallelograms) $EQ$, $EO$,
$ER$ (respectively). But, the three (parallelograms) $MB$, $BK$, and $BN$ are (both) equal and similar to the three  opposite (parallelograms),
and the three (parallelograms) $EQ$, $EO$, and $ER$ are (both) equal and similar to the three opposite
 (parallelograms) [Prop.~11.24]. Thus, the solids $BGML$ and $EHQP$ are contained by
 equal numbers of similar (and similarly laid out) planes. Thus, solid $BGML$ is similar to solid $EHQP$ [Def.~11.9].
 And similar parallelepiped solids are in the cubed ratio of corresponding sides [Prop.~11.33].
 Thus, solid $BGML$ has to solid $EHQP$ the cubed ratio that the corresponding side $BC$ (has) to the corresponding
 side $EF$. And as solid $BGML$ (is) to solid $EHQP$, so pyramid $ABCG$ (is) to pyramid
 $DEFH$, inasmuch as the pyramid is the sixth part of the solid, on account of the prism, being half of the
 parallelepiped solid [Prop.~11.28], also being three times the pyramid [Prop.~12.7].
 Thus, pyramid $ABCG$ also has to pyramid $DEFH$ the cubed ratio that $BC$ (has) to $EF$. (Which is)
 the very thing it was required to show.


 \begin{center}
 \large{Corollary}
 \end{center}
 
 So, from this, (it is) also clear that similar pyramids  having polygonal bases (are) to one another as the cubed
 ratio of their corresponding sides. For, dividing them into  the pyramids (contained) within them which have triangular bases, with 
 the similar  polygons of the bases also being divided into similar triangles (which are) both equal in number, and corresponding,  to the wholes
 [Prop.~6.20].
 As  one pyramid having a triangular base in the former (pyramid having a polygonal base is)  to  one pyramid 
 having a triangular  base  in the latter (pyramid having a polygonal base),  so  (the sum of) all the pyramids having triangular bases in  the former pyramid will also
 be to (the sum of) all the pyramids having triangular bases in the latter  pyramid [Prop.~5.12]---that is to say, the (former) pyramid itself
 having  a polygonal base to the (latter) pyramid having a polygonal base. And a pyramid having a triangular base is to a (pyramid) having a triangular base in the cubed ratio of
 corresponding sides [Prop.~12.8]. Thus, a (pyramid) having a polygonal base also has to  to a (pyramid) having a similar base
 the
 cubed ratio of a (corresponding) side to a (corresponding) side.
 
%%%%
%12.9
%%%%
\pdfbookmark[1]{Proposition 12.9}{pdf12.9}

\begin{center}
{\large Proposition 9}
\end{center}

The bases of equal pyramids which also have triangular bases are reciprocally
proportional to their heights. And those pyramids which have triangular bases whose bases are reciprocally
proportional to their heights  are equal.

\epsfysize=2.5in
\centerline{\epsffile{Book12/fig09e.eps}}

For let there be (two) equal pyramids having the triangular bases $ABC$ and $DEF$, and apexes
the points $G$ and $H$ (respectively). I say that the bases of the pyramids $ABCG$ and
$DEFH$ are reciprocally proportional to their heights, and (so) that as base $ABC$ is to base
$DEF$, so the height of pyramid $DEFH$ (is) to the height of pyramid $ABCG$.

For let the parallelepiped solids $BGML$ and $EHQP$ have been completed. And since pyramid
$ABCG$ is equal to pyramid $DEFH$, and solid $BGML$ is six times pyramid $ABCG$ (see previous
proposition), and solid $EHQP$ (is) six times pyramid $DEFH$, solid $BGML$ is thus equal to
solid $EHQP$. And the bases of equal parallelepiped solids are reciprocally proportional
to their heights [Prop.~11.34]. Thus, as base $BM$ is to base $EQ$, so the height of
solid $EHQP$ (is) to the height of solid $BGML$. But, as base $BM$ (is) to base $EQ$, so triangle
$ABC$ (is) to triangle $DEF$ [Prop.~1.34]. And, thus, as triangle $ABC$ (is) to triangle
$DEF$, so the height of solid $EHQP$ (is) to the height of solid $BGML$ [Prop.~5.11]. But, the height of
solid $EHQP$ is the same as the height of pyramid $DEFH$, and the height of solid $BGML$ is the
same as the height of pyramid $ABCG$. Thus, as base $ABC$ is to base $DEF$, so the
height of pyramid $DEFH$ (is) to the height of pyramid $ABCG$. Thus, the bases of pyramids
$ABCG$ and $DEFH$ are reciprocally proportional to their heights.

And so, let the bases of pyramids $ABCG$ and $DEFH$ be reciprocally proportional to their heights, and (thus) let
base $ABC$ be to base $DEF$, as the height of pyramid $DEFH$ (is) to the height of pyramid $ABCG$.
I say that pyramid $ABCG$ is equal to pyramid $DEFH$.

For, with the same construction, since as base $ABC$ is to base $DEF$, so the height of pyramid $DEFH$ (is)
to the height of pyramid $ABCG$, but as base $ABC$ (is) to base $DEF$, so parallelogram $BM$ (is)
to parallelogram $EQ$ [Prop.~1.34], thus as parallelogram $BM$ (is) to
parallelogram $EQ$, so the height of pyramid $DEFH$  (is) also to the height of pyramid $ABCG$ [Prop.~5.11]. But, 
the height of pyramid $DEFH$ is the same as the height of parallelepiped $EHQP$, and the height of
pyramid $ABCG$ is the same as the height of parallelepiped $BGML$. Thus, as base $BM$ is to
base $EQ$, so the height of parallelepiped $EHQP$ (is) to the height of parallelepiped $BGML$.
And those parallelepiped solids whose bases are reciprocally proportional to their
heights are equal [Prop.~11.34]. Thus, the parallelepiped solid $BGML$ is equal to the parallelepiped
solid $EHQP$. And pyramid $ABCG$ is a sixth part of $BGML$, and pyramid $DEFH$ a sixth part of
parallelepiped $EHQP$. Thus, pyramid $ABCG$ is equal to pyramid $DEFH$.

Thus,  the bases of equal pyramids which also have triangular bases are reciprocally
proportional to their heights. And those pyramids having triangular bases whose bases are reciprocally
proportional to their heights  are equal. (Which is) the very thing it was required to show.

%%%%
%12.10
%%%%
\pdfbookmark[1]{Proposition 12.10}{pdf12.10}

\begin{center}
{\large Proposition 10}
\end{center}

Every cone is the third part of the cylinder which has the same base as it, and an equal height.

For let there be a cone (with) the same base as a cylinder, (namely) the circle $ABCD$, and an equal height. I say that the cone is
the third part of the cylinder---that is to say, that the cylinder is three times the cone.

\epsfysize=2.5in
\centerline{\epsffile{Book12/fig10e.eps}}

For if the cylinder is not three times the cone then the cylinder will be either more than three times, or less than three times, (the cone). Let it, first
of all, be  more than three times (the cone). And  let the square $ABCD$ have been inscribed in circle $ABCD$ [Prop.~4.6].
So, square $ABCD$ is more than half of circle $ABCD$  [Prop.~12.2]. And let a prism
of equal height to the cylinder have been set up on square $ABCD$. So, the prism set up is more than half of the
cylinder, inasmuch as  if we also circumscribe a square around circle $ABCD$ [Prop.~4.7] then the square inscribed in circle $ABCD$ is half of the circumscribed (square). And the  solids  set up on them are parallelepiped
prisms  of equal height.  And parallelepiped solids having the same height are to one
another as their bases [Prop.~11.32]. And, thus, the prism set up on square $ABCD$ is half
of the prism set up on the square circumscribed about circle $ABCD$. And the cylinder is less than the prism
set up on the square circumscribed about circle $ABCD$. Thus, the prism set up on square $ABCD$ of the
same height as the cylinder is more than half of the cylinder. Let the circumferences $AB$, $BC$, $CD$, and $DA$
have been cut in half at points $E$, $F$, $G$, and $H$. And let $AE$, $EB$, $BF$, $FC$, $CG$, $GD$, $DH$,
and $HA$ have been joined. And thus each of the triangles $AEB$, $BFC$, $CGD$, and $DHA$ is more than
half of the segment of circle $ABCD$ about it, as was shown previously [Prop.~12.2].
Let prisms of equal height to the cylinder have been set up on each of the triangles $AEB$, $BFC$, $CGD$,
and $DHA$.  And each of the prisms set up is greater than the half part of the segment of the cylinder about it---inasmuch
as if we  draw (straight-lines) parallel to $AB$, $BC$, $CD$, and $DA$ through points $E$, $F$, $G$, and $H$
(respectively), and complete the parallelograms on $AB$, $BC$, $CD$, and $DA$, and set up parallelepiped solids
of equal height to the cylinder on them, then the prisms on triangles $AEB$, $BFC$, $CGD$, and $DHA$ are each half
of the set up (parallelepipeds). And the segments of the cylinder are less than the set up parallelepiped solids. 
Hence, the prisms on triangles $AEB$, $BFC$, $CGD$, and $DHA$ are also greater than half of the segments
of the cylinder about them. So (if) the remaining circumferences are cut in half, and straight-lines are joined,
and prisms of equal height to the cylinder are set up on each of the triangles, and this is done
continually,  then we will (eventually) leave some segments of the cylinder whose (sum) is less than the excess by which the
cylinder exceeds three times the cone [Prop.~10.1]. Let them have been left, and let them be $AE$, $EB$, $BF$, $FC$, $CG$,
$GD$, $DH$, and $HA$. Thus, the remaining prism whose base (is) polygon $AEBFCGDH$, and height  the
same as the cylinder, is greater than three times the cone. But, the prism whose base is polygon $AEBFCGDH$,
and height the same as the cylinder,  is three times the pyramid whose base is polygon $AEBFCGDH$, and
apex the same as the cone [Prop.~12.7~corr.]. And thus the pyramid whose base [is] polygon 
$AEBFCGDH$, and apex the same as the cone, is greater than the cone having (as) base circle $ABCD$. But (it is)
also less. For it is encompassed by it. The very thing (is) impossible. Thus, the cylinder is not more than
three times the cone.

So, I say that neither (is) the cylinder less than three times the cone.

For, if possible, let the cylinder be less than three times the cone. Thus, inversely, the cone is greater than the third
part of the cylinder. So, let the square $ABCD$ have been inscribed in circle $ABCD$  [Prop.~4.6]. Thus, square $ABCD$ is greater
than half of circle $ABCD$. 
And let a pyramid having the same apex as the cone have been set up on square  $ABCD$. Thus, the pyramid set up
is greater than the half part of the cone, inasmuch as we showed previously that if we circumscribe a square
about the circle [Prop.~4.7] then the square $ABCD$ will be half of the square circumscribed about the circle [Prop.~12.2]. And if we set up   on the squares parallelepiped solids---which are also
called prisms---of the same height as the cone, then the (prism) set up on square $ABCD$ will be half of the (prism) set up on the square circumscribed
about the circle. For they are to one another as their bases [Prop.~11.32]. Hence, (the same) also (goes for) the thirds.  Thus, the pyramid whose base is square $ABCD$ is half of the pyramid set up on the square
circumscribed about the circle [Prop.~12.7~corr.]. And the pyramid set up on the square
circumscribed about the circle is greater than the cone. For it encompasses it. Thus, the pyramid whose
base is square $ABCD$, and apex the same as the cone, is greater than half of the cone. Let the circumferences $AB$,
$BC$, $CD$, and $DA$ have been cut in half at points $E$, $F$, $G$, and $H$ (respectively).  And let $AE$, $EB$, $BF$,
$FC$, $CG$, $GD$, $DH$, and $HA$ have been joined. And, thus, each of the triangles $AEB$, $BFC$,
$CGD$, and $DHA$ is greater than the half part of the segment of circle $ABCD$ about it [Prop.~12.2].
And let pyramids having the same apex as the cone have been set up on each of the triangles $AEB$,
$BFC$, $CGD$, and $DHA$. And, thus, in the same way, each of the pyramids set up is more than the  half part of
the segment of the cone about it. So, (if) the remaining circumferences are cut in half, and straight-lines
are joined, and pyramids having the same apex as the cone are set up on each of the triangles, and this is
done continually, then we will (eventually) leave some segments of the cone 
whose (sum) is less than
the excess by which the cone exceeds the third part of the cylinder [Prop.~10.1]. 
Let them have been left, and let them be the (segments) on $AE$, $EB$, $BF$, $FC$, $CG$, $GD$,
$DH$, and $HA$. Thus, the remaining pyramid whose base is polygon $AEBFCGDH$, and apex the
same as the cone, is greater than the third part of the cylinder. But, the pyramid whose base is
polygon $AEBFCGDH$, and apex the same as the cone, is the third part of the prism whose base
is polygon $AEBFCGDH$, and height the same as the cylinder [Prop.~12.7~corr.].
Thus, the prism whose base is polygon $AEBFCGDH$, and height the same as the cylinder, is greater than
the cylinder whose base is circle $ABCD$. But, (it is) also less. For it is encompassed by it. The very thing
is impossible. Thus, the cylinder is not less than three times the cone. And it was shown that neither (is it)
greater than three times (the cone). Thus, the cylinder (is) three times the cone. Hence, the cone is the
third part of the cylinder.

Thus, every cone is the third part of the cylinder which has the same base as it, and an equal height.
(Which is) the very thing it was required to show.

%%%%
%12.11
%%%%
\pdfbookmark[1]{Proposition 12.11}{pdf12.11}

\begin{center}
{\large Proposition 11}
\end{center}

Cones and cylinders having the same height are to one another as their bases.

Let there be cones and cylinders of the same height whose bases [are]  the circles $ABCD$ and $EFGH$,
axes $KL$ and $MN$, and diameters of the bases $AC$ and $EG$ (respectively). I say that as
circle $ABCD$ is to circle $EFGH$, so cone  $AL$ (is) to cone $EN$.

\epsfysize=1.3in
\centerline{\epsffile{Book12/fig11e.eps}}

For if not, then as circle $ABCD$ (is) to circle $EFGH$, so cone $AL$ will be to some solid either less than, or
greater than, cone $EN$. Let it, first of all, be (in this ratio) to (some) lesser (solid), $O$. And let solid $X$ be equal to
that (magnitude) by which solid $O$ is less than cone $EN$. Thus, cone $EN$ is equal to (the sum of) solids
$O$ and $X$. Let the square $EFGH$ have been inscribed in circle $EFGH$ [Prop.~4.6].
Thus, the square is greater than half of the circle [Prop.~12.2]. Let a pyramid
of the same height as the cone have been set up on square $EFGH$. Thus, the pyramid set up is
greater than half of the cone, inasmuch as, if we circumscribe a square about the circle [Prop.~4.7], and set up  on it a pyramid of the same height as the cone, then the inscribed pyramid
is half of the circumscribed pyramid.  For they are to one another as their 
bases [Prop.~12.6]. And the cone (is) less than the circumscribed pyramid. 
Let the circumferences $EF$, $FG$, $GH$, and $HE$ have been cut in half at points $P$, $Q$, $R$,
and $S$. And let $HP$, $PE$, $EQ$, $QF$, $FR$, $RG$, $GS$, and $SH$ have been joined.
Thus, each of the triangles $HPE$, $EQF$, $FRG$, and $GSH$ is greater than half of the segment of
the circle about it [Prop.~12.2]. Let pyramids of the same height
as the cone have been set up on each of the triangles $HPE$, $EQF$, $FRG$, and $GSH$.  And, thus, each of
the pyramids set up is greater than half of the segment of the cone about it [Prop.~12.10].
So, (if) the remaining circumferences are cut in half, and straight-lines are joined, and pyramids  of equal height to the cone are
set up on each of the triangles, and this is 
done continually, then we will (eventually) leave some segments of the cone (the sum of) which is less 
than solid $X$ [Prop.~10.1]. Let them have been left, and let them be the
(segments) on $HPE$, $EQF$, $FRG$, and $GSH$. Thus, the remaining pyramid whose base is
polygon $HPEQFRGS$, and height the same as the cone, is greater than solid $O$ [Prop.~6.18]. 
And let the polygon $DTAUBVCW$, similar, and similarly laid out, to polygon $HPEQFRGS$, have been
inscribed in circle $ABCD$. And on it let a pyramid of the same height as cone 
$AL$ have been set up.
Therefore, since as the (square) on $AC$ is to the (square) on $EG$, so polygon $DTAUBVCW$ (is) to
polygon $HPEQFRGS$ [Prop.~12.1], and as the (square) on $AC$ (is) to the (square) on $EG$,
so circle $ABCD$ (is) to circle $EFGH$ [Prop.~12.2], thus as circle $ABCD$ (is) to
circle $EFGH$, so polygon $DTAUBVCW$ also (is) to polygon $HPEQFRGS$. And as circle $ABCD$ (is) to circle
$EFGH$, so cone $AL$ (is) to solid $O$. And as polygon $DTAUBVCW$ (is) to polygon $HPEQFRGS$, so the
pyramid whose base is polygon $DTAUBVCW$, and apex the point $L$, (is) to  the pyramid whose base is
polygon $HPEQFRGS$, and apex the point $N$ [Prop.~12.6]. And, thus, as cone $AL$ (is)
to solid $O$, so the pyramid whose base is $DTAUBVCW$, and apex the point $L$, (is) to  the pyramid whose base is
polygon $HPEQFRGS$, and apex the point $N$ [Prop.~5.11]. Thus, alternately, as cone $AL$ is to the pyramid within it, so solid $O$
(is) to the pyramid within cone $EN$ [Prop.~5.16]. But, cone $AL$ (is) greater than the
pyramid within it. Thus, solid $O$ (is) also greater than the pyramid within cone $EN$ [Prop.~5.14]. But,
(it is) also less. The very thing (is) absurd. Thus, circle $ABCD$ is not to circle $EFGH$, as cone $AL$
(is) to some solid less than cone $EN$. So, similarly, we can show that neither is circle $EFGH$
to circle $ABCD$, as cone $EN$ (is) to some solid less than cone $AL$.

So, I say that neither is circle $ABCD$ to circle $EFGH$, as cone $AL$ (is) to some solid greater than
cone $EN$.

For, if possible, let it be (in this ratio) to (some) greater (solid), $O$. Thus, inversely, as circle $EFGH$ is to circle
$ABCD$, so solid $O$ (is) to cone $AL$ [Prop.~5.7~corr.]. But, as solid $O$ (is) to cone $AL$, so cone $EN$ (is) to some
solid less than cone $AL$ [Prop.~12.2~lem.].  And, thus, as circle $EFGH$
(is)  to circle $ABCD$, so cone $EN$ (is) to some solid less than cone $AL$. The very thing was shown
(to be) impossible. Thus, circle $ABCD$ is not to circle $EFGH$, as cone $AL$ (is) to some solid greater than
cone $EN$. And, it was shown that neither (is it in this ratio) to (some) lesser (solid). Thus, as circle $ABCD$ is to circle
$EFGH$, so cone $AL$ (is) to cone $EN$.

But, as the cone (is) to the cone, (so) the cylinder (is) to the cylinder. For each (is) three times each [Prop.~12.10]. Thus, circle $ABCD$ (is) also to circle $EFGH$, as  (the ratio of the cylinders) on them (having) the
same height.

Thus, cones and cylinders having the same height are to one another as their bases. (Which is) the very thing
it was required to show.

%%%%
%12.12
%%%%
\pdfbookmark[1]{Proposition 12.12}{pdf12.12}

\begin{center}
{\large Proposition 12}
\end{center}

Similar cones and cylinders are to one another in the cubed ratio of
the diameters of their bases.

Let there be similar cones and cylinders of which the bases (are) the circles $ABCD$ and $EFGH$, the diameters of the bases (are) $BD$ and $FH$,
and the axes of the cones and cylinders (are)  $KL$ and $MN$ (respectively). I say that the cone whose base [is] circle $ABCD$, and apex the
point $L$,  has to the cone whose base [is] circle $EFGH$, and apex the point $N$,  the cubed ratio  that $BD$ (has) to $FH$.

\epsfysize=1.7in
\centerline{\epsffile{Book12/fig12e.eps}}

For if cone $ABCDL$ does not have to cone $EFGHN$ the cubed ratio that $BD$ (has) to $FH$ then cone $ABCDL$ will  have the
cubed ratio to some solid either less than, or greater than, cone $EFGHN$. Let it, first of all, have (such a  ratio) to
(some) lesser (solid), $O$. And let the square $EFGH$ have been inscribed in circle $EFGH$ [Prop.~4.6]. 
Thus, square $EFGH$ is greater than half of circle $EFGH$ [Prop.~12.2]. And let a pyramid having the
same apex as the cone have been set up on square $EFGH$. Thus, the pyramid set up is greater than the half part of the cone [Prop.~12.10]. So, let the circumferences $EF$, $FG$, $GH$, and $HE$ have been cut in half at points $P$, $Q$, $R$, and $S$ (respectively). 
And let $EP$, $PF$, $FQ$, $QG$, $GR$, $RH$, $HS$, and $SE$ have been
joined. And, thus, each of the triangles $EPF$, $FQG$, $GRH$, and $HSE$
is greater than the half part of the segment of circle $EFGH$ about it  [Prop.~12.2]. And let a pyramid having the same apex as
the cone have been set up on each of the triangles $EPF$, $FQG$, $GRH$, and $HSE$. And thus each of the pyramids set up is greater than
the half part of the segment of the cone about it [Prop.~12.10]. So, (if) the the remaining circumferences are cut in half, and 
straight-lines are joined, and pyramids having the same apex as the cone are set up on each of 
the triangles, and this is done continually, then we will (eventually) leave some segments of the 
cone whose (sum) is less than the excess by which cone $EFGHN$ exceeds solid $O$ [Prop.~10.1].
Let them have been left, and let them be the (segments) on $EP$, $PF$, $FQ$, $QG$, $GR$, $RH$, $HS$, and $SE$. Thus, the remaining
pyramid whose base is polygon $EPFQGRHS$, and apex the point $N$, is greater than solid $O$.  And let the polygon
$ATBUCVDW$, similar, and similarly laid out, to polygon $EPFQGRHS$, have been inscribed in circle $ABCD$ [Prop.~6.18].
And let a pyramid having the same apex as the cone have been set up on polygon $ATBUCVDW$. And let $LBT$ be one
of the triangles containing the pyramid whose base is polygon $ATBUCVDW$, and apex the point $L$. And let
$NFP$ be one of the triangles containing the pyramid whose base is triangle $EPFQGRHS$, and apex the point
$N$. And let $KT$ and $MP$ have been joined. And since cone $ABCDL$ is similar to cone $EFGHN$, thus as
$BD$ is to $FH$, so axis $KL$ (is) to axis $MN$ [Def.~11.24].  And as $BD$ (is) to
$FH$, so $BK$ (is) to $FM$. And, thus, as $BK$ (is) to $FM$, so $KL$ (is) to $MN$. And, alternately, as $BK$ (is)
to $KL$, so $FM$ (is) to $MN$ [Prop.~5.16]. And the sides around the equal angles $BKL$ and
$FMN$ are proportional. Thus, triangle $BKL$ is similar to triangle $FMN$
[Prop.~6.6]. Again, since as $BK$ (is) to $KT$, so
$FM$ (is) to $MP$, and (they are) about the equal angles $BKT$ and $FMP$, inasmuch as whatever part angle $BKT$
is of the four right-angles at the center $K$, angle $FMP$ is also the same part of the four right-angles at the  center $M$. 
Therefore, since the sides about equal angles are proportional,  triangle $BKT$ is thus similar to traingle $FMP$ [Prop.~6.6].
Again, since it was shown that as $BK$ (is) to $KL$, so $FM$ (is) to $MN$, and $BK$ (is) equal to $KT$, and
$FM$ to $PM$, thus as $TK$ (is) to $KL$, so $PM$ (is) to $MN$. And the sides about the equal angles
$TKL$ and $PMN$---for (they are both) right-angles---are proportional.  Thus, triangle $LKT$ (is) similar to
triangle $NMP$ [Prop.~6.6]. And since, on account of the similarity of triangles $LKB$ and
$NMF$, as $LB$ (is) to $BK$, so $NF$ (is) to $FM$, and, on account of the  similarity of triangles $BKT$ and
$FMP$, as $KB$ (is) to $BT$, so $MF$ (is) to $FP$ [Def.~6.1], thus, via equality, as $LB$ (is)
to $BT$, so $NF$ (is) to $FP$ [Prop.~5.22]. Again, since, on account of the similarity of
triangles $LTK$ and $NPM$, as $LT$ (is) to $TK$, so $NP$ (is) to $PM$, and, on account of the similarity of
triangles $TKB$ and $PMF$, as $KT$ (is) to $TB$, so $MP$ (is) to $PF$, thus, via equality,  as $LT$ (is) to $TB$, so
$NP$ (is) to $PF$ [Prop.~5.22]. And it was shown that as $TB$ (is) to $BL$, so $PF$ (is) to $FN$.
 Thus, via equality, as $TL$ (is) to $LB$, so $PN$ (is) to $NF$ [Prop.~5.22]. Thus, the sides of triangles
 $LTB$ and $NPF$ are proportional. Thus, triangles $LTB$ and $NPF$ are equiangular [Prop.~6.5]. 
 And, hence, (they are) similar [Def.~6.1]. And, thus, the pyramid whose base is triangle $BKT$, and
 apex the point $L$, is similar to the pyramid whose base is triangle $FMP$, and apex the point $N$. For they are contained
 by equal numbers of similar planes [Def.~11.9]. And similar pyramids which also have triangular
 bases are in the cubed ratio of corresponding sides [Prop.~12.8]. Thus, pyramid
 $BKTL$ has to pyramid $FMPN$ the cubed ratio that $BK$ (has) to $FM$. So, similarly, joining
 straight-lines from (points) $A$, $W$, $D$, $V$, $C$, and $U$ to (center) $K$, and from (points) $E$, $S$, $H$, $R$,
 $G$,  and $Q$ to (center) $M$, and setting up pyramids having the same apexes as the cones on each of the triangles (so formed), we can also show
 that each of the pyramids (on base $ABCD$ taken) in order will have to each of the pyramids (on base $EFGH$ taken) in order the cubed ratio that the corresponding side
 $BK$ (has) to the corresponding side $FM$---that is to say, that $BD$ (has) to $FH$. And (for two sets of proportional magnitudes) as one of the leading
 (magnitudes is) to one of the following, so (the sum of) all of the leading 
 (magnitudes is) to (the sum of) all of the
 following (magnitudes) [Prop.~5.12]. And, thus, as pyramid $BKTL$ (is) to pyramid $FMPN$,
 so the whole pyramid whose base is polygon $ATBUCVDW$, and apex the point $L$, (is) to the whole pyramid
 whose base is polygon $EPFQGRHS$, and apex the point $N$. And, hence, the pyramid whose base is polygon $ATBUCVDW$, and apex the point $L$, has to the pyramid
 whose base is polygon $EPFQGRHS$, and apex the point $N$, the cubed ratio that $BD$ (has) to $FH$. And it was also assumed that the cone whose base is circle $ABCD$, and apex the point $L$, has to solid $O$ the cubed ratio that $BD$ (has)
 to $FH$. Thus, as the cone whose base is circle $ABCD$, and apex the point $L$, is to solid $O$, so the pyramid
 whose base (is) [polygon] $ATBUCVDW$, and apex the point $L$, (is) to the pyramid whose base is polygon $EPFQGRHS$, and apex the point $N$. Thus, alternately, as the cone whose base (is) circle $ABCD$, and apex the point $L$, (is)
 to the pyramid within it whose base (is) the polygon $ATBUCVDW$, and apex the point $L$, so the [solid] $O$ (is)
 to the pyramid whose base is polygon  $EPFQGRHS$, and apex the point $N$ [Prop.~5.16]. And the aforementioned cone (is)
 greater than the pyramid within it. For it encompasses it. Thus, solid $O$ (is) also greater than the pyramid whose base is polygon  $EPFQGRHS$, and apex the point $N$. But, (it is) also less. The very thing is impossible. Thus, the cone whose
 base (is) circle $ABCD$, and apex the [point] $L$, does not have to some solid less than the cone whose
 base (is) circle $EFGH$, and apex the point $N$, the cubed ratio that $BD$ (has) to $EH$.  So, similarly,
 we can show that neither does cone $EFGHN$ have to some solid less than cone $ABCDL$ the cubed
 ratio that $FH$ (has) to $BD$.
 
So, I say that neither does cone $ABCDL$ have to some solid greater than cone $EFGHN$ the cubed ratio
that $BD$ (has) to $FH$.

For, if possible, let it have (such a ratio) to a greater (solid), $O$. Thus, inversely,  solid $O$ has to
cone $ABCDL$ the cubed ratio that $FH$ (has) to $BD$ [Prop.~5.7~corr.]. 
And as solid $O$ (is) to cone $ABCDL$, so cone $EFGHN$ (is) to some solid less than cone
$ABCDL$ [12.2 lem.]. Thus, cone $EFGHN$ also has to some solid less than cone $ABCDL$ the cubed
ratio that $FH$ (has) to $BD$. The very thing was shown (to be) impossible. Thus, cone $ABCDL$
does not have to some solid greater than cone $EFGHN$ the cubed ratio than $BD$ (has) to $FH$. And
it was shown that neither (does it have such a ratio) to a lesser (solid).  Thus, cone $ABCDL$ has
to cone $EFGHN$ the cubed ratio that $BD$ (has) to $FG$.

And as the cone (is) to the cone, so the cylinder (is) to the cylinder. For a
cylinder is three times a cone on the same base as the cone, and of the same height as it [Prop.~12.10]. 
Thus, the cylinder also has to the cylinder the cubed ratio that $BD$ (has) to $FH$.

Thus, similar cones and cylinders are in the cubed ratio of the diameters of
their bases. (Which is) the
very thing it was required to show.

%%%%
%12.13
%%%%
\pdfbookmark[1]{Proposition 12.13}{pdf12.13}

\begin{center}
{\large Proposition 13}
\end{center}

If a cylinder is cut by a plane which is parallel to the opposite planes (of the cylinder) then
as the cylinder (is) to the cylinder, so the axis will be to the axis.

\epsfysize=1.1in
\centerline{\epsffile{Book12/fig13e.eps}}

For let the cylinder $AD$ have been cut by the plane $GH$ which is parallel to the opposite planes (of the cylinder), $AB$
and $CD$. And let the plane $GH$ have met the axis at point $K$. I say that as cylinder $BG$
is to cylinder $GD$, so axis $EK$ (is) to axis $KF$.

For let axis $EF$ have been produced in each direction to points $L$ and $M$. And let any number whatsoever (of
lengths), $EN$ and $NL$, equal to axis $EK$, be set out (on the axis $EL$), and any number whatsoever
(of lengths), $FO$ and $OM$, equal to (axis) $FK$, (on the axis $KM$). And let the cylinder $PW$, whose bases
(are) the circles $PQ$ and $VW$, have been
conceived on axis $LM$. And let planes parallel to $AB$, $CD$, and the bases of cylinder $PW$, have been produced
through points $N$ and $O$, and let them have made the circles $RS$ and $TU$ around the centers $N$ and $O$ (respectively).
And since axes $LN$, $NE$, and $EK$ are equal to one another,  the cylinders $QR$, $RB$, and $BG$ are to one another
as their bases [Prop.~12.11]. But the bases are equal. Thus, the cylinders $QR$, $RB$, and $BG$
(are) also equal to one another. Therefore, since the axes $LN$, $NE$, and $EK$ are equal to one another, and the cylinders
$QR$, $RB$, and $BG$ are also equal to one another, and the number (of the former) is equal to the number (of the latter), 
thus as many multiples as axis $KL$ is of axis $EK$, so many multiples is cylinder $QG$ also of cylinder $GB$.
And so, for the same (reasons),  as many multiples as axis $MK$  is of axis $KF$, so many multiples is cylinder $WG$
also of cylinder $GD$.  And if axis $KL$ is equal to axis $KM$ then cylinder $QG$ will also be equal to
cylinder $GW$, and if the axis (is) greater than the axis then the cylinder (will also be) greater than the cylinder, and if (the axis is) less
then (the cylinder will also be) less. So, there are four magnitudes---the axes $EK$ and $KF$, and the cylinders
$BG$ and $GD$---and equal multiples have been taken of axis $EK$ and 
cylinder $BG$---(namely), axis $LK$ and
cylinder $QG$---and of axis $KF$
 and  cylinder $GD$---(namely), axis
$KM$ and  cylinder $GW$. And it has been shown that if axis $KL$ exceeds axis $KM$ then cylinder
$QG$ also exceeds  cylinder $GW$, and if (the axes are) equal then (the cylinders are) equal, and if ($KL$ is) less then ($QG$ is) less.
Thus, as axis $EK$ is to axis $KF$, so cylinder $BG$ (is) to cylinder $GD$ [Def.~5.5]. (Which is)
the very thing it was required to show.

%%%%
%12.14
%%%%
\pdfbookmark[1]{Proposition 12.14}{pdf12.14}

\begin{center}
{\large Proposition 14}
\end{center}

Cones and cylinders which are on equal bases are to one another as their heights.

\epsfysize=1.8in
\centerline{\epsffile{Book12/fig14e.eps}}

For let  $EB$ and $FD$ be cylinders on  equal bases, (namely) the circles $AB$ and $CD$ (respectively). I say that as cylinder $EB$
is to cylinder $FD$, so axis $GH$ (is) to axis $KL$.

For let the axis $KL$ have been produced to point $N$. And let $LN$ be made equal to axis $GH$. 
And let the cylinder $CM$ have been conceived about axis $LN$. Therefore, since cylinders $EB$ and $CM$
have the same height they are to one another as their bases [Prop.~12.11]. And the bases are
equal to one another. Thus, cylinders $EB$ and $CM$ are also equal to one another. And since cylinder $FM$ has been cut
by the plane $CD$, which is parallel to its opposite planes, thus as cylinder $CM$ is to cylinder $FD$, so
axis $LN$ (is) to axis $KL$ [Prop.~12.13]. And cylinder $CM$ is equal to cylinder $EB$, and
axis $LN$ to axis $GH$.  Thus, as cylinder $EB$ is to cylinder $FD$, so axis $GH$ (is) to axis $KL$.  And as
cylinder $EB$ (is) to cylinder $FD$, so cone $ABG$ (is) to cone $CDK$ [Prop.~12.10].
Thus, also, as axis $GH$ (is) to axis $KL$, so cone $ABG$ (is) to cone $CDK$, and cylinder $EB$ to cylinder $FD$.
(Which is) the very thing it was required to show.

%%%%
%12.15
%%%%
\pdfbookmark[1]{Proposition 12.15}{pdf12.15}

\begin{center}
{\large Proposition 15}
\end{center}

The bases of equal cones and cylinders are reciprocally proportional to their heights. And, those
cones and cylinders whose bases (are) reciprocally proportional to their heights are equal.

\epsfysize=1.1in
\centerline{\epsffile{Book12/fig15e.eps}}

Let there be equal cones and cylinders whose bases are the circles $ABCD$ and $EFGH$, and  the diameters of (the bases) $AC$ and $EG$, and
(whose) axes (are) $KL$ and $MN$, which are also the heights of the cones and cylinders (respectively).
And let the cylinders $AO$ and $EP$ have
been completed. I say that the bases of cylinders $AO$  and $EP$ are reciprocally proportional to their heights, and (so) as base
$ABCD$ is to base $EFGH$, so height $MN$ (is) to height $KL$.

For height $LK$ is either equal to height $MN$, or not. Let it, first of all, be equal. And cylinder $AO$ is also equal to cylinder $EP$.
And cones and cylinders having the same height are to one another as their bases [Prop.~12.11]. 
Thus, base $ABCD$ (is) also equal to base $EFGH$. And, hence, reciprocally, as base $ABCD$ (is) to base $EFGH$,
so height $MN$ (is) to height $KL$. And so, let height $LK$ not be equal to $MN$, but let $MN$ be greater. And let $QN$, equal to
$KL$, have been cut off from height $MN$. And let the cylinder $EP$ have been cut, through point $Q$, by the plane $TUS$ (which is)
parallel to the planes of the circles $EFGH$ and $RP$. And let cylinder $ES$ have been conceived, with base the circle $EFGH$, and
height $NQ$. And since cylinder $AO$ is equal to cylinder $EP$, thus, as cylinder $AO$ (is) to cylinder $ES$, so cylinder
$EP$ (is) to cylinder $ES$ [Prop.~5.7]. But, as cylinder $AO$ (is) to cylinder $ES$, so base $ABCD$ (is)
to base $EFGH$. For cylinders $AO$ and $ES$ (have)  the same height [Prop.~12.11]. 
And as cylinder $EP$ (is) to (cylinder) $ES$, so height $MN$ (is) to height  $QN$. For cylinder $EP$
has been cut by a plane which is parallel to its opposite planes [Prop.~12.13].
And, thus, as base $ABCD$ is to base $EFGH$, so height $MN$ (is) to height $QN$ [Prop.~5.11]. And height $QN$
(is) equal to height $KL$. Thus, as base $ABCD$ is to base $EFGH$, so height $MN$ (is) to height $KL$.
Thus, the bases of cylinders $AO$ and $EP$ are reciprocally proportional to their heights.

And, so, let the bases of cylinders $AO$ and $EP$ be reciprocally proportional to their heights, and (thus) let
base $ABCD$ be to base $EFGH$, as height $MN$ (is) to height $KL$. I say that cylinder $AO$ is equal to
cylinder $EP$.

For, with the same construction, since  as base $ABCD$ is to base $EFGH$, so  height $MN$ (is) to height $KL$, and
height $KL$ (is) equal to height $QN$, thus, as base $ABCD$ (is) to base $EFGH$, so height $MN$ will be to height $QN$.
But, as base $ABCD$ (is) to base $EFGH$, so cylinder $AO$ (is) to cylinder $ES$. For they are the same height [Prop.~12.11]. 
And as height $MN$ (is) to [height] $QN$, so cylinder $EP$ (is) to cylinder $ES$ [Prop.~12.13].
Thus, as cylinder $AO$ is to cylinder $ES$, so cylinder $EP$ (is) to (cylinder) $ES$ [Prop.~5.11]. Thus, cylinder $AO$ (is)
equal to cylinder $EP$ [Prop.~5.9]. In the same manner, (the proposition can) also (be demonstrated)
for the cones. (Which is) the very thing it was required to show.

%%%%
%12.16
%%%%
\pdfbookmark[1]{Proposition 12.16}{pdf12.16}

\begin{center}
{\large Proposition 16}
\end{center}

There being two circles about the same center, to inscribe an equilateral and even-sided polygon in the
greater circle, not touching the lesser circle.

\epsfysize=2in
\centerline{\epsffile{Book12/fig16e.eps}}

Let $ABCD$ and $EFGH$ be the given two circles, about the same center, $K$. So, it is necessary
to inscribe an equilateral and even-sided polygon in the greater circle $ABCD$, not touching circle $EFGH$.

Let the straight-line $BKD$ have been drawn through the center $K$. And let $GA$ have been drawn, at right-angles to the
straight-line $BD$, through point $G$, and let it have been drawn through to $C$. Thus, $AC$ touches  circle $EFGH$ [Prop.~3.16~corr.]. 
So, (by) cutting circumference $BAD$ in half, and the half of it in half, and 
doing this continually, we will (eventually) leave
a circumference less than $AD$ [Prop.~10.1]. Let it have been left, and let it be $LD$. And let $LM$ have been drawn, from $L$, 
perpendicular to $BD$, and let it have been drawn through to $N$. And let $LD$ and $DN$ have been joined. 
Thus, $LD$ is equal to $DN$ [Props.~3.3, 1.4]. 
And since $LN$ is
parallel to $AC$ [Prop.~1.28], and $AC$ touches circle $EFGH$, $LN$ thus does not touch circle $EFGH$. Thus,
even more so, $LD$ and $DN$ do not touch circle $EFGH$. And if we continuously insert (straight-lines) equal to straight-line
$LD$ into circle $ABCD$ [Prop.~4.1] then an equilateral and even-sided polygon, not touching
the lesser circle $EFGH$, will have been inscribed in circle $ABCD$.$^\dag$ (Which is) the very thing it was required to do.
{\footnotesize\noindent$^\dag$ Note that the chord of the polygon, $LN$, does not touch the inner circle either.}

%%%%
%12.17
%%%%
\pdfbookmark[1]{Proposition 12.17}{pdf12.17}

\begin{center}
{\large Proposition 17}
\end{center}

There being two spheres about the same center, to inscribe  a polyhedral solid in the greater
sphere,  not
touching the lesser sphere on its surface.

\epsfysize=3.4in
\centerline{\epsffile{Book12/fig17e.eps}}

Let two spheres have been conceived about the same center, $A$. So, it is necessary to inscribe a polyhedral solid in the greater
sphere, not touching the lesser sphere on its surface.

Let the spheres have been cut by some plane through the center. So, the sections will be circles, inasmuch as a sphere is generated by  the diameter remaining behind,
and a semi-circle being carried around [Def.~11.14]. And, hence,  whatever position we
conceive  (of for) the semi-circle, the  plane produced through it will  make a circle on the surface of the sphere. And (it is) clear
that (it is) also a great (circle), inasmuch as the diameter of the sphere, which is also manifestly the  diameter of  the semi-circle
and the circle, is greater than  all of the (other)  [straight-lines] drawn across in the circle or the sphere [Prop.~3.15]. Therefore, let $BCDE$ be the circle in the greater sphere, and $FGH$ the circle in the lesser sphere. And let two diameters of them have been drawn
at right-angles to one another, (namely), $BD$ and $CE$. And there being two circles about the same center---(namely), $BCDE$
and $FGH$---let an equilateral and even-sided polygon have been inscribed in the greater circle, $BCDE$, not touching the
lesser circle, $FGH$ [Prop.~12.16], of which let the sides in the quadrant $BE$ be
$BK$, $KL$, $LM$, and $ME$. And, $KA$ being joined, let it have been drawn across to $N$. And let $AO$ have been
set up at point $A$, at right-angles to the plane of circle $BCDE$. And let it meet the surface of the (greater) sphere at $O$. 
And let planes have been produced through  $AO$ and each of $BD$  and $KN$. So, according to the aforementioned (discussion),
they will make great circles on the surface of the (greater) sphere. Let them make (great circles), of which let $BOD$ and $KON$
be semi-circles on the diameters $BD$ and $KN$ (respectively). And since $OA$ is at right-angles
to the plane of circle $BCDE$,  all of the planes through $OA$ are thus also at right-angles to the plane of circle $BCDE$
[Prop.~11.18]. And, hence, the semi-circles $BOD$ and $KON$ are also at right-angles to the
plane of circle $BCDE$. And since semi-circles $BED$, $BOD$, and $KON$ are equal---for (they are) on the
equal diameters $BD$ and $KN$ [Def.~3.1]---the quadrants $BE$, $BO$, and $KO$ are also equal to one another. Thus, as many
sides of the polygon as are in quadrant $BE$, so many are also in quadrants $BO$ and $KO$ equal to the straight-lines
$BK$, $KL$, $LM$, and $ME$. Let them have been inscribed, and let them be $BP$, $PQ$, $QR$, $RO$, $KS$,
$ST$, $TU$, and $UO$. And let $SP$, $TQ$, and $UR$ have been joined. And let perpendiculars have been drawn from
$P$ and $S$ to the plane of circle $BCDE$ [Prop.~11.11]. So, they will fall on the common sections of the planes $BD$ and $KN$ (with $BCDE$),
inasmuch as the planes of $BOD$ and $KON$ are also at right-angles to the plane of circle $BCDE$ [Def.~11.4].
Let them have fallen, and let them be $PV$ and $SW$.  And let $WV$ have been joined.  And since  $BP$ and
$KS$ are equal (circumferences) having been cut off  in the equal semi-circles $BOD$ and $KON$ [Def.~3.28], and  $PV$ and $SW$ are
perpendiculars having been drawn (from them), $PV$ is [thus] equal to $SW$, and $BV$ to $KW$ [Props.~3.27, 1.26]. And the whole of
$BA$ is also equal to the whole of $KA$. And, thus, as $BV$ is to $VA$, so $KW$ (is) to $WA$. 
$WV$ is thus parallel to $KB$ [Prop.~6.2]. And since $PV$ and $SW$ are each at right-angles to the plane
of circle $BCDE$,  $PV$ is thus parallel to $SW$ [Prop.~11.6]. And it was also shown (to be) equal to it.
And, thus, $WV$ and $SP$ are equal and parallel [Prop.~1.33]. And since $WV$ is parallel to $SP$,
but $WV$ is parallel to $KB$, $SP$ is thus also parallel to $KB$ [Prop.~11.1]. And  $BP$ and
$KS$ join them. Thus, the quadrilateral $KBPS$ is in one plane, inasmuch as if there are two parallel
straight-lines, and a random point is taken on each of them, then the straight-line joining the points is in the same plane as
the parallel (straight-lines) [Prop.~11.7]. So, for the same (reasons), each of the quadrilaterals
$SPQT$ and $TQRU$ is also in one plane. And triangle $URO$ is also in one plane [Prop.~11.2].
So, if we conceive straight-lines joining points $P$, $S$, $Q$, $T$, $R$,
and $U$ to $A$ then some solid polyhedral  figure will have been constructed between the circumferences $BO$ and $KO$,
being composed of pyramids whose bases (are) the quadrilaterals $KBPS$, $SPQT$, $TQRU$, and the triangle $URO$, 
and apex the point $A$. And if we also make the same construction on each of the sides $KL$, $LM$, and $ME$, just as on $BK$, 
and, further, (repeat the construction) in the remaining three quadrants, then some polyhedral figure
which has been inscribed in the sphere will have been constructed, being contained by pyramids whose bases (are) the
aforementioned quadrilaterals, and triangle $URO$, and the (quadrilaterals and triangles) similarly arranged
to them, and apex the point $A$.

So, I say that the aforementioned polyhedron  will not touch the lesser sphere on the surface on which  the circle $FGH$ is (situated).

Let the perpendicular (straight-line) $AX$ have been drawn from point $A$ to the plane $KBPS$, and let it meet the
plane at point $X$ [Prop.~11.11]. And let $XB$ and $XK$ have been joined. And since $AX$ is at right-angles to the plane
of quadrilateral $KBPS$, it is thus also at right-angles to all of the  straight-lines joined to it which are also in the plane of the quadrilateral
[Def.~11.3]. Thus, $AX$ is at right-angles to each of $BX$ and $XK$. And since $AB$ is equal
to $AK$, the (square) on $AB$ is also equal to the (square) on $AK$. And the (sum of the
squares) on $AX$ and $XB$ is equal to the (square) on $AB$. For the angle at $X$ (is) a right-angle [Prop.~1.47]. And the (sum of the squares) on $AX$ and $XK$ is equal to the (square) on
$AK$ [Prop.~1.47]. Thus, the (sum of the squares) on $AX$ and $XB$
is equal to the (sum of the squares) on $AX$ and $XK$. Let the (square) on $AX$ have been subtracted from both. Thus, the
remaining (square) on $BX$ is equal to the remaining (square) on $XK$. Thus, $BX$ (is) equal to $XK$. So, similarly,
we can show that  the straight-lines joined from $X$ to $P$ and $S$ are equal to each of $BX$ and $XK$.
Thus, a circle drawn (in the plane of the quadrilateral) with center $X$, and radius one of $XB$ or $XK$, will
also pass through $P$ and $S$, and the quadrilateral $KBPS$ will be inside the circle.

And since $KB$ is greater than $WV$, and $WV$ (is) equal to $SP$, $KB$ (is) thus greater than
$SP$. And $KB$ (is) equal to each of $KS$ and $BP$. Thus, $KS$ and $BP$ are each greater than $SP$. 
And since quadrilateral $KBPS$ is in a circle, and $KB$, $BP$, and $KS$ are equal (to one another), and $PS$ (is) less (than them),
and $BX$ is the radius of the circle, the (square) on $KB$ is thus greater than
double the (square) on $BX$.$^\dag$
Let the perpendicular  $KY$ have been drawn from $K$ to $BV$.$^\ddag$ And since $BD$ is less than double
$DY$, and as $BD$ is to $DY$, so the (rectangle contained) by $DB$ and $BY$ (is) to the (rectangle contained) by $DY$
and $YB$---a square being described on $BY$, and a (rectangular) parallelogram (with short side equal to $BY$) completed on $YD$---the (rectangle contained) by $DB$ and $BY$ is thus also less than double the (rectangle contained) by $DY$ and $YB$.
And, $KD$ being joined, the (rectangle contained) by $DB$ and $BY$ is equal to the (square) on $BK$, and the
(rectangle contained) by $DY$ and $YB$ equal to the (square) on $KY$ [Props.~3.31, 6.8~corr.]. Thus, the (square) on $KB$ is  less than double the (square) on $KY$. But, the (square) on $KB$ is greater than
double the (square) on $BX$. 
Thus, the (square) on $KY$ (is) greater than the (square) on $BX$.  And since $BA$ is equal to $KA$, the (square) on $BA$ is
equal to the (square) on $AK$. And the (sum of the squares) on $BX$ and $XA$ is equal to the (square) on $BA$, and
the (sum of the squares) on $KY$ and $YA$ (is) equal to the (square) on 
$KA$   [Prop.~1.47]. 
Thus, the (sum of the squares) on $BX$ and $XA$ is equal to the (sum of the squares) on $KY$ and $YA$, of
which the (square) on $KY$ (is) greater than the (square) on $BX$. Thus, the remaining (square) on $YA$ is less than the
(square) on $XA$. Thus, $AX$ (is) greater than $AY$. Thus, $AX$ is much greater than $AG$.$^\S$ And $AX$ is (a perpendicular) on one of the bases of the
polyhedron, and $AG$ (is a perpendicular) on the surface of the lesser sphere. Hence, the polyhedron will not touch the lesser sphere on its
surface.

Thus, there being two spheres about the same center, a polyhedral solid has been inscribed in the greater sphere which does not
touch the lesser sphere on its surface. (Which is) the very thing it was required to do.
{\footnotesize\noindent$^\dag$ Since $KB$, $BP$,
and $KS$ are greater than the sides of an inscribed square, which are each of length $\sqrt{2}\,BX$.\\
$^\ddag$ Note that points $Y$ and $V$ are actually identical.\\
$^\S$ This conclusion depends on the fact that the chord of the polygon in proposition 12.16 does not touch the inner circle.}~\\


\begin{center}
{\large Corollary}
\end{center}

And, also, if a similar polyhedral solid  to that in sphere $BCDE$ is inscribed in another sphere then the polyhedral solid in sphere $BCDE$ has
to the polyhedral solid in the other sphere the cubed ratio that the diameter of sphere $BCDE$ has to the diameter of the other
sphere. For if the solids are divided into similarly numbered, and similarly situated, pyramids, then the pyramids will be
similar. And similar pyramids are in the cubed ratio of corresponding sides [Prop.~12.8~corr.]. 
Thus, the pyramid whose base is quadrilateral $KBPS$, and apex the point $A$, will have to the similarly situated pyramid in the other
sphere the cubed ratio that a corresponding side (has) to a corresponding side. That is to say, that of radius $AB$ of the
sphere about center $A$ to the radius of the other sphere. And, similarly, each pyramid in the sphere about center $A$ will have to
each similarly situated pyramid in the other sphere the cubed ratio that $AB$ (has) to the radius of the other sphere. And as
one of the leading (magnitudes is) to one of the following (in two sets of proportional magnitudes), so (the sum of) all the
leading (magnitudes is) to (the sum of) all of the following (magnitudes)  [Prop.~5.12]. 
Hence, the whole polyhedral solid in the sphere about center $A$ will have to the whole polyhedral solid in the other [sphere] the
cubed ratio that (radius) $AB$ (has) to the radius of the other sphere. That is to say, that diameter $BD$ (has) to the diameter of the
other sphere. (Which is) the very thing it was required to show.

%%%%
%12.18
%%%%
\pdfbookmark[1]{Proposition 12.18}{pdf12.18}

\begin{center}
{\large Proposition 18}
\end{center}

Spheres  are to one another in the cubed ratio of their respective diameters.

\epsfysize=2.7in
\centerline{\epsffile{Book12/fig18e.eps}}

Let the spheres $ABC$ and $DEF$ have been conceived, and (let) their diameters (be) $BC$ and $EF$ (respectively). 
I say that sphere $ABC$ has to sphere $DEF$ the cubed ratio that $BC$ (has) to $EF$.

For if sphere $ABC$ does not have to sphere $DEF$ the cubed ratio that $BC$ (has) to $EF$ then  sphere $ABC$ will have
to some (sphere) either less than, or greater than, sphere $DEF$ the cubed ratio that $BC$ (has) to $EF$.
Let it, first of all, have (such a ratio) to a lesser (sphere), $GHK$. And let $DEF$ have been conceived about the same
center as $GHK$. And let a polyhedral solid have been inscribed in the greater sphere $DEF$, not touching the lesser
sphere $GHK$ on its surface [Prop.~12.17]. And let a polyhedral solid, similar to the polyhedral
solid in sphere $DEF$, have also been inscribed in sphere $ABC$. Thus, the polyhedral solid in sphere $ABC$ has to
the polyhedral solid in sphere $DEF$ the cubed ratio that $BC$ (has) to $EF$ [Prop.~12.17~corr.].
And sphere $ABC$ also has to sphere $GHK$ the cubed ratio that $BC$ (has) to $EF$.  Thus, as sphere $ABC$
is to sphere $GHK$, so the polyhedral solid in sphere $ABC$ (is) to the polyhedral solid is sphere
$DEF$. [Thus], alternately, as sphere $ABC$ (is) to the polygon within it, so sphere $GHK$ (is) to the polyhedral
solid within sphere $DEF$ [Prop.~5.16]. And sphere $ABC$ (is) greater than the polyhedron
within it. Thus, sphere $GHK$ (is) also greater than the polyhedron within sphere $DEF$ [Prop.~5.14]. But, (it is) also less. For it is encompassed by it. Thus, sphere $ABC$ does not have to (a sphere) less than sphere
$DEF$ the  cubed ratio that diameter $BC$ (has) to $EF$. So, similarly, we can show that sphere
$DEF$ does not have to  (a sphere) less than sphere $ABC$ the cubed ratio 
that $EF$ (has) to $BC$ either.

So, I say that sphere $ABC$ does not have to some (sphere) greater than sphere $DEF$ the cubed ratio that
$BC$ (has) to $EF$ either.

For, if possible, let it have (the cubed ratio) to a greater (sphere), $LMN$. Thus, inversely, sphere $LMN$
(has) to sphere $ABC$ the cubed ratio that diameter $EF$ (has) to diameter $BC$ [Prop.~5.7~corr.].
And as sphere $LMN$ (is) to sphere $ABC$, so sphere $DEF$ (is) to some (sphere) less than sphere $ABC$, inasmuch
as  $LMN$ is greater than $DEF$, as was shown before [Prop.~12.2~lem.]. And, thus, sphere
$DEF$ has to some (sphere) less than sphere $ABC$ the cubed ratio that $EF$ (has) to $BC$. The very thing
was shown (to be) impossible. Thus, sphere $ABC$ does not have to some (sphere) greater than sphere $DEF$
the cubed ratio that $BC$ (has) to $EF$. And it was shown that neither (does it have such a ratio) to a lesser
(sphere). Thus, sphere $ABC$ has to sphere $DEF$ the cubed ratio that $BC$ (has) to $EF$. (Which is)
the very thing it was required to show.