%%%%%%
% BOOK 13
%%%%%%
\pdfbookmark[0]{Book 13}{book13}
\pagestyle{plain}
\begin{center}
{\Huge ELEMENTS BOOK 13}\\
\spa\spa\spa
{\huge\it The Platonic Solids\symbolfootnote[2]{The five regular solids---the cube, tetrahedron ({\em i.e.}, pyramid), octahedron, icosahedron, and dodecahedron---were problably discovered by the school of Pythagoras. They are generally termed ``Platonic'' solids because they feature prominently in Plato's famous dialogue {\em Timaeus}. Many of the theorems contained in this book---particularly those which pertain to the last two solids---are ascribed to Theaetetus of Athens.}}
\end{center}
\newpage

%%%%
%13.1
%%%%
\pdfbookmark[1]{Proposition 13.1}{pdf13.1}
\pagestyle{fancy}
\cfoot{\gr{\thepage}}
\lhead{\large\gr{STOIQEIWN \ggn{13}.}}
\rhead{\large ELEMENTS BOOK 13}

\begin{center}
{\large Proposition 1}
\end{center}

If a straight-line is cut in extreme and mean ratio then the square on the
greater piece, added to half of the whole, is  five times the square
on the half.

\epsfysize=2.5in
\centerline{\epsffile{Book13/fig01e.eps}}

For let the straight-line $AB$ have been cut in extreme and mean ratio at
point $C$, and let $AC$ be the greater piece. And let the straight-line
$AD$ have been produced in a straight-line with $CA$. And let
$AD$ be made (equal to) half of $AB$. I say that the (square) on
$CD$ is five times the (square) on $DA$. 

For let the squares $AE$ and $DF$ have been described on $AB$ and
$DC$ (respectively). And let the figure in $DF$ have been drawn.  
And let $FC$ have been drawn across to $G$. And since $AB$ has been
cut in extreme and mean ratio at $C$, the (rectangle
contained) by $ABC$ is thus equal to the (square) on $AC$ [Def.~6.3, Prop.~6.17].
And  $CE$ is the (rectangle contained) by $ABC$, and $FH$ the
(square) on $AC$. Thus, $CE$ (is) equal to $FH$. 
And since $BA$ is double $AD$, and $BA$ (is) equal to $KA$,
and $AD$ to $AH$, $KA$ (is) thus also double $AH$. 
And as $KA$ (is) to $AH$, so $CK$ (is) to $CH$ [Prop.~6.1].
Thus, $CK$ (is) double $CH$. And $LH$ plus $HC$ is also double $CH$
[Prop.~1.43].  Thus, $KC$ (is) equal to $LH$ plus $HC$. And
$CE$ was also shown (to be) equal to $HF$. Thus, the whole square
$AE$ is equal to the gnomon $MNO$. And since $BA$ is double
$AD$, the (square) on $BA$ is four times the (square) on $AD$---that
is to say, $AE$ (is four times) $DH$. And $AE$ (is) equal to
gnomon $MNO$. And, thus, gnomon $MNO$ is also four times $AP$. 
Thus, the whole of $DF$ is five times $AP$. And $DF$ is the (square)
on $DC$, and $AP$ the (square) on $DA$. Thus, the
(square) on $CD$ is five times the (square) on $DA$. 

Thus, if a straight-line is cut in extreme and mean ratio then the square on the
greater piece, added to half of the whole, is five times the square
on the half. (Which is) the very thing it was required to show.

%%%%
%13.2
%%%%
\pdfbookmark[1]{Proposition 13.2}{pdf13.2}

\begin{center}
{\large Proposition 2}
\end{center}

If the square on a straight-line is five times the (square) on a piece of
it, and  double the aforementioned piece is cut in extreme and
mean ratio, then the greater piece  is  the remaining part of the
original straight-line. 

\epsfysize=2.5in
\centerline{\epsffile{Book13/fig02e.eps}}

For let the square on the straight-line $AB$ be five times the (square)
on the piece of it, $AC$. And let $CD$ be double $AC$. I say that
if $CD$ is cut in extreme and mean ratio then the greater piece
is $CB$. 

For let the squares $AF$ and $CG$ have been described on each of
$AB$ and $CD$ (respectively). And let the figure  in $AF$ have been drawn.
And let $BE$ have been drawn across. And since the (square) on $BA$
is five times the (square) on $AC$, $AF$ is five times $AH$. 
Thus, gnomon $MNO$ (is) four times $AH$. 
And since
$DC$ is double $CA$, the (square) on $DC$ is thus four times the (square)
on $CA$---that is to say, $CG$ (is four times) $AH$. And the gnomon
$MNO$ was also shown (to be) four times $AH$. Thus, gnomon $MNO$
(is) equal to $CG$. And since $DC$ is double $CA$, and $DC$
(is) equal to $CK$, and $AC$ to $CH$, [$KC$ (is) thus also double $CH$], (and) $KB$ (is) also double $BH$
[Prop.~6.1]. And $LH$ plus $HB$ is also double $HB$ [Prop.~1.43].
Thus, $KB$ (is) equal to $LH$ plus $HB$. And the whole gnomon
$MNO$ was also shown (to be) equal to the whole of $CG$. Thus, the
remainder $HF$ is also equal to (the remainder) $BG$.  And $BG$
is the (rectangle contained) by $CDB$. For  $CD$ (is) equal to $DG$.
And $HF$ (is) the square on $CB$. Thus, the (rectangle contained)
by $CDB$ is equal to the (square) on $CB$. Thus, as $DC$ is to $CB$,
so $CB$ (is) to $BD$ [Prop.~6.17]. And $DC$ (is) greater than $CB$ (see lemma). Thus, $CB$ (is) also greater than $BD$ [Prop.~5.14]. Thus, if the
straight-line $CD$ is cut in extreme and mean ratio then the greater piece is $CB$.

Thus, if the square on a straight-line is five times the (square) on a piece of
itself, and  double the aforementioned piece is cut in extreme and
mean ratio, then the greater piece is the remaining part of the
original straight-line.  (Which is) the very thing it was required to show.


\begin{center}
{\large Lemma}
\end{center}

And it can be shown that double $AC$ ({\em i.e.}, $DC$) is greater than $BC$, as follows.

For if  (double $AC$ is) not (greater than $BC$), if possible, let $BC$ be  double $CA$. Thus, the (square) on $BC$ (is) four times the (square)  on $CA$. Thus, the (sum of) the (squares)
on $BC$ and $CA$ (is) five times the (square)  on $CA$. And the
(square) on $BA$ was assumed (to be) five times the (square) on $CA$. 
Thus, the (square) on $BA$ is equal to the (sum of) the (squares) on $BC$
and $CA$. The very thing (is) impossible [Prop.~2.4]. Thus,
$CB$ is not  double $AC$. So, similarly, we can show that 
a (straight-line) less than $CB$ is not double $AC$ either. 
For (in this case) the absurdity is much [greater].

Thus, double $AC$ is greater than $CB$. (Which is) the very thing it
was required to show.

%%%%
%13.3
%%%%
\pdfbookmark[1]{Proposition 13.3}{pdf13.3}

\begin{center}
{\large Proposition 3}
\end{center}

If a straight-line is cut in extreme and mean ratio then the square
on the lesser piece added to half of the greater piece is five
times the square on half of the greater piece.\\

\epsfysize=2.5in
\centerline{\epsffile{Book13/fig03e.eps}}

For let some straight-line $AB$ have been cut in extreme and mean ratio
at point $C$. And let $AC$ be the greater piece. And let $AC$ have been
cut in half at $D$. I say that the (square) on $BD$ is five times the
(square) on $DC$. 

For let the square $AE$ have been described on $AB$. And let the
figure have been drawn double. Since $AC$ is double $DC$, the
(square) on $AC$ (is) thus four times the (square) on $DC$---that
is to say, $RS$ (is four times) $FG$. And since the (rectangle contained)
by $ABC$ is equal to the (square) on $AC$ [Def.~6.3, Prop.~6.17],
and $CE$ is the (rectangle contained) by $ABC$,  $CE$ is thus equal
to $RS$. And $RS$ (is) four times $FG$. Thus, $CE$ (is) also four times
$FG$. Again, since $AD$ is equal to $DC$, $HK$ is also equal to
$KF$. Hence, square $GF$ is also equal to square $HL$. Thus,
$GK$ (is) equal to $KL$---that is to say, $MN$ to $NE$. 
Hence, $MF$ is also equal to $FE$. But, $MF$ is equal to $CG$.
Thus, $CG$ is also equal to $FE$. Let $CN$ have been added to both.
Thus, gnomon $OPQ$ is equal to $CE$. But, $CE$ was shown (to be)
equal to four times $GF$. Thus, gnomon $OPQ$  is also four times
square $FG$. Thus, gnomon $OPQ$ plus square $FG$ is five times
$FG$. But, gnomon $OPQ$ plus square $FG$ is (square) $DN$. 
And  $DN$ is the (square) on $DB$, and $GF$ the (square)
on $DC$. Thus, the (square) on $DB$ is five times the (square)
on $DC$. (Which is) the very thing it was required to show.

%%%%
%13.4
%%%%
\pdfbookmark[1]{Proposition 13.4}{pdf13.4}

\begin{center}
{\large Proposition 4}
\end{center}

If a straight-line is cut in extreme and mean ratio then the sum of the
squares on the whole and the lesser piece is three times the square on
the greater piece.\\

\epsfysize=2.5in
\centerline{\epsffile{Book13/fig04e.eps}}

Let $AB$ be a straight-line, and let it have been cut in extreme and mean
ratio at $C$, and let $AC$ be the greater piece. I say that the (sum of the
squares) on $AB$ and $BC$ is three times the (square) on $CA$.

For let the square $ADEB$ have been described on $AB$, and
let the (remainder of the) figure have been drawn. Therefore, since
$AB$ has been cut in extreme and mean ratio at $C$,  and $AC$ is the greater piece, the (rectangle contained) by $ABC$ is thus equal to the
(square) on $AC$ [Def.~6.3, Prop.~6.17]. And $AK$ is the (rectangle
contained) by $ABC$, and $HG$ the (square) on $AC$. Thus,
$AK$ is equal to $HG$.  And since $AF$ is equal to $FE$ [Prop.~1.43],
let $CK$ have been added to both. Thus, the whole of $AK$  is equal to
the whole of $CE$. Thus, $AK$ plus $CE$ is double $AK$. But, $AK$
plus $CE$ is the gnomon $LMN$ plus the square $CK$. Thus, 
gnomon $LMN$ plus square $CK$ is double $AK$. But, indeed,
$AK$ was also shown (to be) equal to $HG$. Thus, gnomon $LMN$
plus [square $CK$ is double $HG$. Hence, gnomon $LMN$ plus] the
 squares $CK$ and $HG$ is three times the square $HG$.
And  gnomon $LMN$ plus the  squares  $CK$ and
$HG$ is the whole of $AE$ plus $CK$---which are the 
squares on $AB$ and $BC$ (respectively)---and $GH$ (is) the square on $AC$. 
Thus, the (sum of the) squares on $AB$ and $BC$ is three times the
square on $AC$. (Which is) the very thing it was required to show.

%%%%
%13.5
%%%%
\pdfbookmark[1]{Proposition 13.5}{pdf13.5}

\begin{center}
{\large Proposition 5}
\end{center}

If a straight-line is cut in extreme and mean ratio, and a
(straight-line) equal to the greater piece is added to it, then the
whole straight-line has been cut in extreme and mean ratio, and
 the original straight-line is the greater piece.

\epsfysize=2.in
\centerline{\epsffile{Book13/fig05e.eps}}

For let the straight-line $AB$ have been cut in extreme and mean ratio
at point $C$. And let $AC$ be the greater piece. And let $AD$
be [made] equal to $AC$. I say that the straight-line $DB$ has been cut in extreme and
mean ratio at $A$, and that the original straight-line $AB$ is
the greater piece.

For let the square $AE$ have been described on $AB$, and
let the (remainder of the) figure have been drawn. And since
$AB$ has been cut in extreme and mean ratio at $C$, the (rectangle
contained) by $ABC$ is thus equal to the (square) on $AC$
[Def.~6.3, Prop.~6.17].  And $CE$ is the (rectangle contained)
by  $ABC$,  and $CH$ the (square) on $AC$. But, $HE$ is equal
to $CE$ [Prop.~1.43], and $DH$ equal to $HC$. Thus, $DH$ is also
equal to $HE$. [Let $HB$ have been added to both.] Thus, the whole
of $DK$ is equal to the whole of $AE$. And $DK$ is the
(rectangle contained) by $BD$ and $DA$. For $AD$ (is) equal to
$DL$. And $AE$ (is) the (square) on $AB$.  Thus, the (rectangle
contained) by $BDA$ is equal to the (square) on $AB$. Thus, as
$DB$ (is) to $BA$, so $BA$ (is) to $AD$ [Prop.~6.17]. And
$DB$ (is) greater than $BA$. Thus, $BA$ (is) also greater
than $AD$ [Prop.~5.14].

Thus, $DB$ has been cut in extreme and mean ratio at $A$, and the
greater piece is $AB$. (Which is) the very thing it was required to show.

%%%%
%13.6
%%%%
\pdfbookmark[1]{Proposition 13.6}{pdf13.6}

\begin{center}
{\large Proposition 6}
\end{center}

If a rational straight-line is cut in extreme and mean ratio then each of
the pieces is that irrational (straight-line) called an apotome.

\epsfysize=0.3in
\centerline{\epsffile{Book13/fig06e.eps}}

Let  $AB$ be a rational straight-line cut in extreme and mean ratio
at $C$, and let $AC$ be the greater piece. I say that  $AC$ and
$CB$ is each that irrational (straight-line) called an apotome.

For let $BA$ have been produced, and let $AD$ be made (equal) to
half of $BA$. Therefore, since the straight-line $AB$ has been cut in
extreme and mean ratio at $C$, and $AD$, which is half of $AB$, has been added to the greater piece $AC$, the (square) on $CD$ is thus 
five times the (square) on $DA$ [Prop.~13.1]. Thus, the (square) on $CD$
has to the (square) on $DA$ the ratio which a number (has) to a number.
The (square) on $CD$ (is) thus commensurable with the (square) on
$DA$ [Prop.~10.6]. And the (square) on $DA$ (is) rational.
For $DA$ [is] rational, being half of $AB$, which is rational.
Thus, the (square) on $CD$ (is) also rational [Def.~10.4]. 
Thus, $CD$ is also rational. And since the (square) on $CD$ does not have to
the (square) on $DA$ the ratio which a square number (has) to a square
number, $CD$ (is) thus incommensurable in length with $DA$ [Prop.~10.9].
Thus, $CD$  and $DA$ are rational (straight-lines which are) commensurable in square only. Thus, $AC$ is an apotome [Prop.~10.73].  Again, since
$AB$ has been cut in extreme and mean ratio, and $AC$ is the greater
piece, the (rectangle contained) by $AB$ and $BC$ is thus equal
to the (square) on $AC$ [Def.~6.3, Prop.~6.17]. Thus, the (square) on
the apotome $AC$, applied to the rational (straight-line) $AB$, makes $BC$
as width. And the (square) on an apotome, applied to a rational
(straight-line), makes a first apotome as width [Prop.~10.97].
Thus, $CB$ is a first apotome. And $CA$ was also
shown (to be) an apotome.

Thus, if a rational straight-line is cut in extreme and mean ratio then each of
the pieces is that irrational (straight-line) called an apotome.

%%%%
%13.7
%%%%
\pdfbookmark[1]{Proposition 13.7}{pdf13.7}

\begin{center}
{\large Proposition 7}
\end{center}

If  three angles, either  consecutive or not consecutive, of an equilateral pentagon
are equal then the pentagon will be equiangular.

\epsfysize=2.in
\centerline{\epsffile{Book13/fig07e.eps}}

For let three angles of the equilateral pentagon $ABCDE$---first of
all, the consecutive (angles) at $A$, $B$, and $C$----be equal to
one another. I say that pentagon $ABCDE$ is equiangular.

For let $AC$, $BE$, and $FD$ have been joined. And since the two (straight-lines) $CB$ and $BA$ are equal to the two (straight-lines) $BA$ and
$AE$, respectively, and angle $CBA$ is equal to angle $BAE$,  base
$AC$ is thus equal to base $BE$, and triangle $ABC$  equal to triangle
$ABE$, and the remaining angles will be equal to the remaining angles
which the equal sides subtend [Prop.~1.4],  (that is), $BCA$ (equal) to $BEA$, and $ABE$ to $CAB$. And hence side $AF$ is also equal to side $BF$ [Prop.~1.6]. And the whole  of $AC$ was also shown (to be) equal to
the whole of $BE$. Thus, the remainder $FC$ is also equal to the remainder $FE$.
And $CD$ is also equal to $DE$. So, the two (straight-lines) $FC$ and
$CD$ are equal to the two (straight-lines) $FE$ and $ED$ (respectively). 
And $FD$ is their common base. Thus, angle $FCD$ is equal to angle
$FED$ [Prop.~1.8]. And $BCA$ was also shown (to be) equal to 
$AEB$. And thus the whole of  $BCD$ (is) equal to the whole of $AED$.
But, (angle) $BCD$ was assumed (to be) equal to the angles at $A$ and
$B$. Thus, (angle) $AED$ is also equal to the angles at $A$ and $B$. So, 
similarly, we can show that angle $CDE$ is also equal to the angles
at $A$, $B$, $C$. Thus, pentagon $ABCDE$ is equiangular.

And so let consecutive  angles not be equal, but let
the (angles) at points $A$, $C$, and $D$ be equal. I say that pentagon $ABCDE$ is also
equiangular in this case.

For let $BD$ have been joined. And since the two (straight-lines) $BA$
and $AE$ are equal to the (straight-lines) $BC$ and $CD$, and they contain
equal angles,   base $BE$ is thus equal to base $BD$, and triangle $ABE$
is equal to triangle $BCD$, and the remaining angles will be equal
to the remaining angles which the equal sides subtend [Prop.~1.4]. 
Thus, angle $AEB$ is equal to (angle) $CDB$. And angle $BED$
is also equal to (angle) $BDE$,  since side $BE$ is also equal to side
$BD$ [Prop.~1.5]. Thus, the whole angle $AED$ is also equal to the
whole (angle) $CDE$. But, (angle) $CDE$ was assumed (to be) equal
to the angles at $A$ and $C$.  Thus, angle $AED$ is also equal to the (angles)
at $A$ and $C$. So, for the same (reasons), (angle) $ABC$ is also equal
to the angles at  $A$, $C$, and $D$. Thus, pentagon $ABCDE$ is equiangular.
(Which is) the very thing it was required to show.

%%%%
%13.8
%%%%
\pdfbookmark[1]{Proposition 13.8}{pdf13.8}

\begin{center}
{\large Proposition 8}
\end{center}

If straight-lines subtend two consecutive angles of an equilateral and equiangular pentagon then they cut one another in extreme and
mean ratio, and their greater pieces are equal to  the sides of the pentagon.

\epsfysize=2.in
\centerline{\epsffile{Book13/fig08e.eps}}

For let the two straight-lines, $AC$ and $BE$, cutting one another
at point $H$, have subtended  two consecutive angles, at $A$ and $B$ (respectively), of
the equilateral and equiangular pentagon $ABCDE$. I say that
each of them has been cut in extreme and mean ratio at point $H$, and
that their greater pieces are equal to the sides of the pentagon.

For let the circle $ABCDE$ have been circumscribed about pentagon
$ABCDE$ [Prop.~4.14]. And since the two straight-lines $EA$ and
$AB$ are equal to the two (straight-lines) $AB$ and $BC$ (respectively),
and they contain equal angles, the base $BE$ is thus equal to the base
$AC$, and triangle $ABE$ is equal to triangle $ABC$, and the remaining
angles will be equal to the remaining angles, respectively,
which the equal sides subtend [Prop.~1.4]. Thus, angle $BAC$ is equal
to (angle) $ABE$. Thus, (angle) $AHE$ (is) double (angle) $BAH$ [Prop.~1.32]. And $EAC$ is also double $BAC$, inasmuch as
circumference $EDC$ is also double circumference $CB$ [Props.~3.28, 6.33].
Thus, angle $HAE$ (is) equal to (angle) $AHE$. Hence, straight-line
$HE$ is also equal to (straight-line) $EA$---that is to say, to (straight-line)
$AB$ [Prop.~1.6]. And since straight-line $BA$ is equal to $AE$,
angle $ABE$ is also equal to $AEB$ [Prop.~1.5]. But, $ABE$
was shown (to be) equal to $BAH$. Thus, $BEA$ is also equal to
$BAH$. And (angle) $ABE$ is common to the two triangles
$ABE$ and $ABH$. Thus, the remaining angle $BAE$ is equal to
the remaining (angle) $AHB$ [Prop.~1.32]. Thus, triangle $ABE$
is equiangular to triangle $ABH$. Thus, proportionally, as $EB$ is
to $BA$, so $AB$ (is) to $BH$ [Prop.~6.4].  And $BA$ (is) equal to
$EH$. Thus, as $BE$ (is) to $EH$, so $EH$ (is) to $HB$. And
$BE$ (is) greater than $EH$.  $EH$ (is) thus also greater than $HB$
[Prop.~5.14]. Thus, $BE$ has been cut in extreme and mean ratio at
$H$, and the greater piece $HE$ is equal to the side of the pentagon. 
So, similarly, we can show that $AC$ has also been cut in extreme and
mean ratio at $H$, and that  its greater piece $CH$ is equal to the side
of the pentagon. (Which is) the very thing it was required to show.

%%%%
%13.9
%%%%
\pdfbookmark[1]{Proposition 13.9}{pdf13.9}

\begin{center}
{\large Proposition 9}
\end{center}

If the side of a hexagon and  of a decagon inscribed in the same circle
are added together then the whole straight-line has been cut in extreme and
mean ratio (at the junction point), and its greater piece is the side of the hexagon.$^\dag$

\epsfysize=2.25in
\centerline{\epsffile{Book13/fig09e.eps}}

Let $ABC$ be a circle. And of the figures inscribed in circle $ABC$, 
let $BC$ be the side of a decagon, and $CD$ (the side) of a hexagon.
And let them be (laid down) straight-on (to one another). I say that
the whole straight-line $BD$ has been cut in extreme and mean
ratio (at $C$), and that   $CD$ is its greater piece. 

For let the center of the circle, point $E$, have been found [Prop.~3.1],
and let $EB$, $EC$, and $ED$ have been joined, and let $BE$ have been
drawn across to $A$. Since $BC$ is a side on an equilateral decagon,
circumference $ACB$ (is) thus five times circumference $BC$. Thus,
circumference $AC$ (is) four times $CB$. And as circumference $AC$
(is) to $CB$, so angle $AEC$ (is) to $CEB$ [Prop.~6.33]. Thus,
(angle) $AEC$ (is) four times $CEB$. And since angle $EBC$
(is) equal to $ECB$ [Prop.~1.5], angle $AEC$ is thus double $ECB$ [Prop.~1.32]. 
And since straight-line $EC$ is equal to $CD$---for each of them is equal
to the side of the hexagon [inscribed] in circle $ABC$ [Prop.~4.15~corr.]---
angle $CED$ is also equal to angle $CDE$ [Prop.~1.5]. Thus,
angle $ECB$ (is) double $EDC$ [Prop.~1.32]. But,
$AEC$ was shown (to be) double $ECB$. Thus, $AEC$ (is) four times
$EDC$. And $AEC$ was also shown (to be) four times $BEC$. 
Thus, $EDC$ (is) equal to $BEC$. And angle $EBD$ (is) common
to the two triangles $BEC$ and $BED$. Thus, the remaining (angle)
$BED$ is equal to the (remaining angle) $ECB$ [Prop.~1.32]. 
Thus, triangle $EBD$ is equiangular to triangle $EBC$. Thus,
proportionally, as $DB$ is to $BE$, so $EB$ (is) to $BC$ [Prop.~6.4].
And $EB$ (is) equal to $CD$. Thus, as $BD$ is to $DC$, so $DC$
(is) to $CB$. And $BD$ (is) greater than $DC$. Thus,
$DC$ (is) also greater than $CB$ [Prop.~5.14]. Thus, the
straight-line $BD$ has been cut in extreme and mean ratio [at $C$],
and $DC$  is its greater piece. (Which is), the very thing it was required to show.{\footnotesize\noindent$^\dag$ If the circle is of unit radius
then the side of the hexagon is 1, whereas the side of the decagon is $(1/2)\,(\sqrt{5}-1)$.}

%%%%
%13.10
%%%%
\pdfbookmark[1]{Proposition 13.10}{pdf13.10}

\begin{center}
{\large Proposition 10}
\end{center}

If an equilateral pentagon is inscribed in a circle then the square on the
side of the pentagon is (equal to) the (sum of the squares) on the (sides) of
the hexagon and of the decagon inscribed in the same circle.$^\dag$

\epsfysize=2.5in
\centerline{\epsffile{Book13/fig10e.eps}}

Let $ABCDE$ be a circle. And let the equilateral pentagon $ABCDE$
have been inscribed in circle $ABCDE$. I say that the square on the
side of pentagon $ABCDE$ is the (sum of the squares) on the sides
of the hexagon and of the decagon inscribed in circle $ABCDE$. 

For let the center of the circle, point $F$, have been found [Prop.~3.1]. And,
$AF$ being joined, let it have been drawn across to point $G$. And
let $FB$ have been joined. And let $FH$ have been drawn from $F$
perpendicular to $AB$. And let it have been drawn across to $K$. 
And let $AK$ and $KB$ have been joined.  And, again, let $FL$ have
been drawn from $F$ perpendicular to $AK$. And let it have been drawn
across to $M$. And let $KN$ have been joined.

Since circumference $ABCG$ is equal to circumference $AEDG$,
of which $ABC$ is equal to $AED$, the remaining circumference
$CG$ is thus equal to the remaining (circumference) $GD$. And $CD$ (is the side) of the pentagon. 
$CG$ (is) thus (the side) of the decagon. And since  $FA$ is equal to $FB$,
and $FH$ is perpendicular (to $AB$), angle $AFK$ (is) thus also equal
to $KFB$ [Props.~1.5, 1.26]. Hence, circumference $AK$
is also equal to $KB$ [Prop.~3.26]. Thus, circumference $AB$ (is)
double circumference $BK$. Thus, straight-line $AK$ is the side of the decagon. 
So, for the same (reasons, circumference)  $AK$ is also double $KM$. And since
circumference $AB$ is double circumference $BK$, and
circumference $CD$ (is) equal to circumference $AB$, circumference
$CD$ (is) thus also double circumference $BK$. And
circumference $CD$ is also double $CG$. Thus, circumference
$CG$ (is) equal to circumference $BK$. But, $BK$
is double $KM$, since $KA$ (is) also (double $KM$). Thus,
(circumference) $CG$ is also double $KM$. But, indeed, circumference $CB$ is also
double circumference $BK$. For circumference $CB$ (is) equal to $BA$. 
Thus, the whole circumference $GB$ is also double $BM$. 
Hence, angle $GFB$ [is] also double  angle $BFM$ [Prop.~6.33].
And $GFB$ (is) also double $FAB$. For $FAB$ (is) equal to $ABF$. 
Thus, $BFN$ is also equal to $FAB$. And angle $ABF$ (is) common
to the two triangles $ABF$ and $BFN$. Thus, the
remaining (angle) $AFB$ is equal to the remaining (angle) $BNF$
[Prop.~1.32]. Thus, triangle $ABF$ is equiangular to triangle
$BFN$. Thus, proportionally, as straight-line $AB$ (is) to
$BF$, so $FB$ (is) to $BN$ [Prop.~6.4]. Thus, the (rectangle contained)
by $ABN$ is equal to the (square) on $BF$ [Prop.~6.17]. Again,
since $AL$ is equal to $LK$, and $LN$ is common and at right-angles (to $KA$), 
base $KN$ is thus equal to base $AN$  [Prop.~1.4]. And, thus, angle
$LKN$ is equal to angle $LAN$. But, $LAN$ is equal to
$KBN$ [Props.~3.29, 1.5]. Thus, $LKN$ is also equal to $KBN$. 
And the (angle) at $A$ (is) common to the two triangles $AKB$ and
$AKN$. Thus, the remaining (angle) $AKB$ is equal to the
remaining (angle) $KNA$ [Prop.~1.32]. Thus, triangle $KBA$ is
equiangular to triangle $KNA$. Thus, proportionally, as straight-line
$BA$ is to $AK$, so $KA$ (is) to $AN$ [Prop.~6.4]. Thus, the
(rectangle contained) by $BAN$ is equal to the (square) on $AK$ [Prop.~6.17]. 
And  the (rectangle contained) by $ABN$ was also shown (to be)
equal to the (square) on $BF$. Thus, the (rectangle contained) by $ABN$
plus the (rectangle contained) by $BAN$, which is the (square) on $BA$
[Prop.~2.2],
is equal to the (square) on $BF$ plus the (square) on $AK$. 
And $BA$ is the side of the pentagon, and $BF$ (the side) of the hexagon
[Prop.~4.15~corr.], and $AK$ (the side) of the decagon.

Thus, the square on the side of the pentagon  (inscribed in a circle) is (equal to) the (sum of the squares)
on the (sides) of the hexagon and of  the decagon inscribed in the
same circle.
{\footnotesize\noindent$^\dag$ If the circle is of unit radius then the side
of the pentagon is $(1/2)\,\sqrt{10-2\,\sqrt{5}}$.}

%%%%
%13.11
%%%%
\pdfbookmark[1]{Proposition 13.11}{pdf13.11}

\begin{center}
{\large Proposition 11}
\end{center}

If an equilateral pentagon is inscribed in a circle which has a rational  diameter  then the side of the pentagon is that irrational
(straight-line) called minor.

\epsfysize=2.5in
\centerline{\epsffile{Book13/fig11e.eps}}

For let the equilateral pentagon $ABCDE$ have been inscribed  in the circle $ABCDE$ which has a rational diameter. I say that the side of pentagon
[$ABCDE$] is that irrational (straight-line) called minor.

For let the center of the circle, point $F$, have been found [Prop.~3.1].
And let $AF$ and $FB$ have been joined. And let them have been drawn
across to points $G$ and $H$ (respectively). And let $AC$ have been joined.
And let $FK$ made (equal) to the fourth part of $AF$. And $AF$ (is) rational. $FK$ (is) thus also rational. And $BF$ is also rational. Thus,
the whole of $BK$ is rational. And since circumference $ACG$ is equal
to circumference $ADG$, of which $ABC$ is equal to $AED$, the
remainder $CG$ is thus equal to the remainder $GD$.
And if we join $AD$ then the angles at $L$ are inferred (to be) right-angles, 
and $CD$ (is inferred to be) double $CL$ [Prop.~1.4].  So, for the same (reasons), 
the (angles) at $M$ are also right-angles, and $AC$ (is) double $CM$. Therefore,
since angle $ALC$ (is) equal to $AMF$, and (angle) $LAC$ (is) common
to the two triangles $ACL$ and $AMF$, the remaining (angle) $ACL$ is thus
equal to the remaining (angle) $MFA$ [Prop.~1.32]. Thus, triangle 
$ACL$  is equiangular to triangle $AMF$. Thus, proportionally, 
as $LC$ (is) to $CA$, so $MF$ (is) to $FA$ [Prop.~6.4]. And (we can take) the
doubles of the leading (magnitudes). Thus, as double $LC$ (is) to $CA$,
so double $MF$ (is) to $FA$. And as double $MF$ (is) to $FA$, so
$MF$ (is) to half of $FA$. And, thus, as double $LC$ (is) to $CA$,
so $MF$ (is) to half of $FA$. And (we can take) the halves of the following
(magnitudes). Thus, as double $LC$ (is) to half of $CA$, so $MF$
(is) to the fourth of $FA$. And $DC$ is double $LC$, and  $CM$ half
of $CA$, and $FK$ the fourth part of $FA$.  Thus, as $DC$
is to $CM$, so $MF$ (is) to $FK$. Via composition, as the
sum of $DCM$ ({\em i.e.}, $DC$ and $CM$)
(is) to $CM$, so $MK$ (is) to $KF$ [Prop.~5.18]. And, thus, as the (square) on the
sum of $DCM$ (is) to the (square) on $CM$, so the (square) on $MK$
(is) to the (square) on $KF$. And since the
greater piece of a (straight-line) subtending two sides of a
pentagon, such as $AC$, (which is) cut in extreme and mean ratio is equal to the side of the pentagon [Prop.~13.8]---that is to say,
to $DC$----and the square on the greater piece added to half of the
whole is five times the (square) on half of the whole [Prop.~13.1], 
and $CM$ (is) half of the whole, $AC$, thus the (square) on
$DCM$, (taken) as one, is five times the (square) on $CM$.
And the (square) on $DCM$, (taken) as one,  (is)  to the (square) on $CM$, so the (square) on $MK$ was shown (to be) to the (square) on 
$KF$.  Thus, the (square) on $MK$ (is) five times the (square) on 
$KF$. And the square on $KF$ (is) rational. For the diameter
(is) rational. Thus, the (square) on $MK$ (is) also rational. 
Thus, $MK$ is rational [in square only]. And since $BF$ is four times
$FK$, $BK$ is thus five times $KF$. Thus, the (square) on $BK$
(is) twenty-five times the (square) on $KF$.  And the (square) on $MK$
(is) five times the square on $KF$. Thus, the (square) on $BK$ (is) five
times the (square) on $KM$.  Thus, the (square) on $BK$ does not
have to the (square) on $KM$ the ratio which a square number (has) to
a square number. Thus, $BK$ is incommensurable in length with $KM$ [Prop.~10.9].
And each of them is a rational (straight-line).  Thus, $BK$ and $KM$ are rational
(straight-lines which are) commensurable in square only. And if from a rational (straight-line)
a rational (straight-line) is subtracted, which is commensurable in square
only with the whole, then the remainder is that irrational (straight-line
called) an apotome [Prop.~10.73]. Thus, $MB$ is an apotome, and $MK$ its attachment. So, I say that (it is) also a fourth (apotome). So, let the
(square) on $N$ be (made) equal to that (magnitude) by which the (square)
on $BK$ is greater than the (square) on $KM$. Thus, the
square on $BK$ is greater than the (square) on $KM$ by the (square) on $N$.
And since $KF$ is commensurable (in length) with $FB$ then, via composition, 
$KB$ is also commensurable (in length) with $FB$ [Prop.~10.15]. But, $BF$ is commensurable (in length)
with $BH$. Thus, $BK$ is also commensurable (in length) with $BH$ [Prop.~10.12]. And since the
(square) on $BK$ is five times the (square) on $KM$, the (square) on $BK$
thus has to the (square) on $KM$ the ratio which 5 (has) to one. Thus, via
conversion, the (square) on $BK$ has to the (square) on $N$ the ratio
which 5 (has) to 4 [Prop.~5.19~corr.], which is not (that) of a
square (number) to a square (number). $BK$ is thus incommensurable (in length)
with $N$ [Prop.~10.9]. Thus, the square on $BK$ is greater than
the (square) on $KM$ by the (square) on (some straight-line which is) incommensurable (in length) with ($BK$). Therefore, since the square on the whole,
$BK$, is greater than the (square) on the attachment, $KM$, by the
(square) on (some straight-line which is) incommensurable (in length) with ($BK$),
and the whole, $BK$, is commensurable (in length) with the (previously) laid down
rational (straight-line) $BH$, $MB$ is thus a fourth apotome
[Def.~10.14]. And the rectangle contained by a rational (straight-line)
and a fourth apotome is irrational, and its square-root is that irrational
(straight-line) called minor [Prop.~10.94]. And the square on $AB$
is the rectangle contained by $HBM$, on account of joining $AH$, (so that)
triangle $ABH$ becomes equiangular  with triangle $ABM$ [Prop.~6.8], and (proportionally) as
$HB$ is to $BA$, so $AB$ (is) to $BM$.

Thus, the side $AB$ of the pentagon is that irrational (straight-line)
called minor.$^\dag$ (Which is) the very thing it was required to show.
{\footnotesize\noindent$^\dag$ If the circle has unit radius then the side of the pentagon is $(1/2)\,\sqrt{10-2\,\sqrt{5}}$. However, this length can
be written in the ``minor'' form (see Prop.~10.94) $(\rho/\sqrt{2})\,\sqrt{1+k/\sqrt{1+k^2}} - (\rho/\sqrt{2})\,\sqrt{1-k/\sqrt{1+k^2}}$, with $\rho=\sqrt{5/2}$ and $k=2$.}

%%%%
%13.12
%%%%
\pdfbookmark[1]{Proposition 13.12}{pdf13.12}

\begin{center}
{\large Proposition 12}
\end{center}

If an equilateral triangle is inscribed in a circle then the
square on the side of the triangle is three times the (square) on
the radius of the circle.

Let there be a circle $ABC$, and let the equilateral triangle $ABC$ have
been inscribed in it  [Prop.~4.2]. I say that the square on one side of triangle
$ABC$ is three times the (square) on the radius of circle $ABC$. 

\epsfysize=2.5in
\centerline{\epsffile{Book13/fig12e.eps}}

For let the center, $D$, of circle $ABC$ have been found [Prop.~3.1].
And  $AD$ (being) joined, let it have been drawn across to $E$. 
And let $BE$ have been joined.

And since triangle $ABC$ is equilateral, circumference $BEC$ is thus the
third part of the circumference of circle $ABC$. Thus, 
circumference $BE$ is the sixth part of the circumference of the circle. 
Thus, straight-line $BE$ is (the side) of a hexagon. Thus, it is equal to the radius
$DE$ [Prop.~4.15~corr.]. And since $AE$ is double $DE$, the (square)
on $AE$ is four times the (square) on $ED$---that is to say, of the
(square) on $BE$.  And the (square) on $AE$ (is) equal to
the (sum of the squares) on $AB$ and $BE$ [Props.~3.31, 1.47]. 
Thus, the (sum of the squares) on $AB$ and $BE$ is four times the
(square) on $BE$. Thus, via separation, the (square) on $AB$ is three
times the (square) on $BE$. And $BE$ (is) equal to $DE$. Thus, the
(square) on $AB$ is three times the (square) on $DE$.

Thus, the square on the side of the triangle is three times the (square)
on the radius [of the circle]. (Which is) the very thing it was required to show.

%%%%
%13.13
%%%%
\pdfbookmark[1]{Proposition 13.13}{pdf13.13}

\begin{center}
{\large Proposition 13}
\end{center}

To construct a (regular) pyramid ({\em i.e.}, a tetrahedron), and to
enclose (it) in a given sphere, and to show that the square on the
diameter of the sphere is one and a half times the (square) on the side 
of the pyramid. 

\epsfysize=3.5in
\centerline{\epsffile{Book13/fig13e.eps}}

Let the diameter $AB$ of the given sphere be laid out, and let it have been
cut at point $C$ such that $AC$ is double $CB$ [Prop.~6.10]. And let the semi-circle 
$ADB$  have been drawn on $AB$. And let $CD$ have been drawn from point
$C$ at right-angles to $AB$. And let $DA$ have been joined. 
And let the circle $EFG$ be laid down having a radius equal to $DC$,
and let the equilateral triangle $EFG$ have been inscribed in circle
$EFG$ [Prop.~4.2]. And let the center of the circle, point $H$,
have been found [Prop.~3.1]. And let $EH$, $HF$, and $HG$ have been
joined. And let $HK$ have been set up, at point $H$, at right-angles to the
plane of circle $EFG$ [Prop.~11.12]. And let $HK$, equal to the straight-line $AC$, have been cut off from $HK$. And let $KE$, $KF$, and $KG$ have been joined.
And since $KH$ is at right-angles to the plane of circle $EFG$, it will
thus also make right-angles with all of the straight-lines joining it
(which are) also in the plane of circle $EFG$ [Def.~11.3]. And $HE$, $HF$,
and $HG$ each join it. Thus, $HK$ is at right-angles to each of $HE$,
$HF$, and $HG$. And since  $AC$ is equal to $HK$, and $CD$ to
$HE$, and they contain right-angles, the base $DA$ is thus equal to the base 
$KE$ [Prop.~1.4]. So, for the same (reasons),  $KF$ and
$KG$ is each equal to $DA$. Thus, the three (straight-lines)
$KE$, $KF$, and $KG$ are equal to one another. And since 
$AC$ is double $CB$, $AB$ (is) thus triple $BC$. And as 
$AB$ (is) to $BC$, so the (square) on $AD$ (is) to the (square)
on $DC$, as will be shown later [see lemma]. Thus, the (square)
on $AD$ (is) three times the (square) on $DC$. And the (square)
on $FE$ is also three times the (square) on $EH$ [Prop.~13.12], 
and $DC$ is equal to $EH$. Thus, $DA$ (is) also equal to $EF$.
But, $DA$ was shown (to be) equal to each of $KE$, $KF$, and
$KG$. Thus, $EF$, $FG$, and $GE$ are equal to 
$KE$, $KF$, and $KG$, respectively. Thus, the four triangles $EFG$, $KEF$,
$KFG$, and $KEG$ are equilateral. Thus, a pyramid, whose base is triangle
$EFG$, and apex the point $K$,   has been
constructed from four equilateral triangles.

So, it is also necessary to enclose it in the given sphere, and to show that
the square on the diameter of the sphere is one and a half times the (square)
on the side of the pyramid. 

For let the straight-line $HL$ have been produced in a straight-line with $KH$, and let
$HL$ be made equal to $CB$. And since as $AC$ (is) to $CD$, 
so $CD$ (is) to $CB$ [Prop.~6.8~corr.], and $AC$ (is) equal to $KH$, and $CD$ to
$HE$, and $CB$ to $HL$, thus as $KH$ is to $HE$, so $EH$ (is) to $HL$.
Thus,  the (rectangle contained) by $KH$ and $HL$ is equal to the
(square) on $EH$ [Prop.~6.17]. And each of the angles $KHE$ and
$EHL$ is a right-angle. Thus, the semi-circle drawn on $KL$ 
will also pass through $E$ [inasmuch as if we join $EL$ then the angle
$LEK$ becomes a right-angle, on account of triangle $ELK$
becoming equiangular to each of the triangles $ELH$ and $EHK$ [Props.~6.8, 3.31]\,]. So, if
$KL$ remains (fixed), and the semi-circle is carried around, and again
established at the same (position) from which it began to be moved, 
it will also pass through points $F$ and $G$, (because) if  $FL$ and
$LG$ are joined, the angles at $F$ and $G$ will similarly become  right-angles. And the pyramid will have been enclosed by the given sphere.
For the diameter, $KL$,  of the sphere is equal to the diameter, $AB$,
of the given sphere---inasmuch as $KH$ was made equal to $AC$, 
and $HL$ to $CB$. 

So, I say that the square on the diameter of the sphere is
one and a half times the (square) on the side of the pyramid.

For since $AC$ is double $CB$, $AB$ is thus triple $BC$. Thus,
via conversion, $BA$ is one and a half times $AC$. And as
$BA$ (is) to $AC$, so the (square) on $BA$ (is) to the (square) on $AD$
[inasmuch as if $DB$ is joined then as $BA$ is to $AD$, so
$DA$ (is) to $AC$, on account of the similarity of triangles
$DAB$ and $DAC$. And as the first is to the third (of four
proportional magnitudes), so the (square) on the first (is) to the
(square) on the second.] Thus, the (square) on $BA$
(is) also one and a half times the (square) on $AD$. 
And $BA$ is the diameter of the given sphere, and $AD$
(is) equal to the side of the pyramid.

Thus, the square on the diameter of the sphere is one and a half
times the (square) on the side of the pyramid.$^\dag$ (Which is) the very thing
it was required to show.
{\footnotesize\noindent$^\dag$ If the radius of the sphere is unity then the side of the pyramid ({\em i.e.}, tetrahedron) is $\sqrt{8/3}$.}


\epsfysize=2.5in
\centerline{\epsffile{Book13/fig13ae.eps}}

\begin{center}
{\large Lemma}
\end{center}

It must be shown that as $AB$ is to $BC$, so the (square) on
$AD$ (is) to the (square) on $DC$.

For, let the figure of the semi-circle have been set out, and let $DB$ have
been joined. And let the square $EC$ have been described on
$AC$. And let the parallelogram $FB$ have been completed. 
Therefore, since, on account of triangle $DAB$ being
equiangular to  triangle $DAC$ [Props.~6.8, 6.4], (proportionally) as $BA$
is to $AD$, so $DA$ (is) to $AC$,  the (rectangle contained)
by $BA$ and $AC$  is thus equal to the (square) on $AD$ [Prop.~6.17]. 
And since as $AB$ is to $BC$, so $EB$ (is) to $BF$ [Prop.~6.1].
And $EB$ is  the (rectangle contained) by $BA$ and $AC$---for $EA$ (is) equal to $AC$. And $BF$  the (rectangle
contained) by $AC$ and $CB$. Thus, as $AB$ (is) to $BC$,
so the (rectangle contained) by $BA$ and $AC$ (is) to the
(rectangle contained) by $AC$ and $CB$. And the (rectangle contained)
by $BA$ and $AC$ is equal to the (square) on $AD$, and the
(rectangle contained) by $ACB$ (is) equal to the (square) on 
$DC$. For the perpendicular $DC$ is the mean proportional
to the pieces  of the base, $AC$ and $CB$, on account of
$ADB$ being a right-angle [Prop.~6.8~corr.]. Thus, as
$AB$ (is) to $BC$, so the (square) on $AD$ (is) to the (square)
on $DC$. (Which is) the very thing it was required to show.

%%%%
%13.14
%%%%
\pdfbookmark[1]{Proposition 13.14}{pdf13.14}

\begin{center}
{\large Proposition 14}
\end{center}

To construct an octahedron, and to enclose (it) in a (given) sphere, like in the preceding (proposition), and to show that the square
on the diameter of the sphere is double the (square) on the
side of the octahedron.

Let the diameter $AB$ of the given sphere be laid out, and let it have
been cut in half at $C$. And let the semi-circle $ADB$ have been
drawn on $AB$. And let $CD$ be drawn from $C$ at right-angles to
$AB$. And let $DB$ have been joined. And let the square $EFGH$,
having each of its sides equal to $DB$, be laid out. And let $HF$
and $EG$ have been joined. And let the straight-line $KL$
have been set up, at point $K$, at right-angles to the plane of square 
$EFGH$ [Prop.~11.12]. And let it have been drawn across  on the other side of
the plane, like $KM$. And let $KL$ and $KM$, equal to
one of $EK$, $FK$, $GK$, and $HK$,  have been cut off from 
$KL$ and $KM$, respectively. And let $LE$, $LF$, $LG$, $LH$, $ME$, $MF$,
$MG$, and $MH$ have been joined.  

\epsfysize=3.25in
\centerline{\epsffile{Book13/fig14e.eps}}

And since  $KE$ is equal to $KH$, and angle $EKH$ is a right-angle, 
the (square) on the  $HE$ is thus double the (square) on $EK$ [Prop.~1.47].
Again, since $LK$ is equal to $KE$, and angle $LKE$
is a right-angle, the (square) on $EL$ is thus double the (square) on
$EK$ [Prop.~1.47]. And the (square) on $HE$ was also shown (to be)
double the (square) on $EK$.  Thus, the (square) on $LE$
is equal to the (square) on $EH$. Thus, $LE$ is equal to $EH$. So,
for the same (reasons), $LH$ is also equal to $HE$. Triangle
$LEH$ is thus equilateral. So, similarly, we can show that each of
the remaining triangles, whose bases are the sides of the
square $EFGH$, and apexes the points $L$ and $M$, are equilateral.
Thus, an octahedron contained by eight
equilateral triangles has been constructed.

So, it is also necessary to enclose it by the given sphere, and to show that the
square on the diameter of the sphere is double the (square) on the
side of the octahedron.

For since the three (straight-lines) $LK$, $KM$, and $KE$ are equal
to one another, the semi-circle drawn on $LM$ will thus also pass
through $E$. And, for the same (reasons), if $LM$ remains (fixed), and the semi-circle is carried around, and again established at the same
(position) from which it began to be moved, then it will also pass
through points $F$, $G$, and $H$, and the octahedron will have been
enclosed by a sphere. So, I say that (it is) also (enclosed)
by the given (sphere). For since $LK$ is equal to $KM$, and $KE$
(is) common, and they contain right-angles, the base
$LE$ is thus equal to the base $EM$ [Prop.~1.4]. And since angle $LEM$
is a right-angle---for (it is) in a semi-circle [Prop.~3.31]---the
(square) on $LM$ is thus double the (square) on $LE$ [Prop.~1.47]. 
Again, since $AC$ is equal to $CB$, $AB$ is double $BC$. 
And as $AB$ (is) to $BC$, so the (square) on $AB$ (is) to the
(square) on $BD$ [Prop.~6.8, Def.~5.9]. Thus, the (square) on 
$AB$ is double the (square) on $BD$. And the (square) on $LM$
was also shown (to be) double the (square) on $LE$. And the (square)
on $DB$ is equal to the (square) on $LE$. For $EH$ was made equal
to $DB$. Thus, the (square) on $AB$ (is) also equal to the (square) on 
$LM$. Thus, $AB$ (is) equal to $LM$. And $AB$ is the diameter of the
given sphere. Thus, $LM$ is equal to the diameter of the given sphere.

Thus, the octahedron has been enclosed by the given sphere,  and it has
been simultaneously  proved that the square on the diameter of the sphere
is double the (square) on the side of the octahedron.$^\dag$ (Which is) the
very thing it was required to show.
{\footnotesize\noindent$^\dag$ If the radius of the sphere is unity then the
side of octahedron is $\sqrt{2}$.}

%%%%
%13.15
%%%%
\pdfbookmark[1]{Proposition 13.15}{pdf13.15}

\begin{center}
{\large Proposition 15}
\end{center}

To construct a cube, and to enclose (it) in a sphere, like in the (case of the)
pyramid, and to show that the square on the diameter of the sphere
is three times the (square) on the side of the cube.

Let the diameter $AB$ of the given sphere be laid out, and let it have been
cut at $C$ such that $AC$ is double $CB$. And let the semi-circle
$ADB$ have been drawn on $AB$. And let $CD$ have been drawn from $C$
at right-angles to $AB$. And let $DB$ have been joined. And let the
square $EFGH$, having (its) side equal to $DB$, be laid out. And
let $EK$, $FL$, $GM$, and $HN$ have been drawn from (points)
$E$, $F$, $G$, and $H$, (respectively), at right-angles to the plane of
square $EFGH$. And let $EK$, $FL$, $GM$, and $HN$, equal
to one of $EF$, $FG$, $GH$, and $HE$, have been cut off from
$EK$, $FL$, $GM$, and $HN$, respectively. And let $KL$,
$LM$, $MN$, and $NK$ have been joined. Thus, a cube contained
by six equal squares has been constructed. 

So, it is also necessary
to enclose it by the given sphere, and to show that the square on
the diameter of the sphere is three times the (square) on the side of the cube.

\epsfysize=3.in
\centerline{\epsffile{Book13/fig15e.eps}}

For let $KG$ and $EG$ have been joined. And since
angle $KEG$ is a right-angle---on account of $KE$ also being at right-angles
to the plane $EG$, and manifestly also to the straight-line $EG$ [Def.~11.3]---the semi-circle drawn on $KG$ will thus also pass through
point $E$. Again, since $GF$ is at right-angles to each of $FL$
and $FE$, $GF$ is thus also at right-angles to the plane $FK$ [Prop.~11.4]. 
Hence, if we also join $FK$ then $GF$ will also be at right-angles to
$FK$. And, again, on account of this, the semi-circle drawn on
$GK$ will also pass through point $F$. Similarly, it will also pass through
the remaining (angular) points of the cube. So, if $KG$ remains (fixed), 
and the semi-circle is carried around, and again established at the
same (position) from which it began to be moved, then the cube will
have been enclosed by a sphere. So, I say that (it is) also (enclosed)
by the given (sphere). For since $GF$ is equal to $FE$, and the
angle at $F$ is a right-angle, the (square) on $EG$ is thus double
the (square) on $EF$ [Prop.~1.47]. And $EF$ (is) equal to
$EK$. Thus, the (square) on $EG$ is double the (square) on $EK$.
Hence, the (sum of the squares) on $GE$ and $EK$---that is to say, the
(square) on $GK$ [Prop.~1.47]---is three times the  (square) on $EK$. And since
$AB$ is three times $BC$, and as $AB$ (is) to $BC$, so the
(square) on $AB$ (is) to the (square) on $BD$ [Prop.~6.8, Def.~5.9], 
the (square) on $AB$ (is) thus three times the (square) on $BD$. 
And the (square) on $GK$ was also shown (to be) three times
the (square) on $KE$. And $KE$ was made equal to $DB$. 
Thus, $KG$ (is) also equal to $AB$. And $AB$ is the
radius of the given sphere. Thus, $KG$ is also equal to the diameter of the
given sphere.

Thus, the cube has been enclosed by the given sphere. And it
has simultaneously been shown that the square on the diameter of
the sphere is three times the (square) on the side of the cube.$^\dag$ (Which is)
the very thing it was required to show.
{\footnotesize\noindent$^\dag$ If the radius of the sphere is unity 
then the side of the cube is $\sqrt{4/3}$.}

%%%%
%13.16
%%%%
\pdfbookmark[1]{Proposition 13.16}{pdf13.16}

\begin{center}
{\large Proposition 16}
\end{center}

To construct an icosahedron, and to enclose (it) in a sphere, like the
aforementioned figures, and to show that the side of the
icosahedron is that irrational (straight-line) called minor.

\epsfysize=1.5in
\centerline{\epsffile{Book13/fig16ae.eps}}

Let the diameter $AB$ of the given sphere be laid out, and let it have been
cut at $C$ such that $AC$ is four times $CB$ [Prop.~6.10]. And
let the semi-circle $ADB$ have been drawn on $AB$. And let the straight-line $CD$ have been drawn from $C$ at right-angles to $AB$. And let
$DB$ have been joined. And let the circle $EFGHK$ be set down,
and let its radius be equal to $DB$.  And let the equilateral and
equiangular pentagon $EFGHK$ have been inscribed in circle
$EFGHK$ [Prop.~4.11]. And let the circumferences $EF$, $FG$,
$GH$, $HK$, and $KE$ have been cut in half at points $L$, $M$,
$N$, $O$, and $P$ (respectively). And let $LM$, $MN$, $NO$,
$OP$, $PL$, and $EP$ have been joined. Thus, pentagon $LMNOP$
is also equilateral, and $EP$ (is) the side of the decagon (inscribed in the circle).
And let the straight-lines $EQ$, $FR$, $GS$, $HT$, and $KU$, which
are equal to the radius of circle $EFGHK$,  have
been set up at right-angles to the plane of the circle, at points $E$, $F$, $G$, $H$, and $K$ (respectively). And let $QR$, $RS$, $ST$, $TU$,
$UQ$, $QL$, $LR$, $RM$, $MS$, $SN$, $NT$, $TO$,
$OU$, $UP$, and $PQ$ have been joined.

And since $EQ$ and $KU$ are each at right-angles to the same plane, $EQ$ is thus parallel to $KU$ [Prop.~11.6]. And it is also equal to it. And straight-lines joining equal and parallel (straight-lines) on the same side are
(themselves) equal and parallel [Prop.~1.33]. Thus, $QU$
is equal and parallel to $EK$. And $EK$ (is the side) of an equilateral
pentagon (inscribed in circle $EFGHK$). Thus, $QU$ (is) also  the
side of an equilateral pentagon inscribed in circle $EFGHK$.  So, for the
same (reasons), $QR$, $RS$, $ST$, and $TU$ are also  the sides of
an equilateral pentagon inscribed in circle $EFGHK$. Pentagon
$QRSTU$ (is) thus equilateral. And side $QE$
is (the side) of a hexagon (inscribed in circle $EFGHK$),
and $EP$ (the side) of a decagon, and (angle) $QEP$ is a right-angle, 
thus $QP$ is  (the side) of a pentagon (inscribed in the same circle).
For the square on the side of a pentagon is (equal to the sum of)
the (squares) on (the sides of) a hexagon and a decagon inscribed in the
same circle  [Prop.~13.10]. So, for the same (reasons), $PU$
is also the side of a pentagon. And $QU$ is also (the
side) of a pentagon. Thus, triangle $QPU$ is equilateral. So, for the
same (reasons), (triangles) $QLR$, $RMS$, $SNT$, and $TOU$
are each also equilateral. And since $QL$ and $QP$ were each shown
(to be the sides) of a pentagon,  and $LP$ is  also (the side) of a pentagon, triangle $QLP$ is thus equilateral. So,
for the same (reasons), triangles $LRM$, $MSN$, $NTO$, and
$OUP$ are each also equilateral. 

\epsfysize=3.5in
\centerline{\epsffile{Book13/fig16e.eps}}

Let the center, point $V$, of circle $EFGHK$ have been found [Prop.~3.1]. 
And let $VZ$ have been set up, at (point) $V$,  at right-angles to the plane of the
circle. And let it have been produced on the other side (of the circle), 
like $VX$. And let $VW$ have been cut off 
(from $XZ$ so as to be equal to the side) of a hexagon, and each of $VX$ and 
$WZ$ (so as to be equal to the side) of a decagon. And let
$QZ$,  $QW$, $UZ$, $EV$, $LV$, $LX$, and
$XM$ have been joined.

And since $VW$ and $QE$ are each at right-angles to the plane of the
circle, $VW$ is thus parallel to $QE$ [Prop.~11.6].  And they are also
equal. $EV$ and $QW$ are thus equal and parallel (to one another) [Prop.~1.33]. And $EV$ (is the side) of a hexagon. Thus, $QW$
(is) also (the side) of a hexagon. And since $QW$ is (the side) of a hexagon, 
and $WZ$ (the side) of a decagon, and angle $QWZ$ is a right-angle
[Def.~11.3, Prop.~1.29], 
$QZ$ is thus (the side) of a pentagon [Prop.~13.10]. So, for the same
(reasons), $UZ$ is also (the side) of a pentagon---inasmuch as, if we join
$VK$ and $WU$ then they will be equal and opposite. 
And $VK$, being (equal)
to the radius (of the circle), is (the side) of a hexagon [Prop.~4.15~corr.].
Thus, $WU$ (is) also the side of a hexagon.
 And $WZ$
(is the side) of a decagon, and (angle) $UWZ$ (is) a right-angle. 
Thus, $UZ$ (is the side) of a pentagon [Prop.~13.10]. And $QU$ is
also (the side) of a pentagon. Triangle $QUZ$ is thus equilateral. So,
for the same (reasons), each of the remaining triangles, whose bases
are the straight-lines $QR$, $RS$, $ST$, and $TU$, and apexes the point
$Z$, are also equilateral. Again, since $VL$ (is the side) of a hexagon, and $VX$ (the side) of a decagon, and angle $LVX$ is a right-angle, 
$LX$ is thus (the side) of a pentagon [Prop.~13.10]. So, for the same
(reasons), if we join $MV$, which is (the side) of a hexagon, $MX$
is also inferred (to be the side) of a pentagon. And $LM$ is also (the side)
of a pentagon. Thus, triangle $LMX$ is equilateral. So, similarly,
it can be shown that each of the remaining triangles, whose bases
are the (straight-lines) $MN$, $NO$, $OP$, and $PL$, and apexes the
point $X$,  are also equilateral. Thus, an icosahedron contained by
twenty equilateral triangles has been constructed.

So, it is also necessary to enclose it in the given sphere, and to show
that the side of the icosahedron is that irrational (straight-line) called minor.

For, since $VW$ is (the side) of a hexagon, and $WZ$ (the side) of
a decagon, $VZ$ has thus been cut in extreme and mean
ratio at $W$, and $VW$ is its greater piece  [Prop.~13.9]. Thus, as
$ZV$ is to $VW$, so $VW$ (is) to $WZ$. And $VW$ (is) equal
to $VE$, and $WZ$ to $VX$.  Thus, as $ZV$ is to $VE$,
so $EV$ (is) to $VX$. And angles $ZVE$ and
$EVX$ are right-angles. Thus, if we join straight-line $EZ$ then angle
$XEZ$ will be a right-angle, on account of the similarity of triangles
$XEZ$ and $VEZ$.  [Prop.~6.8]. So, for the same (reasons), since
as $ZV$ is to $VW$, so $VW$ (is) to $WZ$,  and $ZV$ (is) equal to
$XW$, and $VW$ to $WQ$, thus as $XW$ is to $WQ$, so $QW$
(is) to $WZ$. And, again, on account of this,  if we join $QX$ then the
angle at $Q$ will be a right-angle [Prop.~6.8]. Thus, the semi-circle
drawn on $XZ$ will also pass through $Q$ [Prop.~3.31]. And if $XZ$ remains
fixed, and the semi-circle is carried around, and again established at
the same (position) from which it began to be moved, then it will
also pass through (point) $Q$, and (through) the remaining (angular) points
of the icosahedron. And the icosahedron will have been enclosed by a sphere.
 So, I say that (it is) also (enclosed) by the given (sphere). 
 For let $VW$ have been cut in half at $a$. And since the straight-line $VZ$ has been cut in extreme and mean ratio at $W$, and  $ZW$ is its lesser piece,
 then the square on $ZW$ added to half of the greater piece, $Wa$, is
 five times the (square) on half of the greater piece [Prop.~13.3]. 
 Thus, the (square) on $Za$ is five times the (square) on $aW$. And
 $ZX$ is double $Za$, and $VW$ double $aW$. Thus,
 the (square) on $ZX$ is five times the (square) on $WV$. And since
 $AC$ is four times $CB$, $AB$ is thus five times $BC$. And as
 $AB$ (is) to $BC$, so the (square) on $AB$ (is) to the
 (square) on $BD$ [Prop.~6.8, Def.~5.9]. Thus, the (square)
 on $AB$ is five times the (square) on $BD$. And the (square) on 
 $ZX$ was also shown (to be) five times the (square) on $VW$. 
 And $DB$ is equal to $VW$. For each of them is equal to the
 radius of circle $EFGHK$.  Thus, $AB$ (is) also equal to
 $XZ$.  And $AB$ is the diameter of the given sphere. Thus,
 $XZ$ is equal to the diameter of the given sphere. Thus, the icosahedron
 has been enclosed by the given sphere.
 
 So, I say that the side of the icosahedron is that irrational (straight-line)
 called minor. For since the diameter of the sphere is rational, and the
 square on it is five times the (square) on the radius of circle
 $EFGHK$, the radius of circle $EFGHK$ is thus also rational.
 Hence, its diameter is also rational. And if an equilateral
 pentagon is inscribed in a circle having a rational diameter then
 the side of the pentagon is that irrational (straight-line) called minor [Prop.~13.11]. And the side of pentagon $EFGHK$ is (the side)
 of the icosahedron. Thus, the side of the icosahedron is that irrational
 (straight-line) called minor.


\begin{center}
{\large Corollary}
\end{center}

So,  (it is) clear, from this, that the square on the diameter of the sphere
is five times the square on the radius of the circle from which the icosahedron has been described,
and that the the diameter of the sphere is the sum of (the side) of the hexagon, 
and two of (the sides) of the decagon, inscribed in the same circle.$^\dag$ 
{\footnotesize\noindent$^\dag$ If the radius of the sphere is unity then the
radius of the circle is $2/\sqrt{5}$, and the sides of the hexagon, decagon,
and pentagon/icosahedron are $2/\sqrt{5}$, $1-1/\sqrt{5}$, and $(1/\sqrt{5})\,\sqrt{10-2\,\sqrt{5}}$, respectively.}

%%%%
%13.17
%%%%
\pdfbookmark[1]{Proposition 13.17}{pdf13.17}

\begin{center}
{\large Proposition 17}
\end{center}

To construct a dodecahedron, and to enclose (it) in a sphere, like the
aforementioned figures, and to show that the side of the dodecahedron 
is that irrational (straight-line) called an apotome.

\epsfysize=3in
\centerline{\epsffile{Book13/fig17e.eps}}

Let two planes of the aforementioned cube [Prop. 13.15], $ABCD$ and $CBEF$,  (which are) at right-angles to one another,
be laid out. And let the sides $AB$, $BC$, $CD$, $DA$, $EF$,
$EB$, and $FC$  have each been cut in half at points $G$, $H$, $K$, $L$,
$M$, $N$, and $O$ (respectively). And let $GK$, $HL$, $MH$, and
$NO$ have been joined. And let $NP$, $PO$, and $HQ$ have each been
cut in extreme and mean ratio at points $R$, $S$, and $T$ (respectively).
And let their greater pieces be $RP$, $PS$, and $TQ$ (respectively).
And let $RU$, $SV$, and $TW$ have been set up on the exterior side of the cube,  at points
$R$, $S$, and $T$ (respectively), at right-angles to the planes of the cube.
 And let them be made equal to $RP$,
$PS$, and $TQ$. And let
$UB$, $BW$, $WC$, $CV$, and $VU$ have been joined.

I say that the pentagon $UBWCV$ is equilateral, and in one plane, and,
further, equiangular. For let $RB$, $SB$, and $VB$ have been joined. 
And since the straight-line $NP$ has been cut in extreme and mean
ratio at $R$, and $RP$ is the greater piece,  the (sum of the squares)
on $PN$ and $NR$ is thus three times the (square) on $RP$ [Prop.~13.4].
And $PN$ (is) equal to $NB$, and $PR$ to $RU$. 
Thus, the (sum of the squares) on $BN$ and $NR$ is three times the
(square) on $RU$. And the (square) on $BR$ is equal to the (sum of the squares) on $BN$ and $NR$ [Prop.~1.47].  
Thus, the (square) on $BR$ is three times the (square) on $RU$. 
Hence, the (sum of the
squares) on $BR$ and $RU$ is four times the (square) on $RU$. And
the (square)
on $BU$ is equal to the (sum of the squares) on $BR$ and $RU$ [Prop.~1.47]. Thus, the (square) on $BU$ is four times the
(square) on $UR$.  Thus, $BU$ is double $RU$.  And
$VU$ is also double $UR$, inasmuch as $SR$ is also double $PR$---that is to say, $RU$. Thus, $BU$ (is) equal to $UV$. So, similarly, it can be shown
that  each of $BW$, $WC$, $CV$ is equal to each of $BU$ and $UV$. 
Thus, pentagon $BUVCW$ is equilateral. So, I say that it is also in one plane. For let $PX$ have been drawn from $P$, parallel to
each of $RU$ and $SV$, on the exterior side of the cube. And let
$XH$ and $HW$ have been joined. I say that $XHW$ is a straight-line.
For since $HQ$ has been cut in extreme and mean ratio at $T$,
and  $QT$  is its greater piece, thus as $HQ$ is to $QT$, so $QT$
(is) to $TH$. And $HQ$ (is) equal to $HP$, and $QT$ to each of
$TW$ and $PX$. Thus, as $HP$ is to $PX$, so $WT$ (is) to
$TH$. And $HP$ is parallel to $TW$. For of each of them is at right-angles
to the plane $BD$ [Prop.~11.6]. And $TH$ (is parallel) to
$PX$. For each of them is at right-angles to the plane $BF$ [Prop.~11.6].
And if two triangles, like
$XPH$ and $HTW$, having two sides proportional to two sides,
are placed together at a single angle  such that their corresponding sides are also parallel then the remaining sides will be straight-on (to one another) [Prop.~6.32]. Thus, $XH$ is straight-on to
$HW$. And every straight-line is in one plane [Prop.~11.1]. Thus,
pentagon $UBWCV$ is in one plane.

So, I say that it is also equiangular.

For since the straight-line $NP$ has been cut in extreme and mean ratio at
$R$, and $PR$ is the greater piece [thus as the sum of
$NP$ and $PR$ is to $PN$, so $NP$ (is) to $PR$], and $PR$
(is) equal to $PS$ [thus as $SN$ is to $NP$, so $NP$ (is) to $PS$], 
$NS$ has thus also been cut in extreme and mean ratio at $P$, and $NP$
is the greater piece [Prop.~13.5]. Thus, the (sum of the squares) on
$NS$ and $SP$ is three times the (square) on $NP$ [Prop.~13.4].
And $NP$ (is) equal to $NB$, and $PS$ to $SV$. Thus, the (sum of the)
squares on  $NS$ and $SV$ is three times the (square) on $NB$.
Hence, the (sum of the squares) on $VS$, $SN$, and $NB$
is four times the (square) on $NB$. And the (square) on $SB$
is equal to the (sum of the squares) on
$SN$ and $NB$  [Prop.~1.47].
Thus, the (sum of the squares) on $BS$ and $SV$---that is to say, 
the (square) on $BV$ [for angle $VSB$ (is) a right-angle]---is
four times the (square) on $NB$ [Def.~11.3, Prop.~1.47].
Thus, $VB$ is double $BN$. And $BC$ (is) also double $BN$.
Thus, $BV$ is equal to $BC$. And since the two (straight-lines)
$BU$ and $UV$ are equal to the two (straight-lines)
$BW$ and $WC$ (respectively), and the base $BV$ (is) equal to
the base $BC$, angle $BUV$ is thus equal to angle $BWC$ [Prop.~1.8].
So, similarly, we can show that angle $UVC$ is equal to angle
$BWC$.  Thus, the three angles
$BWC$, $BUV$, and $UVC$ are equal to one another.
And if three angles of an equilateral pentagon are equal to one another
then the pentagon is equiangular [Prop.~13.7].  Thus, pentagon
$BUVCW$ is equiangular. And it was also shown (to be)
equilateral. Thus, pentagon $BUVCW$ is equilateral and equiangular,
and it is on one of the sides, $BC$, of the cube. Thus, if we make the
same construction on each of the twelve sides of the cube then some solid
figure contained by twelve equilateral and equiangular pentagons 
will have been constructed,  which is called a dodecahedron.

So, it is necessary to enclose it in the given sphere, and to show that the
side of the dodecahedron is that irrational (straight-line) called an apotome.

For let $XP$ have been produced, and let (the produced straight-line) be $XZ$. Thus, $PZ$ meets
the diameter of the cube, and they cut one another in half.
For, this has been proved in the penultimate theorem of the eleventh book [Prop.~11.38]. Let them cut (one another) at $Z$. Thus, $Z$ is the
center of the sphere enclosing the cube, and $ZP$ (is) half the side of
the cube. So, let $UZ$ have been joined. And since the straight-line 
$NS$ has been cut in extreme and mean ratio at $P$, and its greater piece
is $NP$, the (sum of the squares) on $NS$ and $SP$
is thus three times the (square) on $NP$ [Prop.~13.4]. And $NS$ (is)
equal to $XZ$, inasmuch as $NP$ is also equal to $PZ$, and
$XP$ to $PS$. But, indeed, $PS$ (is) also (equal) to $XU$,
since (it is) also (equal) to $RP$. Thus, the (sum of the squares) on 
$ZX$ and $XU$ is three times the (square) on $NP$.
And  the (square)
on $UZ$ is equal to the (sum of the squares) on $ZX$ and $XU$ [Prop.~1.47].  Thus, the (square) on $UZ$ is three times the
(square) on $NP$. 
 And the square on the
radius of the sphere enclosing the cube is also three times the (square) on half
the side of the cube. For it has previously been demonstrated (how to) construct 
the cube, and to enclose (it) in a sphere, and to show that the square on
the diameter of the sphere is three times the (square) on the
side of the cube [Prop.~13.15]. And if the (square on the) whole (is three
times) the (square on the) whole, then the (square on the) half (is) also (three times) the (square on the) half. 
And $NP$ is half of the side of the cube. Thus, $UZ$ is equal to
the radius of the sphere enclosing the cube. And $Z$ is the center of the
sphere enclosing the cube.
Thus, point $U$
is on the surface of the sphere. So, similarly, we can show that each of
the remaining angles of the dodecahedron is also  on the surface of the sphere. 
Thus, the dodecahedron has been enclosed by the given sphere.

So, I say that the side of the dodecahedron is that irrational straight-line
called an apotome.

For since  $RP$ is the greater piece of $NP$, which has been cut in extreme and mean ratio, and $PS$ is the greater piece of $PO$, which
has been cut in extreme and mean ratio,  $RS$ is thus the greater piece of the
whole of $NO$, which has been cut in extreme and mean ratio.
[Thus, since as $NP$ is to $PR$, (so) $PR$ (is) to $RN$, and  (the same is also true) of the doubles.
For parts have the same ratio as similar multiples (taken in corresponding
order) [Prop.~5.15]. Thus, as $NO$ (is) to $RS$, so $RS$
(is) to the sum of $NR$ and $SO$. And $NO$
(is) greater than $RS$. Thus, $RS$ (is) also
greater than the sum of $NR$ and $SO$ [Prop.~5.14]. Thus, $NO$
has been cut in extreme and mean ratio, and  $RS$ is its greater piece.]
And $RS$ (is) equal to $UV$. Thus, $UV$ is the greater piece of
$NO$, which has been cut in extreme and mean ratio. 
And since the diameter of the sphere is rational, and the square on it
is three times the (square) on the side of the cube,  $NO$, which is
the side of the cube, is thus rational. And if a rational (straight)-line
is cut in extreme and mean ratio then each of the
pieces is the irrational (straight-line called) an apotome.

Thus, $UV$, which is the side of the dodecahedron, is the
irrational (straight-line called) an apotome [Prop. 13.6].



\begin{center}
{\large Corollary}
\end{center}

So, (it is) clear, from this, that the side of the dodecahedron
is the greater piece of the side of the cube, when it is cut in extreme
and mean ratio.$^\dag$ (Which is) the very thing it was required to show.
{\footnotesize\noindent$^\dag$ If the radius of the circumscribed sphere is
unity then the side of the cube is $\sqrt{4/3}$, and the side of the
dodecahedron is $(1/3)\,(\sqrt{15}-\sqrt{3})$.}

%%%%
%13.18
%%%%
\pdfbookmark[1]{Proposition 13.18}{pdf13.18}

\begin{center}
{\large Proposition 18}
\end{center}

To set out the sides of the five (aforementioned) figures, and to compare (them)
with one another.$^\dag$

\epsfysize=2.25in
\centerline{\epsffile{Book13/fig18e.eps}}

Let the diameter, $AB$, of the given sphere be laid out. And let it
have been cut at $C$, such that $AC$ is equal to $CB$, and at $D$, such
that $AD$ is double $DB$. And let the semi-circle $AEB$ have been
drawn on $AB$. And let $CE$ and $DF$ have been drawn from $C$
and $D$ (respectively), at right-angles to $AB$. And let $AF$, $FB$,
and $EB$ have been joined. And since $AD$ is double $DB$, 
$AB$ is thus triple $BD$. Thus, via conversion, $BA$ is one and
a half times $AD$. And as $BA$ (is) to $AD$, so the (square)
on $BA$ (is) to the (square) on $AF$ [Def.~5.9].  For
triangle $AFB$ is equiangular to triangle $AFD$ [Prop.~6.8]. 
Thus, the (square) on $BA$ is one and a half times the (square)
on $AF$. And the square on the diameter of the sphere
is also one and a half times the (square) on the side of the pyramid
[Prop.~13.13].  And $AB$ is the diameter of the sphere. Thus,
$AF$ is equal to the side of the pyramid.

Again, since $AD$ is double $DB$,  $AB$ is thus triple $BD$. And
as $AB$ (is) to $BD$, so the (square) on $AB$ (is) to the (square)
on $BF$ [Prop.~6.8, Def.~5.9]. Thus, the (square) on $AB$
is three times the (square) on $BF$.  And the square on the diameter of
the sphere is also three times the (square) on the side of the cube [Prop.~13.15]. And $AB$ is the diameter of the sphere. Thus, $BF$ is the
side of the cube.

And since $AC$ is equal to $CB$, $AB$ is thus double $BC$.  And
as $AB$ (is) to $BC$, so the (square) on $AB$ (is) to the (square)
on $BE$ [Prop.~6.8, Def.~5.9]. Thus, the (square) on $AB$
is double the (square) on $BE$. And the square on the diameter
of the sphere is also double the (square) on the side of the octagon [Prop.~13.14]. And $AB$ is the diameter of the given sphere. Thus,
$BE$ is the side of the octagon.

So let $AG$ have been drawn from point $A$ at right-angles to the straight-line
$AB$. And let $AG$ be made equal to $AB$. And let $GC$
have been joined. And let $HK$ have been drawn from $H$, perpendicular to $AB$. And since $GA$ is double $AC$. 
For $GA$ (is) equal to $AB$.  And as $GA$ (is) to $AC$, so
$HK$ (is) to $KC$ [Prop.~6.4]. $HK$ (is) thus also double
$KC$. Thus, the (square) on $HK$ is four times the (square)
on $KC$. Thus, the (sum of the squares) on $HK$ and $KC$,
which is the (square) on $HC$ [Prop.~1.47],  is five times the
(square) on $KC$.  And $HC$ (is) equal to $CB$. Thus,
the (square) on $BC$ (is) five times the (square) on  $CK$. 
And since $AB$ is double $CB$, of which $AD$ is double $DB$, 
the remainder $BD$ is thus double the remainder $DC$.
$BC$ (is) thus triple $CD$. The (square) on $BC$ (is) thus nine times 
the (square) on $CD$.  And the (square) on $BC$ (is) five times the (square)
on $CK$.  Thus, the (square) on $CK$ (is) greater than the (square) on $CD$.
$CK$ is thus greater than $CD$. Let $CL$ be made equal to $CK$.
And let $LM$ have been drawn from $L$ at right-angles to $AB$. 
And let $MB$ have been joined. And since the (square) on  $BC$
is five times the (square) on $CK$, and $AB$ is double $BC$, and $KL$
double $CK$, the (square) on $AB$ is thus five times the (square)
on $KL$. And the square on the diameter of the sphere  is also five times
the (square) on the radius of the circle from which the icosahedron has been
described [Prop.~13.16~corr.]. And $AB$ is the diameter of the sphere.
Thus, $KL$ is the radius of the circle from which the icosahedron has
been described. Thus, $KL$ is (the side) of the hexagon (inscribed) in the aforementioned
circle [Prop.~4.15~corr.]. And since the diameter of the sphere is composed of (the side) of the hexagon, and two of (the sides) of the decagon, inscribed
in the aforementioned circle, and $AB$ is the diameter of the sphere, 
and $KL$ the side of the hexagon, and $AK$ (is) equal to $LB$, thus
$AK$ and $LB$ are each sides of the decagon inscribed in the  circle
from which the icosahedron has been described. And since
$LB$ is (the side) of the decagon. And $ML$ (is the side) of the hexagon---for (it is) equal to $KL$, since (it is) also (equal) to $HK$, for they are equally
far from the center. And $HK$ and $KL$ are each double $KC$. 
$MB$ is thus (the side) of the pentagon (inscribed in the circle) [Props.~13.10, 1.47]. 
And (the side) of the pentagon is (the side) of the icosahedron [Prop.~13.16]. 
Thus, $MB$ is (the side) of the icosahedron. 

And since $FB$ is the side of the cube, let it have been cut in extreme and
mean ratio at $N$, and let $NB$ be the greater piece. Thus, $NB$
is the side of the dodecahedron [Prop.~13.17~corr.].

And since the (square) on the diameter of the sphere was shown (to be)
one and a half times the square on the side, $AF$,  of the pyramid,
and twice the square on (the side), $BE$,  of the octagon, and
three times the square on (the side), $FB$, of the cube, 
thus, of whatever (parts) the (square) on the diameter of the
sphere (makes) six, of such (parts) the (square) on (the side) of the pyramid (makes)
four, and (the square) on (the side) of the octagon three, and (the
square) on (the side) of the cube  two. Thus, the (square) on the
side of the pyramid is one and a third times the square on the side of the
octagon, and double the square on (the side) of the cube. 
And the (square) on (the side) of the octahedron is one and a half
times the square on (the side) of the cube. Therefore, the aforementioned
sides of the three figures---I mean, of the pyramid, and of the octahedron, and 
of the cube---are
in rational ratios to one another. And (the sides of) the remaining two (figures)---I mean, of
the icosahedron, and of the dodecahedron---are neither in rational ratios to
one another, nor to the (sides) of the aforementioned (three figures). For they are
irrational (straight-lines): (namely), a minor [Prop.~13.16], and 
an apotome [Prop.~13.17].

(And), we can show that the side, $MB$, of the icosahedron is greater
that the (side), $NB$, or the dodecahedron, as follows.

For, since triangle $FDB$ is equiangular to triangle $FAB$ [Prop.~6.8],
proportionally, as $DB$ is to $BF$, so $BF$ (is) to $BA$ [Prop.~6.4]. 
And since three straight-lines are (continually) proportional, as the
first (is) to the third, so the (square) on the first (is) to the (square) on the
second [Def.~5.9, Prop.~6.20~corr.]. Thus, as $DB$ is to $BA$, so the (square) on $DB$
(is) to the (square) on $BF$. Thus, inversely, as $AB$ (is) to $BD$, so the
(square) on $FB$ (is) to the (square) on $BD$. And $AB$ (is) triple
$BD$. Thus, the (square) on $FB$ (is) three times the (square)
on $BD$. And the (square) on $AD$ is also four times the (square)
on $DB$. For $AD$ (is) double $DB$. Thus, the (square) on $AD$
(is) greater than the (square) on $FB$.  Thus, $AD$ (is) greater than
$FB$. 
Thus, $AL$ is much greater than $FB$. And $KL$ is the greater piece of $AL$, which is cut in extreme and
mean ratio---inasmuch as $LK$ is (the side) of the hexagon, and $KA$
(the side) of the decagon [Prop.~13.9]. And $NB$
is the greater piece of $FB$, which is cut in extreme and mean ratio. 
Thus, $KL$ (is) greater than $NB$.  And $KL$ (is) equal to $LM$. 
Thus, $LM$ (is) greater than $NB$ [and $MB$ is greater than $LM$].
Thus, $MB$, which is (the side) of the icosahedron, is much greater
than $NB$, which is (the side) of the dodecahedron.  (Which is)
the very thing it was required to show.
{\footnotesize\noindent$\dag$ If the radius of the given sphere is unity 
then the sides of the pyramid ({\em i.e.}, tetrahedron), octahedron, cube,
icosahedron, and dodecahedron, respectively, satisfy the following
inequality: $\sqrt{8/3} > \sqrt{2}>\sqrt{4/3}>(1/\sqrt{5})\,\sqrt{10 -2\,\sqrt{5}}
> (1/3)\,(\sqrt{15}-\sqrt{3})$.}

~\\

So, I say that, beside the five aforementioned  figures,  no other (solid) figure
can be constructed (which is) contained by equilateral and equiangular (planes), equal to one another.

For a solid angle cannot be constructed from two triangles, or indeed (two) planes (of any sort) [Def.~11.11]. And (the solid angle) of the
pyramid (is constructed) from three (equiangular) triangles, and (that) of the octahedron
from four (triangles), and (that) of the icosahedron from (five) triangles. 
And a solid angle cannot be (made) from six equilateral and
equiangular triangles set up together at one point. For, since the angles of a
equilateral triangle are (each)  two-thirds of a right-angle, the (sum of the)
six  (plane) angles (containing the solid angle) will be four right-angles. The very thing (is) impossible.
For every solid angle is contained by (plane angles whose sum is) less than four right-angles [Prop.~11.21].  So, for the same (reasons), a solid angle cannot be constructed from more than six  plane angles (equal to two-thirds of a right-angle) either. And the (solid) angle of a cube
is contained by three squares. And (a solid angle contained) by four
(squares is) impossible. For, again, the (sum of the plane angles
containing the solid angle) will be four right-angles. And (the solid angle)
of a dodecahedron (is contained) by three equilateral and equiangular
pentagons. And (a solid angle contained) by four (equiangular
pentagons is) impossible. For, the angle of an equilateral
pentagon being one and one-fifth of right-angle, four (such) angles will be
greater (in sum) than four right-angles. The very thing (is) impossible. And, on account of the same absurdity, a solid angle cannot be constructed from any other
(equiangular) polygonal figures either.

Thus, beside the five aforementioned figures, no other solid figure can be constructed (which is) contained by equilateral and equiangular (planes).
(Which is) the very thing it was required to show.



\epsfysize=2in
\centerline{\epsffile{Book13/fig18ae.eps}}

\begin{center}\vspace*{-7pt}
{\large Lemma}
\end{center}\vspace*{-7pt}

It can be shown that the angle of an equilateral and
equiangular pentagon is one and one-fifth of a right-angle,
as follows.

For let $ABCDE$ be an equilateral and equiangular pentagon, and
let the circle $ABCDE$ have been circumscribed about it [Prop.~4.14]. 
And let its center, $F$, have been found [Prop.~3.1]. 
And let $FA$, $FB$, $FC$, $FD$, and $FE$ have been joined. Thus,
they cut the angles of the pentagon in half at (points)
 $A$, $B$, $C$, $D$, and $E$  [Prop.~1.4].
 And since the five
angles at $F$ are equal (in sum) to four right-angles, and
are also equal (to one another),  (any) one of them, like $AFB$, is thus
one less a fifth of a right-angle. Thus, the (sum of the) remaining
(angles in triangle $ABF$), $FAB$ and $ABF$,  is one plus a fifth  of a right-angle
[Prop.~1.32]. And $FAB$ (is) equal to $FBC$. Thus, the
whole angle, $ABC$, of the pentagon is also one and one-fifth
of a right-angle. (Which is) the very thing it was required to show.
